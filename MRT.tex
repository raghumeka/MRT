\documentclass[12pt]{article}
\usepackage{xspace}
\usepackage[colorlinks=true,linkcolor=red,citecolor=blue]{hyperref}
\usepackage{amssymb,amsmath,amsthm,graphicx,color,bbm}
\usepackage{thmtools, thm-restate}
\usepackage[margin=1.in]{geometry}
\usepackage{multirow,array}
\allowdisplaybreaks

\usepackage[capitalise,nameinlink]{cleveref}

\usepackage{algorithm}
\usepackage[noend]{algpseudocode}

%\floatname{algorithm}{Test}
\renewcommand{\algorithmicrequire}{\textbf{Input:}}
\renewcommand{\algorithmicensure}{\textbf{Output:}}
\renewcommand{\algorithmicindent}{1.0em}


%Makes a wider version of \bar{}
\makeatletter
\newcommand*\rel@kern[1]{\kern#1\dimexpr\macc@kerna}
\newcommand*\widebar[1]{%
  \begingroup
  \def\mathaccent##1##2{%
    \rel@kern{0.8}%
    \overline{\rel@kern{-0.8}\macc@nucleus\rel@kern{0.2}}%
    \rel@kern{-0.2}%
  }%
  \macc@depth\@ne
  \let\math@bgroup\@empty \let\math@egroup\macc@set@skewchar
  \mathsurround\z@ \frozen@everymath{\mathgroup\macc@group\relax}%
  \macc@set@skewchar\relax
  \let\mathaccentV\macc@nested@a
  \macc@nested@a\relax111{#1}%
  \endgroup
}
\makeatother


\newcommand{\restate}[2]{\medskip
\noindent{\bf #1 (restated).}{\sl #2}}


\theoremstyle{plain}
\newtheorem{theorem}{Theorem}[section]
\newtheorem{corollary}[theorem]{Corollary}
\newtheorem{proposition}[theorem]{Proposition}
\newtheorem{lemma}[theorem]{Lemma}
\newtheorem{claim}[theorem]{Claim}
\newtheorem{fact}[theorem]{Fact}
\newtheorem{conjecture}[theorem]{Conjecture}
\newtheorem{exercise}[theorem]{Exercise}
\newtheorem{question}[theorem]{Question}
\newtheorem*{claimNoNum}{Claim}
\newtheorem*{lemmaNoNum}{Lemma}
\newtheorem*{theoremNoNum}{Theorem}
\newtheorem*{questionNoNum}{Question}
\newtheorem{definition}[theorem]{Definition}
\newtheorem{example}[theorem]{Example}
\newtheorem{thm}{Theorem}
\newtheorem{open}[theorem]{Open Problem}

\theoremstyle{remark}
\newtheorem{remark}[theorem]{Remark}
\theoremstyle{plain}

\newenvironment{proofof}[1]{\begin{trivlist} \item {\bf Proof
#1:~~}}
  {\qed\end{trivlist}}


\def\R{{\mathbb{R}}}
\def\F{{\mathbb{F}}}
\def\C{{\mathbb{C}}}
\def\N{{\mathbb{N}}}
\def\Z{{\mathbb{Z}}}
\def\Q{{\mathbb{Q}}}

\renewcommand{\Pr}{\mathop{\bf Pr\/}}
\newcommand{\E}{\mathop{\bf E\/}}
\newcommand{\EE}[1]{\mathop{\bf E\/}\left[ {#1} \right]}
\def\var{{\mathop{\bf Var\/}}}
\def\cov{{\mathop{\bf Cov\/}}}
\newcommand{\Es}[1]{\mathop{\bf E\/}_{{\substack{#1}}}}
\newcommand{\Var}{\mathop{\bf Var\/}}
\newcommand{\Cov}{\mathop{\bf Cov\/}}

\newcommand{\abs}[1]{\left|#1\right|}

\def\Ber{{\mathrm{Ber}}}
\def\Bin{{\mathrm{Bin}}}
\def\pr{\Pr}

\newcommand{\one}{{\mathbbm{1}}}
\def\B{{\{0,1\}}}
\def\pmone{{\{\pm1\}}}
\def\sgn{\mathrm{sgn}}

\def\spn{{\mathrm{span}}}
\def\poly{{\mathrm{poly}}}
\def\OR{{\mathrm{OR}}}
\def\AND{{\mathrm{AND}}}
\def\Inf{{\mathrm{Inf}}}

\def\l{{\ell}}
\newcommand{\modulo}[1]{~(\mathrm{mod~}#1)}
\newcommand{\remove}[1]{}

\newcommand{\deff}{\triangleq}
\newcommand{\mathify}[1]{\ifmmode{#1}\else\mbox{$#1$}\fi}
\newcommand{\paramcomplexityclass}[2]{\mathify{{\bf #1}\left(#2\right)}\xspace}
\newcommand{\complexityclass}[1]{{\bf{#1}}\xspace}
\newcommand{\problemname}[1]{{\textsf{#1}}\xspace}
\newcommand{\scriptcomplexityclass}[3]{\mathify{{\bf #1}_{\bf #2}^{\bf #3}}\xspace}
\newcommand{\inner}[1]{\langle{#1}\rangle}

% We continue with standard complexity classes.
\renewcommand{\P}{\complexityclass{P}}
\newcommand{\NP}{\complexityclass{NP}}
\newcommand{\CONP}{\complexityclass{coNP}}
\newcommand{\BPP}{\complexityclass{BPP}}
\newcommand{\RP}{\complexityclass{RP}}
\newcommand{\CORP}{\complexityclass{coRP}}
\newcommand{\ZPP}{\complexityclass{ZPP}}
\newcommand{\MIP}{\complexityclass{MIP}}
\newcommand{\AM}{\complexityclass{AM}}
\newcommand{\ACzero}{\complexityclass{AC^0}}
\newcommand{\MA}{\complexityclass{MA}}
\newcommand{\DNF}{\complexityclass{DNF}}
\newcommand{\DNFs}{\complexityclass{DNFs}}
\newcommand{\RL}{\complexityclass{RL}}
\newcommand{\Ls}{\complexityclass{L}}


\newcommand{\codim}{\mathrm{codim}}
\newcommand{\mcp}{\mathrm{C_{min}^{\oplus}}}
\newcommand{\Rp}{\mathcal{R}_p}
\newcommand{\spar}{\mathrm{sparsity}}
\newcommand{\tail}{\mathrm{tail}}
\newcommand{\true}{\mathbf{true}}
\newcommand{\false}{\mathbf{false}}
\newcommand{\DT}{\mathrm{DT}}
\newcommand{\ADVpm}{\mathrm{ADV}^{\pm}}
\newcommand{\W}{\mathbf{W}}
\renewcommand{\star}{*}
\newcommand{\Ev}{\mathcal{E}_\nu}
\newcommand{\Evapp}{\tilde{\mathcal{E}}_\nu}
\newcommand{\m}{m}
\newcommand{\eps}{\varepsilon}
\newcommand{\rank}{\mathrm{rank}}
\newcommand{\FF}{\mathcal{F}}
\newcommand{\tildeO}{{\widetilde O}}

%% MACROS for this paper
\newcommand{\zo}{\{0,1\}}
\newcommand{\cB}{\mathcal{B}}
\newcommand{\cA}{\mathcal{A}}
\newcommand{\cM}{\mathcal{M}}
\newcommand{\cE}{\mathcal{E}}
\newcommand{\Eff}{\mathrm{Eff}}
\newcommand{\Dtagx}{{\mathcal{D}'_x}}
\newcommand{\Dx}{{\mathcal{D}_x}}

\newcommand{\D}{{\mathcal{D}}}
\newcommand{\error}{{\mathsf{err}}}

\newcommand{\Vsmall}{V_{\mathsf{small}}}

\renewcommand{\bar}{\widebar}
\renewcommand{\hat}{\widehat}
\renewcommand{\tilde}{\widetilde}

\newcommand{\Bias}{\mathsf{Bias}}


\renewcommand{\boldsymbol}{}
\newcommand{\authnote}[4]{{\bf [{\color{#3} #1's Note:} {\color{#4} #2}]}}
\newcommand{\Anote}[1]{\authnote{Avishay}{#1}{red}{blue}}
\newcommand{\Rnote}[1]{\authnote{Raghu}{#1}{red}{blue}}
\newcommand{\Onote}[1]{\authnote{Omer}{#1}{red}{blue}}
\newcommand{\ignore}[1]{}



\pretolerance=2000

\begin{document}

\title{Pseudorandom Generators for Width-3 Branching Programs}

\author{
Raghu Meka\\
\small UCLA\\
\and 
Omer Reingold\thanks{{\tt reingold@stanford.edu}. Supported in part by NSF grant CCF-1749750.
}\\
\small Stanford University\\
\and 
Avishay Tal\thanks{\texttt{avishay.tal@gmail.com}. Supported by a Motwani Postdoctoral Fellowship and by NSF grant CCF-1749750.
}\\
\small Stanford University\\
}

%\date{}
\maketitle

%%%%%%%%%%%%%%%%%%%%%%%%%%%%%%%%%%%%%%%%%%%%%%%%%%%
\begin{abstract}
For every $\eps>0$,
we construct a pseudorandom generator using $\tilde{O}(\log(n/\eps))$ that $\eps$-fools: (1) read-once polynomials on $n$ variables, (2) locally-monotone read-once branching programs (ROBPs) of length $n$ and width $3$, and (3) constant-width ROBPs of length $n$ with width-$2$ every at most $\poly\log(n)$ layers.

Our construction relies on the Ajtai-Wigderson paradigm \cite{AW85}, CHRT's results \cite{CHRT17} and Viola's \cite{Viola08} or Lovett's \cite{Lovett08} pseudorandom generators for low-degree polynomials.

Furthermore, for width-3 ROBPs programs we have two incomparable results:
\begin{enumerate}
	\item Based on the work of BRRY~\cite{BravermanRRY10}, we construct a pseudorandom generator $\eps$-fooling \textbf{ordered} ROBPs of width-$3$ and length-$n$ with seed length $\tilde{O}(\log(n) \cdot \log(1/\eps))$.
	\item Based on the work of CHHL~\cite{CHHL18}, we construct a pseudorandom generator $\eps$-fooling \textbf{unordered} ROBPs of width-$3$ and length-$n$ swith seed length $\tilde{O}(\log(n) \cdot \poly(1/\eps))$.
\end{enumerate}
This is the first improvement for width-$3$ ROBPs since the work of Nisan~\cite{Nisan92}.
\end{abstract}

%\thispagestyle{empty}

%\clearpage 
%\setcounter{page}{1}
%\tableofcontents
%%%%%%%%%%%%%%%%%%%%%%%%%%%%%%%%%%%%%%%%%%%%%%%%%%%

\newcommand{\Sel}{\mathrm{Sel}}
\section{Preliminaries}

Denote by $U_n$ the uniform distribution over $\pmone^n$, and by $U_S$ for $S\subseteq [n]$ the uniform distribution over $\pmone^S$. 
Denote by $\log$ the logarithm in base $2$. For any function $f: \pmone^n \to \R$, we shorthand by $\E[f] = \E_{x\sim U_n}[f(x)]$ and by  $\Var[f] = \E[f^2]-\E[f]^2$.
For an event $E$ we denote by $\one_{E}$ its indicator function.
\subsection{Restrictions}
For a set $T \subseteq [n]$ and two strings $x \in \pmone^T$, $y\in \pmone^{[n]\setminus T}$ we denote 
by $\Sel_T(x,y)$ the string with $$\Sel_T(x,y)_i = \begin{cases}x_i,& i\in T\\ y_i, &\text{otherwise.}\end{cases}$$
\begin{definition}[Restriction]\label{def:restriction}
  Let $f:\pmone^n\to\R$ be a  function. A {\sf restriction} is a pair $(T,y)$ where $T \subseteq [n]$ and $y \in \pmone^{[n]\setminus T}$. We denote by $f_{T|y}:\pmone^n \to \R$ the function $f$ restricted according to $(T,y)$, defined by $f_{T|y}(x) = f(\Sel_T(x,y))$.
%  \[
% f_{T|y}(x) = f(z), \;\;\;\;\text{where}\;\;\;\;\;z_i = \begin{cases}x_i,& i\in T\\ y_i, &\text{otherwise}\end{cases}.
%  \]
\end{definition}


\begin{definition}[Random Valued Restriction]\label{def:random_valued_restriction}
Let $n \in \N$.
A random variable $(T,y)$, distributed over restrictions of $\pmone^n$ is called {\sf random-valued} if conditioned on $T$, the variable $y$ is uniformly distributed over $\pmone^{[n]\setminus T}$.
\end{definition}


\begin{definition}[$p$-Random Restriction]\label{def:p_random_restriction}
  A {\sf $p$-random restriction} is a random-valued restriction over pairs $(T,y)$ sampled in the following way: For every $i\in[n]$, independently, pick $i$ to $T$ with
  probability $p$;  
  Sample $y$ uniformly from $\pmone^{[n]\setminus T}$.
  We denote this distribution of restrictions by $\Rp$.
\end{definition}

\begin{definition}[The Bias-Function]\label{def:bias_function}
Let $f: \pmone^n \to \R$.
Let $T \subseteq [n]$. 
We denote by $\Bias_T(f): \pmone^n \to  \R$ the function defined by $(\Bias_T(f))(x) = \E_{y\sim U_{[n]\setminus T}}[f_{T|y}(x)]$.
When $T$ is clear from the context, we shorthand $\Bias_T(f)$ as $\tilde{f}$.
\end{definition}

\subsection{Fourier Analysis of Boolean Functions}\label{subsec:Fourier}
Any function $f: \pmone^n \to \R$ has a unique Fourier representation:
\[ f(x) = \sum_{S\subseteq[n]} \hat{f}(S) \cdot \prod_{i\in S} x_i\;,\]
where the coefficients $\hat{f}(S) \in \R$ are given by $\hat{f}(S) = \E_{x\sim U_n} [f(x) \cdot \prod_{i\in S} x_i]$.
We have $\var[f] = \sum_{\emptyset \neq S \subseteq [n]}{\hat{f}(S)^2}$.
We denote the {\sf spectral-norm} of $f$ by $L_1(f) \triangleq \sum_{S \subseteq [n]} |\hat{f}(S)|$.
For any functions $f,g: \pmone^n \to \R$ it holds that $L_1(f\cdot g) \le L_1(f) \cdot L_1(g)$ where equality holds if $f$ and $g$ depends on disjoint sets of variables. Additionally, $L_1(f+g) \le L_1(f) + L_1(g)$.
The following fact relates the Fourier coefficients of a Boolean function and its bias-function.


\begin{fact}[\protect{\cite[Proposition~4.17]{OdonnellBook}}]\label{fact:bias-fnc-Fourier}
	Let $f: \pmone^n \to \R$ and $S, T \subseteq [n]$. Then,
$\widehat{(\Bias_T f)}(S) = \hat{f}(S) \cdot \one_{\{S \subseteq T\}}$
\end{fact}

\subsection{Small-Biased Distributions}
We say that a distribution $\D$ over $\pmone^n$ is {\sf $\delta$-biased}  if for any non-empty $S \subseteq [n]$ it holds that
$\abs{\E_{x\sim \D}[\prod_{i\in S} x_i]} \le \delta$. \cite{NaorNaor93,AlonGHP92,Ta-Shma17} show that $\delta$-biased distributions can be sampled using $O(\log(n/\delta))$ random bits. 

Let $p \in (0,1]$. We say that a distribution $\D_p$ over subsets of $[n]$ is {\sf $\delta$-biased with marginals~$p$} if for any non-empty $S \subseteq [n]$ it holds that
$\Pr_{T\sim \D_p} [S \subseteq T] = p^{|S|} \pm \delta.
$

\begin{claim}\label{claim:sampling-t}
Let $p = 2^{-a}$ for some integer $a>0$, 
let $\D$ be an $\eps$-biased distribution over $\pmone^{na}$.
Define $\D_p$ to be a distribution over subsets of $[n]$ as follows:
Sample $x\sim \D$. 
Output 
$T = 
\{i\in [n]: \bigwedge_{j\in [a]} (x_{(i-1)a + j} = 1)\}$.
Then $\D_p$ is $\eps$-biased with marginals $p$.
\end{claim}
\begin{proof}
For any fixed $S$, 
	the probability that $S \subseteq T$ is exactly the probability that $\bigwedge_{i\in S, j\in [a]} (x_{(i-1)a + j} = 1)$.
	In an $\eps$-biased distribution, the latter event happens with probability $2^{-a\cdot |S|} \pm \eps$ (See \cite{AlonGHP92}).
\end{proof}



\begin{claim}\label{claim:inclusion-exclusion}
If $\D_p$ is $\delta$-biased with marginals $p$, then for any disjoint $S,S'\subseteq [n]$ it holds that 
$\Pr_{T\sim \D_p}[S \cap T = \emptyset, S' \subseteq T] = (1-p)^{|S|}\cdot p^{|S'|} \pm \delta	\cdot 2^{|S|}$.
\end{claim}
\begin{proof}
	By inclusion-exclusion 
	\begin{align*}\Pr_{T \sim \D_p}[S \cap T = \emptyset, S' \subseteq T] &= \sum_{R \subseteq S} (-1)^{|R|} \cdot \Pr_{T \sim \D_p}[R \cup S' \subseteq T] \\
	&= \sum_{R \subseteq S} (-1)^{|R|} \cdot (\Pr_{T \sim U}[R \cup S' \subseteq T] \pm \delta)\\
	&= \Pr_{T \sim U}[S \cap T = \emptyset, S' \subseteq T] \pm 2^{|S|}\cdot \delta.\qedhere\end{align*}
\end{proof}


\subsection{Standard tail bounds for $k$-wise independence}

\begin{lemma}[\protect{\cite[Thm.~4, restated]{SchmidtSS95}}]
\label{lemma:tail_bounds}
Let $\ell$ be an even positive integer.
	Let $X_1, \ldots, X_m$ be some $\ell$-wise independent random variables bounded in $[-1,1]$ with expectation $0$.
	Let $V = \sum_{i=1}^{m}\var[X_i]$.
	Then,
	$\E[(X_1 + \ldots +X_m)^{\ell}] \le \max\{\ell^\ell, (\ell V)^{\ell/2}\}.$
	%\le \ell^\ell + (\ell V)^{\ell/2}.$$ 
\end{lemma}


%\begin{corollary}
%	Let $\ell$ be an even positive integer.
%Let $Y_1, \ldots, Y_m$ be random variables.
%Let $X_1, \ldots, X_m$ be some $\ell$-wise independent random variables bounded in $[-1,1]$ with expectation $0$.
%Let $V = \sum_{i=1}^{m}\var[X_i]$.
%Suppose for all $i_1, \ldots, i_\ell \in [m]$ not necessarily distinct we have $|\E[X_{i_1} \cdots X_{i_\ell}] - \E[Y_{i_1} \cdots Y_{i_\ell}]| \le \eps$.
%Then,
%	$\Pr[|\sum_{i=1}^{m} Y_i| \ge t] \le \frac{\max\{\ell^\ell, (\ell V)^{\ell/2}\} + \eps \cdot m^{\ell}}{t^{\ell}}.$
%\end{corollary}
%\begin{proof}
%		By the assumption $|\E[X_{i_1} \cdots X_{i_\ell}] - \E[Y_{i_1} \cdots Y_{i_\ell}]| \le \eps$  we get 
%		$$\E[(Y_1 + \ldots +Y_m)^{\ell}] \le \E[(X_1 + \ldots +X_m)^{\ell}] + \eps m^{\ell}.$$
%		By Lemma~\ref{lemma:tail_bounds} we get 
%		$\E[(Y_1 + \ldots +Y_m)^{\ell}] \le \max\{\ell^\ell, (\ell V)^{\ell/2}\} + \eps \cdot m^{\ell}.$
%		Applying Markov's inequality promised gives the  bound on $
%\Pr[|\sum_{i=1}^{m} Y_i| \ge t] = 	
%\Pr[(\sum_{i=1}^{m} Y_i)^{\ell}  \ge t^{\ell}]$
%\end{proof}

	



\subsection{Branching Programs}
A {\sf read-once branching program (ROBP)} $B$ of {\sf length} $n$ and {\sf width} $w$ is a directed layered graph with $n+1$ layers of vertices denoted $V_1, \ldots, V_{n+1}$.  Each $V_i$ consists of $w_i \le w$ vertices $\{v_{i,1}, \ldots, v_{i,w_i}\}$, and between every two consecutive layers $V_i$ and $V_{i+1}$ there exists a set of directed edges (from $V_i$ to $V_{i+1}$), denoted $E_i$, such that any vertex in $V_i$ has precisely two out-going edges in $E_i$, one marked by $1$ and one marked by $-1$. %$V_1$ consists of a single vertex, denoted $v_{1,1}$.  
$V_{n+1}$  vertices are marked with either `accept' and `reject'.

A branching program $B$ and an input $x\in \pmone^n$ naturally describes a {\sf computation~path} in the layered graph:
we start at node $v_1 = v_{1,1}$ in $V_1$.
For $i=1, \ldots, n$, we traverse the edge going out from $v_i$ marked by $x_i$ to get to a node $v_{i+1} \in V_{i+1}$.
The resulting computation path is $v_1 \to v_2 \to \ldots \to v_{n+1}$.
We say that $B$ accepts $x$ iff the computation path defined by $B$ and $x$ reaches an accepting node. Naturally $B$ describes a Boolean function $B:\pmone^n \to \pmone$ whose value is $-1$ on input $x$ iff $B$ accepts $x$.

{\sf Unordered branching programs} are defined similarly, expect that there exists a permutation $\pi \in S_n$ such that in step $i$ the computation path follows the edge marked by $x_{\pi_i}$, for $i\in [n]$.
We also consider unordered branching programs on $[n]$ of shorter length $n'\le n$. In such case, the program stops after reading $n'$ input bits. \footnote{Note that since we are in the unordered case, the set of bits being read could be an arbitrary subset of $[n]$ of size $n'$.}

For two programs $B_1$ and $B_2$ defined over disjoint sets of variables and having the end width of $B_1$ equal the start width of $B_2$, we denote by $B_1 \circ B_2$ the concatenation of $B_1$ and $B_2$, defined in the natural way.




The following is a restatement of a result from \cite{CHRT17}. We give its proof for completeness in Appendix~\ref{app:CHRT}.
\begin{theorem}\label{thm:CHRTa}
Let $B$ be an unordered oblivious read-once branching programs with width-$w$ and length-$n$. Let $\eps>0$, $p \le 1/O(\log n)^w$, $k = O(\log(n/\eps))$, and $\D_p$ be a $\delta_T$-biased distribution over subsets of $[n]$ with marginals $p$, 
for some $\delta_T \le p^{2k}$.
Then,
with probability at least $1-\eps$ over $T\sim \D_p$, 
\[
L_1(\tilde{B}) =  \sum_{S \subseteq T}{ |\hat{B}(S)|} \le O((nw)^3/\eps).\]
\end{theorem}

\begin{theorem}[Implied by \protect{\cite[Thm.~2]{CHRT17}} and \protect{\cite[Thm.~4.1]{SteinkeVW14}}]\label{thm:CHRT}
Let $\mathcal{C}$ be the class of all unordered oblivious read-once branching programs on $[n]$ of length at most $n'$ and width at most $w$.
Then,
there exists an explicit pseudorandom generator 
\[
\mathbf{CHRT}:\pmone^{s_{n,n',w,\eps}}\to \pmone^n
\]
that $\eps$-fools $\mathcal{C}$, 
where $s_{n,n',w, \eps} = O(\log(n')^{w+1} \log\log(n') \log(n/\eps))$.
\end{theorem}


\subsection{Helpful Lemmas}

\begin{lemma}\label{lemma:var}
Let $a,b>0$.
If $X$ is a real-valued random variable bounded in $[-b,a]$ with mean $0$, then $\var[X] \le ab$.
\end{lemma}
	\begin{proof}
		$\var[X] = \E[X^2]$ since $\E[X]=0$. As $x^2$ is convex and $X$ domain is bounded, the maximal value that $\E[X^2]$ can get is if all of $X$'s probability mass is on the boundary. 
		Denote by $p = \Pr[X=a]$.
		Since $\E[X]=0$ we get $0 = p\cdot a + (1-p) \cdot (-b)$, i.e., $p = b/(a+b)$, thus 
		\[ \var[X] = \E[X^2] \le a^2 p + (1-p)b^2 = \frac{a^2 b}{a+b} + \frac{ab^2}{a+b} = ab\;.\qedhere\]
	\end{proof}


\begin{theorem}[Hyper-contractivity of Variance]\label{thm:HC}
Let $f: \pmone^k \to \pmone$ be a Boolean function.
Then, 
$\E_{T\sim \Rp}[\var[\tilde{f}]] \le p \cdot \var[f]$. Furthermore, if $p \le 1/3$, then
$\E_{T\sim \Rp}[\var[\tilde{f}]] \le \var[f]^{3/2}$.
%$\E_{T\sim \Rp}[\var[\tilde{f}]] \le \var[f]^{2/(1+p)}$.
\end{theorem}

\begin{proof}
First, observe that using Fact~\ref{fact:bias-fnc-Fourier} and $\var[g] = \sum_{S\neq \emptyset} \hat{g}(S)^2$  we have
$$\E_{T\sim \Rp}[\var[\tilde{f}]] = 
\E_{T\sim \Rp}\bigg[\sum_{S\neq \emptyset} \widehat{(\Bias_T f)}(S)^2\bigg] = 
\sum_{S\neq \emptyset} \hat{f}(S)^2 \cdot \Pr_{T\sim \Rp}[S \subseteq T] = 
\sum_{S\neq \emptyset} {p^{|S|} \cdot \hat{f}(S)^2 }.$$
For the first item, we have $\E_{T\sim \Rp}[\var[\tilde{f}]] = \sum_{S\neq \emptyset} {p^{|S|} \cdot \hat{f}(S)^2 } \le p \cdot \sum_{S\neq \emptyset} {\hat{f}(S)^2} = p\cdot \var[f]$.


For the second item, we use the Hyper-contractivity Theorem \cite{Bonami70} (cf. \cite[Ch.~9]{OdonnellBook}) stating that $\|N_{\rho} g\|_2 \le \|g\|_{1+\rho^2}$ for any function $g:\pmone^n \to \R$ (where $N_{\rho}$ is the noise operator that satisfies $\hat{N_{\rho} g}(S) = \rho^{|S|} \cdot \hat{g}(S)$ for all $S\subseteq [n]$).
%(where $(N_{\rho} g)(x) = \sum_{S \subseteq [n]} \rho^{|S|} \cdot \hat{g}(S) \cdot \prod_{i\in S} x_i$ is the noise operator applied to $g$).
Take $g = f- \E[f]$ and $\rho = \sqrt{p}$.
Then, \looseness=-1
$$
\E_{T\sim \Rp}[\var[\tilde{f}]] = \sum_{S\neq \emptyset} {p^{|S|} \cdot \hat{f}(S)^2 } =  
\|N_{\sqrt{p}} g\|^{2}_2 \le \|g\|_{1+p}^{2} = \E_{x\sim U_k}[|g(x)|^{1+p}]^{2/(1+p)}
$$
We analyze the RHS.
Let $\beta = \E[f]$. % and assume without loss of generality that $\beta\ge 0$. 
Then, $\beta\in [-1,1]$, $\var[f] = 1-\beta^2$, and under the uniform distribution $|g(x)|$ gets value $|1-\beta| = 1-\beta$ with probability $(1+\beta)/2$ and value $|-1-\beta| = 1+\beta$ with probability $(1-\beta)/2$.
We get 
\begin{align*}
	\E_{x\sim U_k}[|g(x)|^{1+p}]
	&= \frac{1+\beta}{2} \cdot (1-\beta)^{1+p} + \frac{1-\beta}{2} \cdot (1+\beta)^{1+p} \\
	&= (1-\beta^2) \cdot (\tfrac{1}{2} (1-\beta)^{p} + \tfrac{1}{2}(1+\beta)^{p}) 
	\le 1-\beta^2 = \var[f]
\end{align*}
where the inequality follows by concavity of $x \mapsto x^{p}$.
%
%
%Let $\alpha = \var[f]$.
%Since $f$ is Boolean $\E[f]^2 = 1-\alpha$.
%Without loss of generality $\E[f] = \sqrt{1-\alpha}$ and $f$ gets value $1$ with probability $(1+\sqrt{1-\alpha})/2$ and value $-1$ with probability $(1-\sqrt{1-\alpha})/2$.
%We get 
%$$
%\E_x[|g(x)|^{4/3}]= \frac{1+\sqrt{1-\alpha}}{2} \cdot |(1-\sqrt{1-\alpha})^{4/3}| + \frac{1-\sqrt{1-\alpha}}{2} \cdot |(-1-\sqrt{1-\alpha})^{4/3}| \le \alpha
%$$
%where the last inequality was verified with wolfram alpha.
Overall if $p\le 1/3$, then 
$\E_{T\sim \Rp}[\var[\tilde{f}]]  \le   \var[f]^{2/(1+p)} \le  \var[f]^{3/2}$.
\end{proof}






\begin{lemma}\label{lemma:vars_prod}
	Suppose $\D_p$ is $\delta_T$-biased distribution with marginals $p$.
Let $\ell \in \N$.
Let $f_1, \ldots, f_\ell: \pmone^n \to \R$ be real valued functions, not necessarily distinct.
Then, $$\abs{\E_{T \sim \D_p}\left[ \prod_{i=1}^{\ell} \var[\tilde{f_i}] \right] - 
\E_{T \sim \Rp}\left[ \prod_{i=1}^{\ell} \var[\tilde{f_i}]\right]} \;\le  \;\delta_T \cdot\prod_{i=1}^{\ell} \var[f_i].$$
\end{lemma}
\begin{proof}
Using Fact~\ref{fact:bias-fnc-Fourier}, for any fixed $T$, we have
\begin{align*}
\prod_{i=1}^{\ell} \var[\tilde{f_i}] &= 
\prod_{i=1}^{\ell} \sum_{S_i \neq \emptyset} \hat{f_i}(S_i)^2 \cdot \one_{\{S_i \subseteq T\}}	= \sum_{S_1, \ldots, S_{\ell} \neq \emptyset} \hat{f_1}(S_1)^2 \cdots \hat{f_\ell}(S_\ell)^2 \cdot \one_{\{S_1 \cup \ldots \cup S_{\ell} \subseteq T\}}.
\end{align*}
Thus,
\begin{align*}
\E_{T \sim \D_p}\left[ \prod_{i=1}^{\ell} \var[\tilde{f_i}]\right] = \sum_{S_1, \ldots, S_{\ell} \neq \emptyset} \hat{f_1}(S_1)^2 \cdots \hat{f_\ell}(S_\ell)^2 \cdot (p^{|S_1 \cup \ldots \cup S_{\ell}|} \pm \delta_T)
\end{align*}
and 
\begin{align*}
\E_{T \sim \Rp}\left[ \prod_{i=1}^{\ell} \var[\tilde{f_i}]\right] = \sum_{S_1, \ldots, S_{\ell} \neq \emptyset} \hat{f_1}(S_1)^2 \cdots \hat{f_\ell}(S_\ell)^2 \cdot p^{|S_1 \cup \ldots \cup S_{\ell}|}.
\end{align*}
The difference between the two is at most 
$$\left|\E_{T \sim \D_p}\left[ \prod_{i=1}^{\ell} \var[\tilde{f_i}]\right] -  \E_{T \sim \Rp}\left[ \prod_{i=1}^{\ell} \var[\tilde{f_i}]\right]\right| \le \delta_T \cdot \sum_{S_1, \ldots, S_{\ell} \neq \emptyset} \hat{f_1}(S_1)^2 \cdots \hat{f_\ell}(S_\ell)^2 = \delta_T \cdot \prod_{i=1}^{\ell} \var[f_i],$$
which completes the proof.
%
%
%Using Fact~\ref{fact:bias-fnc-Fourier}, for any fixed $T$, we have
%\begin{align*}
%\prod_{i=1}^{\ell} \var[\tilde{f_i}] &= 
%\prod_{i=1}^{\ell} \sum_{S_i \neq \emptyset} \hat{f_i}(S_i)^2 \cdot \one_{\{S_i \subseteq T_i\}}	= \sum_{S_1, \ldots, S_{\ell} \neq \emptyset} \hat{f_1}(S_1)^2 \cdots \hat{f_\ell}(S_\ell)^2 \cdot \one_{\{S_1 \cup \ldots \cup S_{\ell} \subseteq T\}}
%\end{align*}
%Thus,
%\begin{align*}&\left|\E_{T \sim \D_p}\left[ \prod_{i=1}^{\ell} \var[\tilde{f_i}]\right] -  \E_{T \sim \Rp}\left[ \prod_{i=1}^{\ell} \var[\tilde{f_i}]\right]\right| 
%\\&\le  \sum_{S_1, \ldots, S_{\ell} \neq \emptyset} \hat{f_1}(S_1)^2 \cdots \hat{f_\ell}(S_\ell)^2 \cdot \abs{\Pr_{T\sim \D_p}[S_1 \cup \ldots S_\ell \subseteq T] - \Pr_{T\sim \R_p}[S_1 \cup \ldots S_\ell \subseteq T]} \\
%& \le \delta_T \cdot \sum_{S_1, \ldots, S_{\ell} \neq \emptyset} \hat{f_1}(S_1)^2 \cdots \hat{f_\ell}(S_\ell)^2 \\
%&\le \delta_T \cdot \prod_{i=1}^{\ell} \var[f_i],\end{align*}
%which completes the proof.
%
\end{proof}
%\newpage


\section{From width-3 ROBPs to the XOR of short ROBPs}
In Section 3, we prove the following main theorem.
\begin{thm}\label{thm:main}
Let $n,w,b\in \N$, $\eps>0$. There exists an explicit pseudorandom restriction assigning $p = 1/O(\log(b \cdot \log(n/\eps)))^{2w}$ fraction of  $n$ variables using $O(w \cdot \log (n/\eps) \cdot (\log\log(n/\eps) + \log(b)))$ random bits, that maintains the acceptance probability of any XOR of ROBPs of width-$w$ and length-$b$ up to error $\eps$.
\end{thm}
The pseudorandom restriction assigns $p$ fraction of the variables as follows: 
\begin{enumerate}
	\item Choose a set of coordinates $T \subseteq [n]$ according to a $\delta_T$-biased distribution with marginals $p$, for $\delta_T := p^{ O(\log(n/\eps))}$.
	\item Assign the variables in $T$ according to a $\delta_x$-biased distribution, for $\delta_x := (\eps/n)^{O(\log b)}$.
\end{enumerate}
Known constructions of small-biased distributions \cite{NaorNaor93,AlonGHP92,Ta-Shma17} show that it suffices to use $O(\log(n/\delta_T) + \log(n/\delta_x)) \le O(w \cdot \log (n/\eps) \cdot (\log\log(n/\eps) + \log(b)))$ random bits to sample the restriction.

In this section, we show how to design pseudorandom restrictions for unordered width-3 ROBPs from pseudorandom restrictions to the XOR of many width-3 ROBPs of length $O(\log (n/\eps))$.
%
%In this section, we reduce the case of width-3 ROBPs to the case of XOR of width-3 ROBPs of length $O(\log (n/\eps))$ using a pseudorandom restriction leaving each variable alive with probability $1/2$. 
We get the following theorem.

\begin{thm}
\label{thm:main_two_steps}
Let $n\in \N, \eps>0$. There exists an explicit pseudorandom restriction assigning $p = 1/O(\log \log(n/\eps))^{6}$ fraction of the variables using $O(\log (n/\eps) \log\log(n/\eps))$ random bits, that maintains the acceptance probability of any unordered width-$3$ length-$n$ ROBP up to error $\eps$.
\end{thm}


\paragraph{Proof Sketch.}
In this section, we shall show that under pseudorandom restrictions leaving each variable alive with probability $1/2$, with high probability, the bias function of a ROBP $B$ can be written as a linear combination (up to a small error) over functions of the form $f_1 \cdot f_2 \cdot \ldots \cdot f_m$ where each $f_i$ is a  short subprogram of the original program of length $O(\log (n/\eps))$, and each $f_i$ is defined on a disjoint set of coordinates. Each function $g$ in the linear combination will have a weight $\alpha_g \in [-1,1]$, and the sum of absolute values of weights over all functions participating in the linear combination will be at most $n$. This will show that any generator that $\eps/n$-fools the XOR of short width-$3$ ROBPs also $\eps$-fools width-$3$ length-$n$ ROBPs under random restrictions.

The reduction will first establish that with high probability (over the choice of the set of coordinates that are left alive) the bias function of a ROBP $B$ can be written as the average of width-$3$ length-$n$ ROBPs, whose vast majority have at most $O(\log (n/\eps))$ layers between every two layers with width-$2$. 
Then, we use a result of Bogdanov et al.~\cite{BogdanovDVY13} that  reduces branching programs with many width-$2$ layers to the XOR of short ROBPs.

We focus on the first part of the reduction. 
First, consider the case when $B$ is locally-monotone. 
In this case, every layer of edges is either the identity layer or a colliding layer \cite{BrodyV10}. Assume without loss of generality that there are no identity layers.
Then, under a pseudorandom restriction, with high probability, in every $O(\log (n/\eps))$ consecutive layers we will have a layer of edges whose corresponding variable is fixed to the value on which the edges in the layer collide, leaving at most $2$ vertices reachable in the next layer of vertices. Removing unreachable vertices, we get that with high probability under the random restriction, in every $O(\log (n/\eps))$ consecutive layers there is a layer of vertices with width-$2$.

However, in the case that $B$ is not locally-monotone (e.g.,  when $B$ is a permutation ROBP) it could the case that the widths of all layers of vertices remain $3$ under the random restriction.
Our main observation is that since the bias function takes the average over all assignments to the restricted variables, the bias function of $B$ does not depend on the labels of edges marked by the restricted variables. 
More formally, for any $T \subseteq [n]$, if $B$ and $C$ are two ROBPs with the same graph structure that only differ on the labels on the edges in layers $[n]\setminus T$, then $\Bias_T(B) = \Bias_T(C)$.
Thus, once $T$ is fixed we may relabel the layers in $[n]\setminus T$ so that they are locally-monotone, yielding a new ROBP $B'$, and then apply the bias function.
Using the analysis of the locally monotone case, we get that the bias function of $B'$ (and thus the bias function of $B$) is the average of $B'$ over all restrictions fixing the coordinates in $[n]\setminus T$, and we know that most of these restricted ROBPs have width-$2$ in every $O(\log (n/\eps))$ consecutive layers.

Essentially, the bias function allows us to imagine as if we are taking the average over restrictions of $B'$ rather than restrictions of $B$, and restrictions of $B'$ are ``simpler'' to fool than restrictions of $B$ since they have many layers with width-$2$.


The formal argument follows.

\begin{theorem}[From width-$3$ to almost width-$2$]\label{thm:the-bias-trick}
%Let $C>0$ be a large enough constant.
Let $B$ be a ROBP of width-$3$ and length-$n$. 
Let $\eps>0$. 
Let $\D_{1/2}$ be a $(\eps/n)^{10}$-biased distribution over subsets of $[n]$ with marginals $1/2$.
Let $T\sim \D_{1/2}$ be a random variable.
Let $B^T$ be the branching program $B$ where the layers in $[n] \setminus T$ are relabeled so that they are locally monotone.
Then, 
$$
\Bias_T(B)(x) = \Bias_T(B^T)(x) = \E_{y\sim U_{[n]\setminus T}}[(B^{T}_{T|y}(x)]
$$
and with  probability at least $1-\eps$ over the choice of $T$ and $y$, $B^{T}_{T|y}$ can be computed by a ROBP of the form $D_1 \circ \ldots \circ D_m$ where $\{D_i\}_{i=1}^{m}$ are defined over disjoint sets of at most $b = (3\log(n/\eps))$ variables, and each $D_i$ is a width-$3$ ROBP with at most $2$ vertices on the first and last layers.
\end{theorem}

\begin{proof}
We first observe that $\Bias_T(B)(x) = \Bias_T(B^T)(x)$.
Indeed, for any fixed $x$, $\Bias_T(B)(x)$ equals the probability that the following random-path in $B$ accepts:
\begin{quote}
Initiate $v_1$ to be the start node of $B$. For $i=1, \ldots, n$ if $i \in T$, take the edge exiting $v_i$ marked by $x_i$, otherwise (i.e., if $i\in [n]\setminus T$) pick a random edge out of the two edges exiting $v_{i}$. Denote by $v_{i+1}$ the node at the end of the edge taken in the $i$-th step. Accept if and only if $v_{n+1}$ is an accepting node.\end{quote}
 Observe that the following random process is oblivious to the labels of edges in layers $[n]\setminus T$, thus it would yield the same probability for $\Bias_T(B)$ and for $\Bias_T(B^{T})$. Overall, we got that $\Bias_T(B)$ and $\Bias_T(B^T)$ are equal as functions.

In the remainder of the proof, we analyze $\Bias_T(B^T)$.
Let $E_{i,1}$ and $E_{i,-1}$ denote the set of edges in the $i$-layer of $B$ marked by $1$ and $-1$ respectively.
We assume without loss of generality that in all layers of edges  $E_{i,1} \neq E_{i,-1}$, as otherwise the $i$-th layer is redundant and may be eliminated.
(Observe that under any relabeling of $B$ this property is preserved.)
By the collision lemma of Brody-Verbin~\cite{BrodyV10}, for any $i\in [n] \setminus T$, layer $i$ in $B^T$ has the following property: either $E_{i,1}$ or $E_{i,-1}$ has at most $2$ end-vertices.

Next, we consider the program $B^{T}_{T|y}$ for a pseudorandom $T$ and a random $y\in \pmone^{[n]\setminus T}$.
For $i=1, \ldots, n$ we say that the $i$-th layer of edges is ``good'' under the choice of $T$ and $y$, if $i \in [n] \setminus T$ and layer $E_{i,y_i}$ of $B^T$ has at most $2$ end-vertices.
Let $b = 3 \log(n/\eps)$.
For $i=1, \ldots, n-b+1$ let $\cE_i$ be the event that none of layers $\{i,i+1, \ldots, i+(b-1)\}$ is good.
Since $T$ is sampled from a $(\eps/n)^{10}$-biased distribution with marginals $1/2$, we have that $T$ is $(\eps/2n)$-almost $b$-wise independent.
Thus, up to an error of $\eps/2n$ we may analyze the event $\cE_i$ under uniform choice of a subset $T\subseteq[n]$.
Indeed, under a uniform choice for $T$ and $y$ each layer is good with probability at least $1/4$, and all $b$ layers are not good with probability at most $(3/4)^{b}$.
Overall, we get $\Pr[\cE_i] \le (3/4)^{b} + (\eps/2n) \le \eps/n$.
By the union bound, $$\Pr[\cE_1 \vee \cE_2 \vee \ldots \vee \cE_{n-b+1}] \le (n-b+1) \cdot (\eps/n) \le \eps.$$
Under the event that all $\cE_i$ are false, we get that $B^T_{T|y}$ has width $2$ in every $b$ layers. %(in fact, we may further eliminate all layers in $[n]\setminus T$ from the program, which can only make things better).
In such a case, we may write the restricted function $B^T_{T|y}$ as $D_1 \circ \ldots \circ D_m$ where each $D_i$ is a width-$3$ and length at most $b$ ROBP with at most $2$ vertices on the first and last layer.
\end{proof}

\begin{theorem}[from almost width-2 to the XOR of short ROBPs  - restatement of \protect{\cite[Thm.~2.1]{BogdanovDVY13}}]
	\label{thm:BDVY}
	%Let $b\in \N$.
	Let $B$ be a ROBP of the form $D_1\circ \ldots \circ D_m$  where $\{D_i\}_{i=1}^{m}$ are defined over disjoint sets of %at most $b$ 
	variables, and each $D_i$ is a width-$3$ ROBP with at most $2$ vertices on the first and last layers.
	Then, (as a real-valued function) $B$ can be written as a linear combination of 
	$\sum_{\alpha \in \B^m} c_{\alpha} \cdot \prod_{i=1}^{n} D_{i,\alpha_i}$
	where $D_{i,0}, D_{i,1}$ are subprograms of $D_{i}$ and $\sum_{\alpha\in \B^n}{|c_\alpha|} \le m$.
	\end{theorem}


\begin{proof}[Proof of Theorem~\ref{thm:main_two_steps}]
We prove that the following pseudorandom restriction maintains the acceptance probability of ROBPs of width-$3$ and length-$n$ up to error $\eps$.
Let $\eps_1 := \eps/2$, $\eps_2 := \eps/2n$.
\begin{enumerate}
	\item Pick $T_0 \subseteq [n]$ using a $(\eps_1/n)^{10}$-biased distribution with marginals $1/2$.
	\item
	\begin{enumerate}
	\item Pick $T\subseteq T_0$ using a $\delta_T$-biased distribution with marginals $p = 1/O(\log \log ( n/\eps_2))^{6}$.
	\item Assign the coordinates in $T$ using a $(\eps_2/n)^{O(\log \log (n/\eps_2))}$-biased distribution $\D_x$.
	\end{enumerate}
\end{enumerate}
%
Equivalently, we prove that the following distribution $\eps$-fools ROBPs of width-$3$ and length-$n$.
\begin{enumerate}
	\item Pick $T_0 \subseteq [n]$ using a $(\eps_1/n)^{10}$-biased distribution with marginals $1/2$.
	\item Assign the coordinates in $[n]\setminus T_0$ uniformly at random.
	\item
	\begin{enumerate}
	\item Pick $T\subseteq T_0$ using a $\delta_T$-biased distribution with marginals $p = 1/O(\log \log ( n/\eps_2))^{6}$.
	\item  Assign the coordinates in $T_0\setminus T$ uniformly at random.
	\item Assign the coordinates in $T$ using a $(\eps_2/n)^{O(\log \log (n/\eps_2))}$-biased distribution $\Dx$.
	\end{enumerate}
\end{enumerate}

Let $y\sim U_{[n]\setminus T_0}$. Let ${\cal G}$ be the event that $B^{T_0}_{T_0|y}$  can be computed by a ROBP of the form $D_1\circ \ldots \circ D_m$  where $\{D_i\}_{i=1}^{m}$ are defined over disjoint sets of at most $b=3\log(n/\eps_1)$ variables, and each $D_i$ is a width-$3$ ROBP with at most $2$ vertices on the first and last layers.
By Theorem~\ref{thm:the-bias-trick} $\Pr({\cal G}) \ge 1-\eps_1$. 
Assuming that ${\cal G}$ happened, then by Theorem~\ref{thm:BDVY},
$B^{T_0}_{T_0|y}$ can be written as 
$\sum_{\alpha \in \B^m} c_{\alpha} \cdot \prod_{i=1}^{n}D_{i,\alpha_i}$
	where $D_{i,\alpha_i}$ are subprograms of $D_{i}$ and $\sum_{\alpha\in \B^n}{|c_\alpha|} \le m$.
For each $\alpha\in \B^m$, using Theorem~\ref{thm:main} we have that 
$$
\left| 
\E_{z\sim U_{T_0}}\left[\prod_{i=1}^{m}D_{i,\alpha_i}(z)\right]
- \E_{T}\E_{x\sim \Dx}\E_{y' \sim U_{T_0 \setminus T}} \left[\prod_{i=1}^{m}D_{i,\alpha_i}(\Sel_{T}(x, y'))\right]
 \right| 
 \le \eps_2\;.
$$
By linearity of expectation and the triangle inequality
$$
\bigg|\E_{z\sim U_{T_0}}\bigg[\sum_{\alpha} c_{\alpha} \cdot \prod_{i=1}^{m}D_{i,\alpha_i}(z)\bigg] - \E_{T}\E_{x\sim \Dx}\E_{y' \sim U_{T_0 \setminus T}}
 \bigg[\sum_{\alpha} c_{\alpha} \cdot \prod_{i=1}^{m} D_{i,\alpha_i}(\Sel_{T}(x, y'))\bigg]\bigg|$$
$$\le \sum_{\alpha}{|c_{\alpha}|} \cdot  \eps_2 \;\le\; m \cdot \eps_2 \;\le\; \eps/2
$$
%
Overall, we get
%
$$\bigg|\E_{z\sim U_n}[B(z)]-
\E_{\substack{T_0,\\y\in U_{\bar{T_0}}}}\;
\E_{\substack{T, x\sim \Dx\\y' \sim U_{T_0 \setminus T}}}\;
[B(\Sel_{T_0}(\Sel_T(x,y'),y)]\bigg| =
$$
$$ 
\bigg|\E_{z\sim U_n}[B(z)]-
\E_{\substack{T_0,\\y\in U_{[n]\setminus T_0}}}\;
\E_{\substack{T, x\sim \Dx\\y' \sim U_{T_0 \setminus T}}}\;\
[B^{T_0}(\Sel_{T_0}(\Sel_T(x,y'),y)]\bigg| = 
$$
\begin{equation}\label{eq:aa}
	\bigg|\E_{\substack{T_0,\\y\in U_{[n]\setminus T_0}}}\;
\E_{\substack{T, z\sim U_{T} \\y' \sim U_{T_0 \setminus T}}}\;\
[B^{T_0}(\Sel_{T_0}(\Sel_T(z,y'),y)]-
\E_{\substack{T_0,\\y\in U_{[n]\setminus T_0}}}\;
\E_{\substack{T, x\sim \Dx\\y' \sim U_{T_0 \setminus T}}}\;\
[B^{T_0}(\Sel_{T_0}(\Sel_T(x,y'),y)]\bigg|
\end{equation}
where the last equality is due to the fact for any $T,T_0$ the distribution of $\Sel_{T_0}(\Sel_T(z,y'),y)$ is the uniform distribution over $\pmone^{n}$.
We bound Expression~\eqref{eq:aa} by 
%
\begin{align*}
	&\E_{T_0,y\in U_{[n]\setminus T_0}}\left[\left|\E_{T} \E_{y' \sim U_{T_0 \setminus T}} \left(\E_{z\sim U_{T}} [B^{T_0}_{T_0|y}(\Sel_T(z,y'))]- \E_{x\sim \Dx} [B^{T_0}_{T_0|y}(\Sel_T(x,y'))]\right)\right|\right]
	\\&\le \Pr\left[\neg {\cal G}\right]  + \E_{T_0,y\in U_{[n]\setminus T_0}}\left[\Big|\E_{T,y' \sim U_{T_0 \setminus T}} \Big(\E_{z\sim U_{T}} [B^{T_0}_{T_0|y}(\Sel_T(z,y'))]- \E_{x\sim \Dx} [B^{T_0}_{T_0|y}(\Sel_T(x,y'))]\Big)\Big| \; \bigg| \; {\cal G} \right]
	\\& \le \eps/2 + \eps/2
\end{align*}
%
where the second summand is bounded by $\eps/2$ according to the above discussion using Theorem~\ref{thm:BDVY} and Theorem~\ref{thm:main}.
	\end{proof}

\newcommand{\Tvar}{\mathbf{{Tvar}}}

\section{Pseudorandom restrictions for the XOR of short ROBPs}
In this section, we prove Theorem~\ref{thm:main}. 
Let $B_1, \ldots, B_m$ be pairwise disjoint subsets of $[n]$, each of size at most $b$. For $i=1, \ldots, m$ let $f_i : \pmone^{B_i} \to \pmone$ be a width $w$ ROBP. We construct a pseudorandom generator that $\eps$-fools $f = \prod_{i=1}^{m}{f_i}$.
We recall the statement of Theorem~\ref{thm:main} and the construction.
\begin{theorem}Let $n,w,b\in \N$, $\eps>0$. There exists an explicit pseudorandom restriction assigning $p = 1/O(\log(b \cdot \log(n/\eps)))^{2w}$ fraction of  $n$ variables using $O(w \cdot \log (n/\eps) \cdot (\log\log(n/\eps) + \log(b)))$ random bits, that maintains the acceptance probability of any XOR of ROBPs of width-$w$ and length-$b$ up to error $\eps$.
\end{theorem}

Recall that the pseudorandom restriction assigns $p$ fraction of the variables as follows: 
%\begin{quote}
\begin{enumerate}
	\item Choose a set of coordinates $T \subseteq [n]$ according to a $\delta_T$-biased distribution with marginals $p$, for $\delta_T := p^{ O(\log(n/\eps))}$.
	\item Assign the variables in $T$ according to a $\delta_x$-biased distribution, for $\delta_x := (\eps/n)^{O(\log b)}$.
\end{enumerate}
%\end{quote}

\paragraph{Analysis.} We shall assume without loss of generality that for all $i=1, \ldots, m$  it holds that $\E[f_i] \ge 0$.
We shall also assume without loss of generality that for all $i=1, \ldots, m$ it holds that $\var[f_i] > 0$ (i.e., that the functions are non-constant). 
Since the functions $f_i$ are Boolean and depend on at most $b$ bits, 
we have $\var[f_i] =\Pr[f_i=1]\cdot \Pr[f_i=-1] \ge  2^{-b}\cdot(1-2^{-b}) \ge 2^{-1-b}$.

We partition the functions into $O(\log b)$ buckets according to their variance.
Let $\sigma_0 = 1$, 
for every $j \in \{1,\ldots, \log_{1.1}(b+1)\}$, let $\sigma_j = 2^{-1.1^j}$ and
$I_j = \{i\in [m]: \var[f_i] \in (\sigma_{j},\sigma_{j-1}]\}$.
Let $C>0$ be a sufficiently large constant.
We consider two cases in our analysis: \begin{description}
\item[Low-Variance Case:] 
For every $j \in \{1,\ldots, \log_{1.1}(b+1)\}$ we have
$$\sum_{i\in I_j} \var[f_i] \le C \cdot \log^2(n/\eps)/(\sigma_{j-1})^{0.1}\;.$$
\item[High-Variance Case:] 
There exists a $j \in \{1,\ldots, \log_{1.1}(b+1)\}$ with $$\sum_{i\in I_j} \var[f_i] > C\cdot \log^2(n/\eps)/(\sigma_{j-1})^{0.1}\;.$$
\end{description}

\paragraph{Setting Up Parameters:}
Let $C'>1$ be a sufficiently large constant.
Set 
\begin{align}
\label{eq:delta_T}
	&\delta_{T} \triangleq p^{2C' \cdot \log(n/\eps)},\\
\label{eq:delta}
&\delta \triangleq  (\eps/n)^{10 C'}, \\
\label{eq:delta'_x}
	&\delta'_x \triangleq (\eps/n)^{100 C'},\\
\label{eq:delta_x}
	&\delta_x \triangleq  (\delta'_x)^{\log_{1.1}(b+1)}.
\end{align}
%Whenever we write $\poly(\eps/n)$, think of it as $(\eps/n)^c$ for a sufficiently large constant $c>1$.

\subsection{Low-Variance Case}
For $j = 1, \ldots, \log_{1.1}(b+1)$, 
let $F_j(x) = \prod_{i \in I_j} f_j(x)$. 
Thus, $f = \prod_{j} F_j$.
Let $\D_p$ be any $\delta_T$-biased distribution with marginals $p$. 
For $j\in \{1,\ldots,\log_{1.1}(b+1)\}$, we shall show that with probability at least $1-\eps/2n$ over the choice of $T\sim \D_p$, it holds that
\begin{equation}\label{eq:n/40}
\abs{\E_{x\sim \Dtagx}[\tilde{F_j}(x)] - 
\E_{z\sim U_T}[\tilde{F_j}(z)]}  \le \eps/n^{40}\;,\footnote{recall that we denote by $\tilde{g} = \Bias_T(g)$ for any function $g$.}
\end{equation}
for any $\delta'_x$-biased distribution $\Dtagx$ over $\pmone^{n}$.
Thus, by union bound Eq.~\eqref{eq:n/40} holds
for all $j \in \{1, \ldots, \log_{1.1}(b+1)\}$ simultaneously with probability at least $1-\eps/2$ over $T\sim\D_p$.
%
Using the following XOR lemma for small-biased distributions from \cite{GopalanMRTV12} we get that any 
 $(\delta'_x)^{\log_{1.1}(b+1)}$-biased distribution, fools $\tilde{f}(x) = \prod_{j=1}^{\log_{1.1}(b+1)}{\tilde{F_j}(x)}$ with error at most $16^{\log_{1.1}(b+1)}\cdot 2(\eps/n^{40})\le \eps/2$ (using $b\le n$).
%
\begin{lemma}[\protect{\cite[Thm.~4.1]{GopalanMRTV12}}, restated]\label{lemma:3.2}
	Let $0 < \eps< \delta\le 1$.
 	Let $F_1, \ldots, F_k : \pmone^n \to [-1,1]$ be functions on disjoint input variables such that each $F_i$ is $\delta$-fooled by any $\eps$-biased distribution.
 	Let $H:[-1,1]^k \to [-1,1]$ be a multilinear function in its inputs.
 	Then $H(F_1(x), \ldots, F_k(x))$ is $(16^k \cdot 2\delta)$-fooled by any $\eps^k$-biased distribution.
\end{lemma}
In Appendix~\ref{app:GMRTV}, we show how to derive Lemma~\ref{lemma:3.2} from \cite[Thm.~4.1]{GopalanMRTV12}.

\medskip
\noindent
In the remainder of this section, we focus on fooling a single $F_j$, that is, fooling the product (i.e., XOR) of functions $\{f_i\}_{i\in I_j}$ for which  $\var[f_i] \in (\sigma_{j}, \sigma_{j-1}]$.
We note that since we are in the ``Low-Variance Case'', then 
 \begin{equation} \label{eq:m}
 |I_j| \le C \cdot \sigma_{j}^{-1} \cdot \sigma_{j-1}^{-0.1}\cdot  \log^2(n/\eps)\;.
 \end{equation}
We handle two cases depending on whether $\sigma_{j-1}$ is big or not.


\paragraph{The case of $\sigma_{j-1} \ge 1/ (C \cdot \log(n/\eps))^{20}$ :}
In this case there are at most $O(\sigma_{j-1}^{-1.2}\cdot \log^2(n/\eps)) \le \poly\log(n/\eps)$ functions in $I_j$, each computed by a width-$w$ ROBP on at most $b$ bits. Thus, $F_j := \prod_{i\in I_j} f_i$ can be computed by a ROBP of length at most $n' = b\cdot \poly\log(n/\eps)$ and width at most $2w$. 
Using Theorem~\ref{thm:CHRTa} on $F_j$ (which has length $n'$ and width $2w$), with probability at least $1-\delta$ the spectral-norm of $\tilde{F_j}$ is at most $O((n' w)^3/\delta)$, thus any $\delta'_x$-biased distribution $O(\delta'_x \cdot (n' w)^3/\delta)$-fools  $\tilde{F_j} = \prod_{i \in I_j}{\tilde{f_i}(x)}$. For a large enough choice for $C'$, $O(\delta'_x \cdot (n' w)^3/\delta) \le \eps/n^{40}$ and we are done. 


\paragraph{The case of $\sigma_{j-1} < 1/ (C \cdot \log(n/\eps))^{20}$ :}
In this case all variances in $I_j$ are certainly smaller than $0.5$, and hence for all $i\in I_j$, we have $\E[f_i]^2 = \E[f_i^2]-\var[f_i] = 1-\var[f_i] \in [0.5,1]$.
Let $$\mu_i = \E[f_i]\qquad\text{and}\qquad g_i(x) \triangleq \frac{f_i(x)}{\mu_i} - 1.$$
Then, $$\prod_{i}{f_i(x)} = \prod_{i}{\mu_i} \cdot (1+g_i(x)).$$
We have $\E[g_i] = 0$ and $\Var[g_i] = \Var[f_i]/\mu_i^2 \in [\var[f_i], \var[f_i] \cdot 2]$.
We will show that with high probability over $T$, any $\delta'_x$-biased distribution fools $\prod_{i}{\mu_i} \cdot \prod_{i}{(1+\tilde{g_i}(x))}$. 

For ease of notation, in this case we think of $I_j$ as $[m]$ and denote by $\sigma = \sigma_{j-1}$.
The proof strategy for this part follows the work of Gopalan and Yehudayoff \cite{GopalanY14}. 
We note that 
$$\prod_{i=1}^{m}{(1+\tilde{g_i}(x))} = 1+\sum_{k=1}^{m} S_k(\tilde{g_1}(x), \tilde{g_2}(x), \ldots, \tilde{g_{m}}(x)),$$ 
where $S_k$ is the $k$-symmetric polynomial given by $S_k(y_1, \ldots, y_m) = \prod_{R \subseteq [m], |R|=k}{\prod_{i \in R} y_i}$.
We show that $x$ and $T$ fool the low-degree symmetric polynomials. Then, the following theorem by Gopalan and Yehudayoff \cite{GopalanY14} bootstraps this to show that $x$ and $T$ also fool the sum of all high-degree symmetric polynomials.

\begin{theorem}[Gopalan-Yehudayoff Tail Inequalities \cite{GopalanY14}]\label{thm:GY}
Let $y_1, \ldots, y_{m} \in \R$.
Suppose 
$|S_{\ell}(y_1, \ldots, y_{m})| \le \frac{t^{\ell}}{\sqrt{\ell!}}$ 
and 
$|S_{\ell+1}(y_1, \ldots, y_m)| \le \frac{t^{\ell+1}}{\sqrt{(\ell+1)!}}$
for some $t$ and $\ell$. 
Then, for every 
$k \in \{\ell, \ldots, {m}\}$ 
it holds that 
$|S_{k}(y_1, \ldots, y_{m})| \le (6et)^{k} \cdot (\ell/k)^{k/2}$.
Furthermore, if $6et \le 1/2$, then
$$\sum_{k=\ell}^{{m}} |S_{k}(y_1, \ldots, y_{m})| \le 2\cdot (6et)^{\ell}.$$
\end{theorem}

\subsubsection*{Analyzing the Symmetric Polynomials}
From Eq.~\eqref{eq:m} and our assumption that 
$\sigma < 1/(C\cdot \log(n/\eps))^{20}$ 
we get that $m \le \sigma^{-1.3}$.
Recall that $C'$ is a sufficiently large constant and recall the definition of $\delta, \delta'_x, \delta_T$ from Eqs.~\eqref{eq:delta_T}, \eqref{eq:delta} and \eqref{eq:delta'_x}.
We set 
\begin{equation}
\ell \triangleq C' \cdot \log(n/\eps)/ \log(1/\sigma)
\end{equation}
%and assume that $\ell$ is an even integer.
In the following, we shall use the facts that $\sigma^{-\ell}, m^{\ell} \ll 1/\delta$ and $\delta'_x \ll \delta$.

\begin{claim}\label{claim:good}
Let $T \sim \D_p$.
Let $R \subseteq [m]$ be a set of size at most $\ell$.
Then, with probability at least $1-O(b \ell w)^3 \cdot \delta$ over the choice of $T$,
$\prod_{i\in R} \tilde{f_i}(x)$ has spectral-norm at most $1/\delta$.
\end{claim}
\begin{proof}
Note that $\prod_{i\in R} f_i(x)$ can be computed by a ROBP with length $b \cdot \ell \le O(b \cdot \log(n/\eps))$ and width $2w$ (as in the case where $\sigma_j$ is big).
Apply Theorem~\ref{thm:CHRTa} to $\prod_{i\in R} f_i(x)$. 
\end{proof}

We say that $T\subseteq[n]$ is a {\sf good} set if for all sets $R \subseteq [{m}]$ of  size at most $\ell$, 
the spectral-norm of $\prod_{i\in R}{\tilde{f}_i}$ is at most $1/\delta$.
We observe that by Claim~\ref{claim:good}, the probability that $T$ is good is at least 
$1-({m}+1)^{\ell}\cdot O(b\ell w)^3 \cdot \delta \ge 1-\eps/10n$ (using Eq.~\eqref{eq:m} and \eqref{eq:delta}).
\begin{claim}\label{claim:S_k small spectral}
If $T$ is good, then for any $k\le \ell+1$, 
$S_{k}(\tilde{g_1}, \tilde{g_2},\ldots, \tilde{g_{m}})$ has  spectral-norm at most 
$\delta^{-1}\cdot (4{m})^{k} \le \delta^{-2}$.
\end{claim}

\begin{proof}
We expand the $k$-symmetric polynomial:
$S_{k}(\tilde{g_1}(x),\ldots, \tilde{g_m}(x)) = \sum_{R \subseteq [m], |R|=k} \prod_{r\in R} \widetilde{g_r}(x)$.
Since $T$ is good, each summand has spectral-norm
\begin{align*} 
L_1\bigg(\prod_{r\in R}  \widetilde{g_r}(x)\bigg)
&=L_1\Bigg(\prod_{r\in R}  \Big(\frac{\widetilde{f_r}(x)}{\E[f_r]} -1\Big)\Bigg) 
\le L_1\bigg(\sum_{Q \subseteq R}  (-1)^{|R|-|Q|} \prod_{r \in Q} \frac{\widetilde{f_r}(x)}{\E[f_r]}\bigg) 
\le 2^{k} \cdot \delta^{-1}\cdot 2^{k}\;,
\end{align*}
(using $\E[f_r] \ge 1/2$).
Summing over all $\binom{m}{k} \le m^k$ summands completes the proof.
\end{proof}

We wish to show that with high probability the total variance under restrictions $\sum_{i}\var[\tilde{f}_i]$ is small. Towards this goal, we prove a bound on the $\ell$-th moment of the total variance.
\begin{claim}\label{claim:Var_fi ell}
$
\E_{T\sim \D_p}[(\sum_{i=1}^m \var[\tilde{f}_i])^{\ell}] \le 
 2 \cdot (2\sigma^{0.2})^\ell
$
\end{claim}
\begin{proof}
Fix $(i_1, \ldots, i_{\ell})\in [m]^{\ell}$, not necessarily distinct indices. By Lemma~\ref{lemma:vars_prod}
\begin{align*}
	\E_{T\sim \D_p}\left[\prod_{j=1}^{\ell} \Var[\tilde{f_{i_j}}]]\right] 
	&\le \E_{T\sim \Rp}\left[\prod_{j=1}^{\ell}\Var[\tilde{f_{i_j}}]\right] 
	+ \delta_T \cdot\prod_{j=1}^{\ell} \Var[{f_{i_j}}],
\end{align*}
from which we deduce
$$
\E_{T\sim \D_p}\left[\Big(\sum_{i=1}^m \var[\tilde{f}_i(z)]\Big)^{\ell}\right] \le 
 \E_{T\sim \Rp}\left[\Big(\sum_{i=1}^m \var[\tilde{f}_i(z)]\Big)^{\ell}\right] + \delta_T \cdot m^{\ell} \sigma^{\ell} \;.
$$
We are left to bound $\E_{T\sim \Rp}[(\sum_{i=1}^m \var[\tilde{f}_i])^{\ell}]$. 
By Fact~\ref{fact:bias-fnc-Fourier}, for any $i\in[m]$, the random variable $X_i = \Var[\tilde{f_i}]/\var[f_i]$ (whose value depends on the choice of $T\sim \Rp$) is bounded in $[0,1]$.
By Theorem~\ref{thm:HC}, its expected value is at most $\Var[f_i]^{0.5} \le \sigma^{0.5}$.
Taking $X = \sum_{i=1}^{m}X_i$, we get that $X$ is the sum of $m$ independent random variables bounded in $[0,1]$.
Using $m \le \sigma^{-1.3}$, we have that $\E[X] \le \sigma^{0.5} \cdot m \le \sigma^{-0.8}$.
Thus, by Chernoff's bounds, with probability at least $1-\exp(-\Omega(\sigma^{-0.8}))$
we have
$X \le 2\cdot \sigma^{-0.8}$.
In such a case 
$\sum_{i} \var[\tilde{f_i}] 
\le 2 \cdot \sigma^{-0.8} \cdot \sigma 
\le 2\sigma^{0.2}$.
We get $\E_{T\sim \Rp}[(\sum_{i=1}^m \var[\tilde{f}_i])^{\ell}] \le \exp(-\Omega(\sigma^{-0.8}))\cdot (\sigma m)^{\ell} + 
(2\sigma^{0.2})^{\ell}$, which gives
\[\E_{T\sim \D_p}\left[\Big(\sum_{i=1}^m \var[\tilde{f}_i]\Big)^{\ell}\right]  \le 
\delta_T \cdot m^{\ell} \sigma^{\ell}  + \exp(-\Omega(\sigma^{-0.8}))\cdot (\sigma m)^{\ell} + 
(2\sigma^{0.2})^{\ell} \le 2\cdot (2 \sigma^{0.2})^{\ell}.\qedhere
\]
\end{proof}


We say that a set $T\subseteq [n]$ is {\sf excellent} if $T$ is good and $\sum_{i}\var[\tilde{g}_i] \le \sigma^{0.1}$.

\begin{claim}
$\Pr_{T\sim \D_p}[\text{$T$ is not excellent}] \le \eps/10n + O(\sigma)^{0.1\ell} \le \eps/2n$	
\end{claim}
\begin{proof}
Note that $\sum_{i}\Var[\tilde{g_i}] \le 2\sum_{i}\Var[\tilde{f_i}]$ and apply Markov's inequality on $(2\sum_{i}\Var[\tilde{f_i}])^{\ell}$ using Claim~\ref{claim:Var_fi ell}.
\end{proof}

\begin{claim}\label{claim:symmetric-excellent}
Let $T$ be an excellent set. Let $\Dtagx$ be any $\delta'_x$-biased distributions.
Then, for $k = 1, \ldots, \ell+1$ we have
$$\E_{x\sim \Dtagx}[ S_{k}^2(\tilde{g}_1(x), \ldots, \tilde{g}_m(x))] \le  \frac{2}{k!} \cdot \sigma^{0.1 k}$$
and $$\left|\E_{x \sim \Dtagx}[S_{k}(\tilde{g}_1(x), \ldots, \tilde{g}_m(x))]\right| \le (\eps/n)^{C'}.$$
\end{claim}

\begin{proof}
	Recall that $\delta = (\eps/n)^{10C'}$ and $\delta'_x = (\eps/n)^{100C'}$. The first claim relies on the following:
	 \begin{enumerate}
		\item $S_k^2$ has small spectral-norm (using Claim~\ref{claim:S_k small spectral}, since $T$ is good) and hence is fooled by $\Dtagx$. In details, its spectral-norm is at most $L_1(S_k)^2 \le \delta^{-4}$ and $\Dtagx$ is $\delta'_x$-biased. Thus $$\Big|\E_{x\sim U_n}[S_k^2(\tilde{g}_1(x), \ldots, \tilde{g}_m(x))] - \E_{x\sim \Dtagx}[S_k^2(\tilde{g}_1(x), \ldots, \tilde{g}_m(x))]\Big| \le \delta^{-4} \cdot \delta'_x \le \delta \ll \frac{1}{k!} \cdot \sigma^{0.1 k}.$$
		\item 	The expectation of $S_k^2(\tilde{g}_1(x), \ldots, \tilde{g}_m(x))$ on a uniformly chosen $x$ is at most 
\begin{align*}\E_{x\sim U_n}[S_k^2(\tilde{g}_1(x), \ldots, \tilde{g}_m(x))] &= \sum_{T, T' \subseteq [m], |T|=|T'|=k} \E_{x\sim U_n} \bigg[\prod_{i\in T} \tilde{g}_i(x)\prod_{i'\in T'} \tilde{g}_{i'}(x)\bigg]\\
&=	\sum_{T\subseteq [m], |T|=k} \E_{x\sim U_n} \bigg[\prod_{i\in T} (\tilde{g}_i(x))^2\bigg] \tag{Since $\E[\tilde{g_i}] = 0$}\\
&
=	\sum_{T\subseteq [m], |T|=k} \prod_{i\in T} \var[{g}_i] \le  \frac{1}{k!} \cdot \Big(\sum_{i=1}^{m}{\var[\tilde{g_i}]}\Big)^{k} \le \frac{1}{k!} \cdot \sigma^{0.1k} \tag{Maclaurin's inequality}
\end{align*}



	\end{enumerate} 
	The second claim relies on the following:	
	\begin{enumerate}
		\item $S_k$ has small spectral-norm (using Claim~\ref{claim:S_k small spectral}, since $T$ is good) and hence is fooled by $\Dtagx$. In details, its spectral-norm is at most $\delta^{-2}$ and $\Dtagx$ is $\delta'_x$-biased. Thus $$\Big|\E_{x\sim U_n}[S_k(\tilde{g}_1(x), \ldots, \tilde{g}_m(x))] - \E_{x\sim \Dtagx}[S_k(\tilde{g}_1(x), \ldots, \tilde{g}_m(x))]\Big| \le \delta^{-2} \cdot \delta'_x \le \delta \le (\eps/n)^{C'}.$$

		\item The expectation of $S_k(\tilde{g}_1(x), \ldots, \tilde{g}_m(x))$ on a uniformly chosen $x$ is $0$.	
		\qedhere
		\end{enumerate} 
\end{proof}

The next lemma combined with Claim~\ref{claim:symmetric-excellent} concludes the low-variance case, since it shows that with high probability, $T$ is excellent, and then $\Dtagx$ is an $(\eps/n^{40})$-PRG for $\prod_{i=1}^{m}{\tilde{f_i}}$ (for a sufficiently large choice of $C'$).

\begin{lemma}
If $T$ is excellent, then $\E_{x\sim \Dtagx}[	\prod_{i=1}^{m}{\tilde{f_i}}] = (\prod_{i=1}^{m}\mu_i) \pm (\eps/n)^{\Omega(C')}$.
\end{lemma}
\begin{proof}
Let $x\sim \Dtagx$, 
and let $E$ be the event that
$|S_{\ell}(\tilde{g_1}(x), \ldots, \tilde{g_m}(x))| \le \frac{t^\ell}{\sqrt{\ell!}}$ 
and 
$|S_{\ell+1}(\tilde{g_1}(x), \ldots, \tilde{g_m}(x))| \le \frac{t^{\ell+1}}{\sqrt{(\ell+1)!}}$.
Picking $t = \sigma^{0.01}$, and using Claim~\ref{claim:symmetric-excellent} the event $E$ happens with probability at least $1-\sigma^{\Omega(\ell)} \ge 1-(\eps/n)^{\Omega(C')}$.
Assuming $E$ occurs, 
 Theorem~\ref{thm:GY} gives
$$\sum_{k=\ell}^{m} |S_{k}(\tilde{g_1}(x), \ldots, \tilde{g_m}(x))| \le 2\cdot (6et)^{\ell} \le \sigma^{\Omega(\ell)} \le (\eps/n)^{\Omega(C')}.$$
Furthermore, for sets of smaller cardinality, i.e., for $k \in \{1,\ldots, \ell-1\}$, Claim~\ref{claim:symmetric-excellent} gives
$$\Big|\E_{x\sim \Dtagx}[S_{k}(\tilde{g_1}(x), \ldots, \tilde{g_m}(x))]\Big| \le (\eps/n)^{C'} \qquad\text{and}\qquad\Big|\E_{x\sim \Dtagx}[S_{k}^2(\tilde{g_1}(x), \ldots, \tilde{g_m}(x))]\Big| \le 1\;.$$
We would like to bound $|\E_{x\sim \Dtagx}[S_{k}(\tilde{g_1}(x), \ldots, \tilde{g_m}(x)) \cdot \one_{E}]|$ for $k\in \{1,\ldots, \ell-1\}$. Towards this end,
we consider the expectation of $S_{k}(\tilde{g_1}(x), \ldots, \tilde{g_m}(x))$ by partitioning into the two cases depending on whether the event $E$ occurred or not.
\begin{align*} \E_{x\sim \Dtagx}&[S_{k}(\tilde{g_1}(x), \ldots, \tilde{g_m}(x))] \\&= \E_{x\sim \Dtagx}[S_{k}(\tilde{g_1}(x), \ldots, \tilde{g_m}(x)) \cdot \one_{E}] + \E_{x\sim \Dtagx}[S_{k}(\tilde{g_1}(x), \ldots, \tilde{g_m}(x)) \cdot \one_{\bar{E}}]\\
&= \E_{x\sim \Dtagx}[S_{k}(\tilde{g_1}(x), \ldots, \tilde{g_m}(x)) \cdot \one_{E}] \pm \sqrt{\E_{x\sim \Dtagx}[S_{k}^2(\tilde{g_1}(x), \ldots, \tilde{g_m}(x))] \cdot \Pr[\bar{E}]	\tag{Cauchy-Schwarz}}\\
&= \E_{x\sim \Dtagx}[S_{k}(\tilde{g_1}(x), \ldots, \tilde{g_m}(x)) \cdot \one_{E}] \pm \sqrt{\Pr[\bar{E}]}\\
&=\E_{x\sim \Dtagx}[S_{k}(\tilde{g_1}(x), \ldots, \tilde{g_m}(x)) \cdot \one_{E}] \pm (\eps/n)^{\Omega(C')}
\end{align*}
Thus, $|\E_{x\sim \Dtagx}[S_{k}(\tilde{g_1}(x), \ldots, \tilde{g_m}(x)) \cdot \one_{E}]| \le  (\eps/n)^{\Omega(C')}$ and we get 
$$
\E_{x\sim \Dtagx}\left[\prod_{i=1}^{m}\tilde{g_i}(x) \cdot \one_{E}\right] = \E_{x\sim \Dtagx}[\one_{E}] \pm  \sum_{k=1}^{m} |\E_{x\sim \Dtagx}[S_{k}(\tilde{g_1}(x), \ldots, \tilde{g_m}(x)) \cdot \one_{E}]| = 1 \pm (\eps/n)^{\Omega(C')}\;.$$
Since the $\tilde{f_i}$'s and $\mu_i$'s are bounded in $[-1,1]$, we get 
\begin{align*}
\E_{x}\left[ \prod_{i}\tilde{f_i}\right] &= 
\E_{x}\left[  \prod_{i}\tilde{f_i} \cdot \one_{E}\right] + \E_{x}\left[ \prod_{i}\tilde{f_i} \cdot \one_{ \neg E}\right]\\
&= \left(\prod_{i}{\mu_i} \cdot \E_{x\sim \Dtagx}\left[ \prod_{i=1}^{m} \tilde{g_i}(x) \cdot \one_{E}\right]\right)  \pm \Pr[\neg E] \\
&= \prod_{i}{\mu_i} \cdot \left(1\pm (\eps/n)^{\Omega(C')}\right)  \pm (\eps/n)^{\Omega(C')} =  \Big(\prod_{i}{\mu_i}\Big)   \pm (\eps/n)^{\Omega(C')}.\qedhere	
\end{align*}
\end{proof}

\subsection{High-Variance Case}

In the high-variance case, there exists a $\sigma \in (0,1]$ and an interval $I_\sigma = \{i: \Var[f_i]  \in (0.4\cdot\sigma^{1.1},\sigma]\}$ (the constant $0.4$ handles the case $\sigma=1$) satisfying:
$$
\sum_{i \in I_\sigma} \Var[f_i] > C\cdot \sigma^{-0.1}\cdot  \log^2(n/\eps)\;.
$$
In this case, the expected value of $\prod_{i=1}^m{f_i}$ under the uniform distribution is rather small:
\begin{align*}
\abs{\E\left[\prod_{i=1}^{m} f_i\right]}
 = \prod_{i=1}^{m} |\E[f_i]|
=  \prod_{i=1}^{m}{\sqrt{1-\var[f_i]}}
\le e^{-\sum_{i=1}^{m}{\var[f_i]/2}} \le e^{-C\cdot \log^2(n/\eps)/2} \le \eps/2.
\end{align*}
Recall that the pseudorandom restriction samples a set $T$ according to some $\delta_T$-biased distribution $\D_p$ with marginals $p$, and a partial assignment to the bits in $T$ according to some $\delta_x$-biased distribution $\Dx$. 
In the high variance case, it suffices to show that $\abs{\E_{T\sim \D_p, x\sim \Dx}\left[\prod_{i=1}^{m}{\tilde{f_i}(x)}\right]} \le \eps/2$.
Fix $T, x$.  Denote by $f^T_{i,x}(y) = (f_{i})_{T|y}(x)$.
Similarly to the calculation in the  case of the uniform distribution, we have
\begin{align*}
\abs{\prod_{i=1}^{m}{\tilde{f_i}(x)}} = \abs{\prod_{i=1}^{m}{\E_{y\sim U_{[n]\setminus T}}[f^T_{i,x}(y)]}}\le e^{-\sum_{i=1}^{m} \var[f^T_{i,x}]/2}
\end{align*}
Thus, it suffices to show that for most $T\sim \D_p$, $x\sim \Dx$ we have $\sum_{i=1}^{m} \var[f^T_{i,x}] \ge 10 \cdot \log(1/\eps)$.

\begin{theorem}[Theorem~\ref{thm:main} - High Variance Case]\label{thm:main high var}
	With probability $1-\eps/4$ over $T\sim \D_p$ and $x\sim \Dx$, it holds that  $\sum_{i\in I_{\sigma}}{\Var[f^T_{i,x}]} \ge 10 \cdot \log(1/\eps)$ .
\end{theorem}

\begin{proof}
Denote by $\Tvar := \sum_{i\in I_{\sigma}}\var[f_i]$.
By our assumption, $\Tvar \ge C \cdot \log^2(n/\eps) \cdot \sigma^{-0.1} \ge 5 \sigma^{-0.1}$.
Since all functions in $I_{\sigma}$ have variance at least $0.4 \cdot \sigma^{1.1}$ we have 
\begin{equation}\label{eq:Tvar vs m}
|I_{\sigma}| \le  \Tvar \cdot \tfrac{1}{0.4} \cdot \sigma^{-1.1} \le \Tvar^{12}
\end{equation}
We remark that in this case, unlike the low-variance case, we do not know how to handle large $\sigma$ easily, so for the rest of the proof $\sigma$ can be anything between $2^{-1-b}$ and $1$.

Fix $T$ and $x$. We expand $\var[f^T_{i,x}]$ 
$$
\var[f^T_{i,x}] = \E_{y\sim U_{[n]\setminus T}}[f^T_{i,x}(y)^2] - \E_{y\sim U_{[n]\setminus T}}[f^T_{i,x}(y)]^2 = 1-\tilde{f_i}(x)^2 \;.
$$
For any fixed $T$, using $\E[f_i] =\E[\tilde{f_i}]$ gives
\begin{align*}
	\E_{z\sim U_T} [\var[f^T_{i,z}]]   
&= 1- \E[(\tilde{f_i})^2] 
= (1-\E[f_i]^2) - (\E[(\tilde{f_i})^2] - \E[\tilde{f_i}]^2)
= \var[f_i] - \var[\tilde{f_i}]
\end{align*}

\begin{claim}[Most $T$'s preserve variance in expectation]\label{claim:preserve_var}
With probability at least $1-\eps/16$ over the choice of  $T \sim \D_p$, it holds that $\E_{z\sim U_T} \left[\sum_{i \in I_{\sigma}}{\Var[f^T_{i,z}]}\right] \ge \Tvar/2.$
\end{claim}
\begin{proof}
Since 
$\E_{z\sim U_T}[\var[f^T_{i,z}]] = \var[f_i] - \var[\tilde{f_i}]$, it suffices to show that with probability $1-\eps/16$ over the choice of $T\sim \D_p$ we have $\sum_{i} \var[\tilde{f_i}] \le \sum_{i} \var[f_i]/2$.
To show that 
$\sum_{i} \var[\tilde{f_i}]$ 
is well-concentrated we analyze its $k$-th moment for $k = C'  \log(1/\eps)$ where $C'$ is a sufficiently large constant.
\begin{align*}
\E_{T\sim \D_p}\left[\Big(\sum_{i \in I_{\sigma}} \var[\tilde{f_i}]\Big)^{k}\right]
= \sum_{i_1, i_2, \ldots, i_{k}\in I_{\sigma}} \E_{T}\left[\prod_{j=1}^{k} 	\var[\tilde{f_{i_j}}]\right]\;.
\end{align*}
Fix $i_1, \ldots, i_k \in I_{\sigma}$, (not necessarily distinct), then by Lemma~\ref{lemma:vars_prod}
\begin{align*}
\E_{T\sim \D_p}\left[\prod_{j=1}^{k} 	\var[\tilde{f_{i_j}}]\right] &\le   \E_{T\sim \Rp}\left[\prod_{j=1}^{k} 	\var[\tilde{f_{i_j}}]\right] + \delta_T	\cdot \prod_{j=1}^{k} 	\var[f_{i_j}]
\end{align*}
Overall, we get
\begin{align*}\E_{T\sim \D_p}\left[\Big(\sum_{i\in I_{\sigma}} \var[\tilde{f_i}]\Big)^{k}\right] \le  \E_{T\sim \Rp}\left[\Big(\sum_{i\in I_{\sigma}} \var[\tilde{f_i}]\Big)^{k}\right] + \delta_T \cdot \Tvar^{k}\;.
\end{align*}
To bound $\E_{T\sim \Rp}[(\sum_{i=1}^{m} \var[\tilde{f_i}])^{k}]$
we use the fact that by Theorem~\ref{thm:HC} 
$$\E_{T \sim \Rp}[\var[\tilde{f_i}]] \le p \cdot \var[f_i] \le 0.1 \cdot \var[f_i]$$
and then by Chernoff's bound 
$\sum_{i\in I_{\sigma}} \var[\tilde{f_i}] \le 0.2 \cdot \Tvar$ 
with probability at least $1-\exp(-\Omega(\Tvar))$.
Since $\sum_{i} \var[\tilde{f_i}]$ is always upper bounded by $\Tvar$, the $k$-moment of the sum is at most
$$(0.2 \cdot \Tvar)^{k} + (\Tvar)^{k} \cdot \exp(-\Omega(\Tvar)) \le 2(0.2 \cdot \Tvar)^{k}$$
We get that 
$\E_{T\sim \D_p}[(\sum_{i\in I_{\sigma}} \var[\tilde{f_i}])^{k}] \le 2(0.2 \cdot \Tvar)^{k} + \delta_T \cdot \Tvar^{k}$.
Since $\delta_T \ll 2^{-4 k}$ this is at most 
$3\cdot (0.2 \cdot \Tvar)^{k}$.
Thus, using Markov's inequality, the probability that $ \sum_{i\in I_{\sigma}} \var[\tilde{f_i}]  \ge  0.5 \cdot \Tvar$ is at most $3\cdot (0.2/0.5)^{k} \le \eps/16$ which completes the proof.
\end{proof}


Let \begin{equation}\label{eq:def ell}
	\ell \triangleq C'\cdot \log(n/\eps)/\log(|I_{\sigma}|)
\end{equation}
where $C'$ is a sufficiently large constant declared before Eq.~\eqref{eq:delta_T}. 
Assume that $\ell$ is an even integer.
Recall that $\delta = (\eps/n)^{-10C'} = |I_{\sigma}|^{-10 \ell}$.
We again define $T$ to be a {\sf good} set if $\prod_{i\in R}\tilde{f_i}$ has spectral-norm at most $1/\delta$ for all sets $R\subseteq I_{\sigma}$ of size at most $\ell$. As in Claim~\ref{claim:good} the probability that $T$ is good is at least $1-(|I_{\sigma}|+1)^{\ell} \cdot O(\ell b w)^3 \cdot \delta \ge 1-\eps/16$.
We define $T$ to be an {\sf excellent} set if $T$ is good and Claim~\ref{claim:preserve_var} holds for $T$. Then, $\Pr[T\text{~is excellent}] \ge 1-\eps/8$.
\begin{claim}
 If $T$ is a good set, then at most $\ell$ of the $\tilde{f_i}$'s have $L_1(\tilde{f_i}) \ge \delta^{-1/\ell}$.
\end{claim}
\begin{proof}
	If $\tilde{f_{i_1}}, \ldots, \tilde{f_{i_\ell}}$ have
	$L_1(\tilde{f_{i_j}})\ge \delta^{-1/\ell}$, then their product  has spectral-norm at least $\delta^{-1}$, since $L_1(\prod_{j=1}^{\ell}\tilde{f_{i_j}}) = \prod_{j=1}^{\ell} L_1(\tilde{f_{i_j}})$ for functions defined on disjoint variables.
\end{proof}

Fix an excellent set $T$. Let $G$ be the of indices $i\in I_{\sigma}$ with $L_1(\tilde{f_i})\le \delta^{-1/\ell}$. We show that with high probability over $x$, $\sum_{i\in G}{\var[f^T_{i,x}]} \ge 0.1 \cdot \Tvar$. We denote by 
$$\error_i(x) := \var[f^T_{i,x}] - \E_{z \sim U}[\var[f^T_{i,z}]] = \var[f^T_{i,x}] - (\var[f_i] - \var[\tilde{f_i}]).$$
Obviously $\E_{z\sim U}[\error_i(z)] = 0$ and $\error_i$ is bounded in $[-\sigma,1]$.
Furthermore, we have that $$\error_i(x) = (1-\tilde{f_i}(x)^2) -(1 - \E_{z\sim U}[\tilde{f_i}(z)^2]) = \E_{z\sim U}[\tilde{f_i}(z)^2] - \tilde{f_i}(x)^2$$
Thus, the error term have small spectral-norm since $L_1(\error_i) \le L_1(\tilde{f_i})^2$. We use this fact to bound 
$\E_{x\sim \Dx}[ (\sum_{i\in G} \error_i(x))^{\ell}]$. (recall that $\ell$ is an even integer.)
%\begin{claim}
%	$$\E_{x\sim \Dx}\Big[ \big(\sum_{i\in G} \error_i(x)\big)^{\ell}\Big] \le \E_{z\sim U}\Big[ \big(\sum_{i\in G} \error_i(x))\big)^{\ell}\Big]+ \delta_x \delta^{-2} \cdot |G|^{\ell}.$$
%\end{claim}
%\begin{proof}
%	The spectral-norm of $(\sum_{i\in G} \error_i(x))^{\ell}$ is at most $(|G| \cdot \delta^{-2/\ell})^{\ell}  = |G|^{\ell} \cdot \delta^{-2}$. Thus, any $\delta_x$ distribution fools $(\sum_{i\in G} \error_i(x))^{\ell}$ with error at most $\delta_x \cdot |G|^{\ell} \cdot \delta^{-2}$.
%\end{proof}
%
%\begin{claim}
%	$\E_{z\sim U}[ (\sum_{i\in G} \error_i(z))^{\ell}] \le  
%	\max\{\ell^{\ell}, (\ell \cdot \Tvar)^{\ell/2}\}$
%\end{claim}
%\begin{proof}
%Note that $\{\error_i(z)\}_{i\in G}$ are independent random variables, where each $\error_i(z)$ is bounded in $[-\var[f_i],1]$ with mean zero, and hence $\var[\error_i] \le \var[f_i]$ (See Lemma~\ref{lemma:var}). We apply Lemma~\ref{lemma:tail_bounds} to complete the proof.
%\end{proof}
%
%Note that $\ell \le \sqrt{\ell \cdot \Tvar}$ making the upper bound on $\E_{z\sim U}[ (\sum_{i\in G} \error_i(z))^{\ell}]$  at most $\left( \ell \cdot \Tvar\right)^{\ell/2}$.
%The upper bound with respect to $x\sim \Dx$ is at most $$\E_{x\sim \Dx}\Big[ \big(\sum_{i\in G} \error_i(x)\big)^{\ell}\Big] 
%\;\le\;  \left(\ell  \cdot  \Tvar\right)^{\ell/2} 
%	+ m^{\ell} \delta^{-2} \delta_x
%\;\le\; 2\cdot \left(\ell  \cdot  \Tvar\right)^{\ell/2}\;.$$
\begin{claim}\label{claim:small l-moment err}
	$$\E_{x\sim \Dx}\Big[ \big(\sum_{i\in G} \error_i(x)\big)^{\ell}\Big] \le 2 \cdot (\ell \cdot \Tvar)^{\ell/2}.$$
\end{claim}
\begin{proof}
	The spectral-norm of $(\sum_{i\in G} \error_i(x))^{\ell}$ is at most $(|G| \cdot \delta^{-2/\ell})^{\ell}  = |G|^{\ell} \cdot \delta^{-2}$. Thus, any $\delta_x$-biased distribution fools $(\sum_{i\in G} \error_i(x))^{\ell}$ with error at most $\delta_x \cdot |G|^{\ell} \cdot \delta^{-2}$ and we get
	$$
	\E_{x\sim \Dx}\Big[ \big(\sum_{i\in G} \error_i(x)\big)^{\ell}\Big] \le \E_{z\sim U}\Big[ \big(\sum_{i\in G} \error_i(x))\big)^{\ell}\Big]+ \delta_x \cdot |G|^{\ell} \cdot \delta^{-2}.
	$$
	
	To bound
	$\E_{z\sim U}[ (\sum_{i\in G} \error_i(z))^{\ell}]$ we use Lemma~\ref{lemma:tail_bounds}.
	We observe that $\{\error_i(z)\}_{i\in G}$ are independent random variables, where each $\error_i(z)$ is bounded in $[-\var[f_i],1]$ with mean zero, and hence $\var[\error_i] \le \var[f_i]$ (See Lemma~\ref{lemma:var}). Applying Lemma~\ref{lemma:tail_bounds} gives
	$$
	\E_{z\sim U}[ (\sum_{i\in G} \error_i(z))^{\ell}] \le  
	\max\{\ell^{\ell}, (\ell \cdot \Tvar)^{\ell/2}\}.$$

Since $\ell \le \sqrt{\ell \cdot \Tvar}$, the upper bound on $\E_{z\sim U}[ (\sum_{i\in G} \error_i(z))^{\ell}]$ is at most $\left( \ell \cdot \Tvar\right)^{\ell/2}$.
Finally, the upper bound with respect to $x\sim \Dx$ is at most \[
\E_{x\sim \Dx}\Big[ \big(\sum_{i\in G} \error_i(x)\big)^{\ell}\Big] 
\;\le\;  \left(\ell  \cdot  \Tvar\right)^{\ell/2} 
	+ \delta_x \cdot |G|^{\ell} \cdot \delta^{-2}
\;\le\; 2\cdot \left(\ell  \cdot  \Tvar\right)^{\ell/2}\;.\qedhere
\]
\end{proof}
%
Using Markov's Inequality and Claim~\ref{claim:small l-moment err} gives \begin{align*}
 \Pr_{x\sim\Dx} \left[\Big|\sum_{i\in G} \error_i(x)\Big| \ge \Tvar / 4\right]
 \le 2\cdot \left(\frac{\sqrt{\ell \cdot \Tvar}}{\Tvar/4}\right)^{\ell} \le O(\sqrt{\ell/\Tvar})^{\ell}\le O(1/\Tvar)^{\ell/4}\;.
 \end{align*}
using $\Tvar \ge \Omega(\log^2(n/\eps))$ and $\ell \le O(\log(n/\eps))$ in the last inequality.
Furthermore, using Eqs.~\eqref{eq:Tvar vs m} and \eqref{eq:def ell}: $O(1/\Tvar)^{\ell/4} \le \abs{I_{\sigma}}^{-\Omega(\ell)} \le (\eps/n)^{\Omega(C')} \le \eps/8$.
In the complement event,
$$
\sum_{i\in G} \var[f^T_{i,x}] =  \sum_{i\in G}(\E_{z}[\var[f^T_{i,z}]] + \error_i(x)) \ge \Tvar/2 - \ell  - \Tvar/4 \ge 0.1 \cdot \Tvar.
$$
Since $\Tvar \ge \Omega(\log^2(n/\eps))$, we get that with  probability at least $1-\eps/4$ over $T\sim \D_p$ and $x\sim \Dx$,  $\sum_{i} \var[f^T_{i,x}]  \ge 10 \log(1/\eps)$. (End of Proof of Theorem~\ref{thm:main high var})
\end{proof}
\newcommand{\VBad}{\mathsf{VarBad}}
\newcommand{\V}{\mathsf{Var}}
\newcommand{\Bad}{\mathsf{Bad}}
\newcommand{\Good}{\mathsf{Good}}


\newcommand{\GMany}{\mathbf{G_{\oplus Many}}}
\newcommand{\GXOR}{\mathbf{GXOR}}




\section{Assigning all the variables: a pseudorandom generator for the XOR of short ROBPs}\label{sec:assign_all}

In Theorem~\ref{thm:main}, we proved that we can pseudorandomly assign $p$-fraction of the coordinates of $f(x) = \prod_{i=1}^{m}{f_i(x)}$, while maintaining its acceptance probability up to an additive error  of $\eps$, using $\tilde{O}(\log(n/\eps)\cdot \log(b))$ random bits.
In this section, we will construct a pseudorandom generator $\eps$-fooling $f$, by applying Theorem~\ref{thm:main} $\poly\log\log(n/\eps)$ times, combined with Lovett's \cite{Lovett08} or Viola's \cite{Viola08} pseudorandom generator for low-degree polynomials, and CHRT's pseudorandom generator for constant-width ROBPs~\cite{CHRT17}. Our main result is:
\begin{thm}\label{thm:GXOR}
Let $n,w,b\in \N$, $\eps>0$. There exists an explicit pseudorandom generator that $\eps$-fools any XOR of ROBPs of  width-$w$ and length-$b$ (defined on disjoint sets of variables),using seed-length $O(\log(b) + \log\log(n/\eps))^{2w+2} \cdot \log(n/\eps)$.
\end{thm}

\paragraph{Assigning $0.9999$-fraction of the variables.}
The first step is rather standard. 
By making $t$ recursive calls to Theorem~\ref{thm:main} we can assign all but $(1-p)^t \le e^{-pt}$ fraction of the coordinates while maintaining the acceptance probability.
Note that we rely on the fact that under restrictions, the restricted function is still of the form $\prod_{i=1}^{m}{g_i(x)}$ where each $g_i$ is a ROBP of width $w$ that depends on at most $b$ bits (in other words, the class of functions we are trying to fool is closed under restrictions).
Setting $t = O(1/p)$, we can assign $0.9999$ fraction of the inputs bits while changing the acceptance probability by at most $\eps/n$.

\begin{claim}\label{claim:assigning-0.9999}
Let $f = \prod_{i=1}^{m} f_i(x)$, with block-length $b$.
Then, there is a pseudorandom restriction fixing each variable with probability $0.9999$, using at most $
 O\big(\log(b) +\log\log(n/\eps)\big)^{2w+1} \cdot \log(n/\eps)$ random bits, and changing the acceptance probability by at most $(\eps/n)$.

%Furthermore, any fixed set $S\subseteq[n]$ of  $k \le 5\log(n/\eps)$ variables remains alive with probability at most $2^t \cdot 0.0001^k$.
Furthermore, any fixed set $S\subseteq[n]$ of  $k \le 5\log(n/\eps)$ variables remains alive with probability at most $2 \cdot 0.0001^k$.
\end{claim}
\begin{proof}[Proof Sketch]
Apply Theorem~\ref{thm:main} with error $\eps/n^2$ for $t = \log(0.0001)/\log(1-p) = O(1/p)$ times iteratively, with independent random bits per each iteration. This changes the probability of acceptance by at most $(\eps/n^2)\cdot t \le \eps/n$ and keeps each variable alive with probability $(1-p)^{t} =0.0001$.
The amount of random bits used to sample the restriction is 
$$O(p^{-1}
 \cdot w  \log (n/\eps)  (\log\log(n/\eps) +  \log(b))
 \le 
 O\big(\log(b) +\log\log(n/\eps)\big)^{2w+1} \cdot \log(n/\eps).$$

Next, we show the furthermore part. Let $S$ be a fixed set of $k$ variables. By Claim~\ref{claim:inclusion-exclusion} the probability that $S$  remains alive under the $t$ pseudorandom restrictions is at most $((1-p)^{k} + 2^{k}\delta_T)^t$. 
Recall that $\delta_T = (n/\eps)^{\omega(1)}$.
For $k \le 5 \log(n/\eps)$, we get 
$((1-p)^{k} + 2^{k} \delta_T)^t \le (1-p)^{kt} \cdot (1+4^{k}\delta_T)^{t} \le 2\cdot 0.0001^{k}$.
\end{proof}


We would like to claim that $f = \prod_{i=1}^{m}f_i$ simplifies after assigning $0.9999$ of the coordinates. 
For a particular function $f_i$, with high probability, at least $1-1/32^b$, the block length decreases under a random restriction by a factor of $2$.
This is due to the fact that on expectation  at most $0.0001\cdot b$ of the variables will survive, and we can apply Chernoff's bound.
Now, if $m\le 16^b$, we can apply a union bound and get that with high probability the block-length decreases by a factor of $2$ in all functions $f_1, \ldots, f_m$ simultaneously.
We seem to have been making progress, going from block-length $b$ to block-length $b/2$, and we might hope that $\log(b)$ iterations of Claim~\ref{claim:assigning-0.9999} are enough to get a function that depends on $O(1)$ many variables (which is easy to fool). 
But, in order to carry the argument, even in the second step, we need to be able to afford the union bound on all functions. 
Ideally, the number of functions that are still alive also decreases from at most $16^b$ to at most $16^{b/2}$, and a similar union bound works replacing $b$ by $b/2$. We can continue similarly as long as in each iteration the block-length decreases by half and the number of functions by a square root.

We run into trouble if at some iteration we have more than $16^{b'}$ functions of block-length $b'$.
The first observation is that in this case the total variance of the functions is extremely high, exponential in $b'$. Recall that the expected value of the product is exponentially small in the total variance. This means that the expected value of the product is doubly-exponentially small in $b'$.
The second observation is that under $(1-\alpha)$-random restrictions, on average, the total variance decreases by a factor of $\alpha$.
Hence, we aim to apply a pseudorandom restriction assigning $(1-\exp(-b'))$ fraction of the variables alive, while keeping the total variance higher than $\log(n/\eps)$. This restriction is extremely aggressive, keeping only a polynomial fraction of the remaining variables alive (compared to say a constant fraction in Claim~\ref{claim:assigning-0.9999}).
However, we claim that in this case, such a restriction maintains the total variance high and thus the expected value of $\prod_{i=1}^{m}{f_i(x)}$ small (at most $\poly(\eps/n)$) in absolute value.

The nice thing about these ``aggressive pseudorandom restrictions'' is that they keep variables alive with such small probability that with high probability each function $f_i$ will depend on at most $O(1)$ variables after the restriction, except for a small number of functions covering at most $O(\log(n/\eps))$ ``bad variables''. This will allow us to fool the restricted function using Lovett's \cite{Lovett08} or Viola's \cite{Viola08} pseudorandom generator for low-degree polynomials.
In the next section, we explain how to handle this case in more details. 
Then, in Section~\ref{sec:4.2} we describe as a thought experiment a ``fake PRG'': an adaptive process that fools the XOR of short ROBPs, but depends on the function being fooled. In Section~\ref{sec:4.3} we show how to eliminate the adaptiveness and construct a true PRG for this class of functions. 

\subsection{PRG for the XOR of many functions with block-length $b$}\label{sec:4.1}

Let $\mathcal{F}_{b,n,t}$ be the class of functions of the form
$f(x) = f_0(x) \cdot \prod_{i=1}^{m}{f_i(x)}$
where $f_0, \ldots, f_m$ are Boolean functions on disjoint sets of variables, $f_0$ (the `junta') depends on at most $t$ variables, $f_1, \ldots, f_m$ are {\bf non-constant} and depend on at most $b$ variables and 
$16^b \le m \le 2\cdot 16^{2b}$.

\begin{lemma}\label{lemma:Gb}
There exists a constant $C>0$ such that the following holds.
For all $n,b,t$ such that  $C\cdot \log\log(n/\eps) \le b \le \log(n)$, there exists an explicit pseudorandom generator $\GMany(b,n,t,\eps):\pmone^{O(t + \log n/\eps)} \to \pmone^n$ that $\eps$-fools $\mathcal{F}_{b,n,t}$.
\end{lemma}

%The algorithm will use a universal large enough constant $C'\ge C$ in its definition.
\begin{algorithm}[H]
\caption{The Pseudorandom Generator $\GMany(b,n,t,\eps)$} 
\begin{algorithmic}[1]
\Require{A block-length $b$, the output length $n$, a junta-size $t$, an error parameter $\eps\in (0,1)$}
\State Set $x:= 1^n$
\State Pick $T\subseteq [n]$ using a $(\eps/n)^{10C}$-biased distribution with marginals $2^{-b}$.
\State Assign coordinates of $x$ in $[n]\setminus T$ using a $(\eps/n)^{10C}$-biased distribution.
\State Assign coordinates of $x$ in $T$ using Viola's generator with error $\frac{\eps}{4} \cdot (\frac{\eps}{n})^{C} \cdot 2^{-t}$ and degree~$16$.
\State \Return $x$.
\end{algorithmic}\label{test:1}
\end{algorithm}

\begin{lemma}\label{lemma:aggressive-real}
Let $C>0$ be a sufficiently large constant.
Let $C\cdot \log\log(n/\eps) \le b \le \log(n)$ be some integer.
Let $f_1, \ldots, f_m$ be non-constant Boolean functions that depend on disjoint sets of at most $b$ variables each.
Assume $m \ge 16^b$.
Suppose $T$ is $(\eps/n)^{10C}$-biased distribution with marginals $2^{-b}$.
Suppose $x$ is sampled from a $(\eps/n)^{10C}$-biased distribution.
Then, with  probability at least $1-(\eps/n)^{C/4}$, at least $4^b$ of the functions $(f_i)_{T|x}$ will be non-constant. 
\end{lemma}

%\begin{lemma}[Lemma~\ref{lemma:aggressive-real}, restated]
%Let $C>0$ be a sufficiently large constant.
%Let $C\cdot \log\log(n/\eps) \le b \le \log(n)$ be some integer.
%Let $f_1, \ldots, f_m$ be non-constant Boolean functions that depend on disjoint sets of at most $b$ variables each.
%Assume $m \ge 16^b$.
%Suppose $T$ is $(\eps/n)^{10C}$-biased distribution with marginals $2^{-b}$.
%Suppose $x$ is sampled from a $(\eps/n)^{10C}$-biased distribution.
%Then, with  probability at least $1-(\eps/n)^{C/4}$, at least $4^b$ of the functions $(f_i)_{T|x}$ will be non-constant. 
%\end{lemma}
\begin{proof}
Without loss of generality $m = 16^b$.
Let $k = C \log(n/\eps)/b$, and note that $k \le 2^b$ since $b\ge C \cdot \log \log(n/\eps)$ for a sufficiently large constant $C>0$.

Let $B_1, \ldots, B_m\subseteq [n]$ be the disjoint sets of variables on which $f_1, \ldots, f_m$ depend respectively.
For any function $f_i: \pmone^{B_i} \to \pmone$, there exists a sensitive pair of inputs $(\alpha^{(i)},\beta^{(i)}) \in \pmone^{B_i}$ such that $\alpha^{(i)}$ and $\beta^{(i)}$ differ in exactly one coordinate $j_i$ and such that $f_i(\alpha^{(i)}) \neq f_i(\beta^{(i)})$.
We say that the sensitive pair ``survives'' the random restriction defined by $(T,x)$ if both $\alpha^{(i)}$ and $\beta^{(i)}$ are consistent with the partial assignment defined by the restriction (i.e., if they agree with $x$ on $B_i \setminus T$).
For each function $f_1,\ldots, f_m$ fix one sensitive pair $(\alpha^{(1)}, \beta^{(1)}), \ldots, (\alpha^{(m)}, \beta^{(m)})$ and denote by $\cE_1, \ldots, \cE_m$ the events that these sensitive pairs survive.
Next, we claim that $\cE_1, \ldots, \cE_m$ are almost $k$-wise independent.
We compare them to the events $\cE'_1, \ldots, \cE'_m$ that indicate whether the sensitive pairs survive under a truly random restriction sampled from $\mathcal{R}_{2^{-b}}$.
Denote by $p_i = \Pr(\cE'_i)$.
Observe that $p_i \ge 2^{1-2b}$ since in order for the pair to survive it is enough that the sensitive coordinate remains alive (happens with probability $2^{-b}$) and that the partial assignment on the remaining coordinates agrees with $\alpha^{(i)}$ (happens with probability at least $2^{1-b}$). %Let $\cE'_1, \ldots, \cE'_m$ be i.i.d. random events happening with probability $p = 2^{1-2b}$.
Then,
	$$
	\E\left[\Big(\sum_{i=1}^{m}{(\one_{\cE_i}-p_i)}\Big)^k\right] \le 
		\E\left[\Big(\sum_{i=1}^{m}(\one_{\cE'_i}-p_i)\Big)^k\right] + (2m)^k \cdot \max_{K\subseteq [m]: |K|\le k} \left|\E[\prod_{i\in K} \one_{\cE_i}] - \E[\prod_{i\in K} \one_{\cE'_i}]\right|.$$ 
	We upper bound the first and second summands separately. 
	By Lemma~\ref{lemma:tail_bounds} and Lemma~\ref{lemma:var}, the first summand is upper bounded by $\max\{k^k, (Vk)^{k/2}\}$ where 
	$$
	V := \sum_{i=1}^{m}{p_i}.
	$$ 
	Since $V \ge m\cdot 2^{1-2b} = 2\cdot 4^b  \ge k$, the first summand is upper bounded by $(Vk)^{k/2}$.
		
	Next, we upper bound the second summand. By Vazirani's XOR lemma, since $x$ is $(\eps/n)^{10C}$-biased, we have that the marginal distribution of
	 any set of at most $k\cdot b$ bits in $x$ is $(\eps/n)^{10C} \cdot 2^{kb/2}$-close to uniform in statistical distance.
	Since $T$ is $(\eps/n)^{10C}$-biased with marginals $2^{-b}$, using Claim~\ref{claim:inclusion-exclusion} we have that the marginal distribution on any set of at most $k$ coordinates in $T$ is $(\eps/n)^{10C} \cdot 4^{k}$-close in statistical distance to the distribution sampled according to $\mathcal{R}_{2^{-b}}$.
	Thus, $|\E[\prod_{i\in K} {\one_{\cE_i}}] - \E[\prod_{i\in K} {\one_{\cE'_i}}]| \le (2^{kb/2} +4^{k}) \cdot (\eps/n)^{10C} \le 2^{kb/2+1} \cdot (\eps/n)^{10C}$ and we get 
	$(2m)^{k} \cdot 2^{kb/2+1} \cdot (\eps/n)^{10C}\le 1$.
	
	Combining the bounds on both summands we get
	$$
	\E\left[\Big(\sum_{i=1}^{m}{(\one_{\cE_i}-p_i)}\Big)^k\right] 
	\;\le\; 
	(Vk)^{k/2} + 1 
	\;\le\; 
	2\cdot (Vk)^{k/2}.
	$$
	Using $V = \sum_{i=1}^{m} p_i$ we get 
	$$\Pr\left[\sum_{i=1}^{m}\one_{\cE_i} \le V/2 \right] \le 
	 \Pr\left[\Big(\sum_{i=1}^{m}{(\one_{\cE_i}-p_i)}\Big)^k  \ge (V/2)^k \right]  \le 2(Vk)^{k/2} \cdot (V/2)^{-k}$$
Using $V \ge 2\cdot 4^b$ and $k\le 2^b$ we get $\Pr\left[\sum_{i=1}^{m}\one_{\cE_i} \le V/2 \right] \le 2(4k/V)^{k/2} \le 2\cdot(2/2^b)^{k/2}\le  (\eps/n)^{C/4}$.
In the complement event, at least $V/2 \ge 4^b$ of the functions $(f_1)_{T|x}, \ldots, (f_m)_{T|x}$ are non-constant.
\end{proof}

%We defer the proof of Lemma~\ref{lemma:aggressive-real} to Appendix~\ref{sec:aggressive}.

\begin{lemma}\label{lemma:low-deg}
Let $f: \F_2^n \to \pmone$.
Suppose $f(x) = h(x) \cdot (-1)^{g(x)}$ where $h$ is a $k$-junta 
and $g$ is a polynomial of degree-$d$ over $\F_2$.
	If $\D$ fools degree-$d$ polynomials over $\F_2$ with error $\eps$, then $\D$ fools $f$ with error $\eps \cdot 2^{k/2}.$
\end{lemma}
\begin{proof}
Let $J$ be the set of variables on which $h$ depends.
Using the Fourier transform of $h$:
$h(x) = \sum_{S \subseteq J} \hat{h}(S) \cdot (-1)^{\sum_{i\in S} x_i}$
we write $f$ as
$f(x) = \sum_{S \subseteq J} \hat{h}(S) \cdot (-1)^{\sum_{i\in S} x_i + g(x)}$.
Note that $\sum_{i\in S} x_i + g(x)$ is a polynomial of degree-$d$ over $\F_2$ as well, thus we get
\begin{align*} \left|\E_{x\sim U}[f(x)]-\E_{x\sim \D}[f(x)]\right| 
&\le \sum_{S}|\hat{h}(S)| \cdot 
\Big|\E_{x\sim U}[(-1)^{\sum_{i\in S} x_i + g(x)}] - \E_{x\sim \D}[(-1)^{\sum_{i\in S} x_i + g(x)}]\Big| \\
&\le L_1(h) \cdot \eps \le 2^{k/2} \cdot \eps\;.\qedhere\end{align*}
\end{proof}


\begin{proof}[Proof of Lemma~\ref{lemma:Gb}]
First note that $f$ has very small expectation under the uniform distribution
$$
\left|\E_{z\sim U}\Big[f_0(z) \cdot \prod_{j=1}^{m} f_j(x)\Big]
\right| \le (1-2^{-b})^{16^{b}} \ll \frac{\eps}{4}.
$$
using the assumption $b\ge C \log \log(n/\eps)$.
Thus, we need to maintain  low-expectancy under the pseudorandom assignment.
By Lemma~\ref{lemma:aggressive-real}, with probability at least $1-(\eps/n)^{C/4}\ge 1-\frac{\eps}{100}$ after the aggressive random restriction at least $4^{b}$ of the functions $f_1, \ldots, f_m$ remain non-constant. Since $b\ge C \log\log(n/\eps)$ we maintained the low-expectancy under aggressive random restrictions. That is, whenever $4^{b}$ of the functions $f_1, \ldots, f_m$ remain non-constant under restriction, the expected value of the restricted function under the uniform distribution is at most $(1-2^{-b})^{4^b} \ll \eps/4$ in absolute value.

Furthermore, we wish to show that with high probability, except for a set of at most $C\log (n/\eps)$ ``bad variables'' all functions have block-length at most $16$.
Recall that there are at most $2\cdot 16^{2b}$ functions.
The probability that any particular $k$ variables survive is at most $2^{-b k} + (\eps/n)^{10C}$.
Pick $k = C \log(n/\eps)/b \le C \log(n/\eps)$.
The probability that at least $k$ variables in at most $\ell$ functions survive is 
$$
\binom{2\cdot 16^{2b}}{\ell} \cdot \binom{\ell \cdot b}{k} \cdot (2^{-b k} + (\eps/n)^{10C})
\le 2 \cdot 2^{\ell + 9b\ell - bk} \le 2^{10b\ell-bk}
$$
If $\ell \le k/16$, then this probability is at most $2^{-6bk/16} = (\eps/n)^{6C/16} \ll \frac{\eps}{100}$.
This means that, with high probability, there are less than $k$ variables from all functions with more than $16$ effective variables remaining.
Otherwise, there would have been $\ell \le k/16$ functions accountable to a total number of more than $k$ variables that remained alive and effective, under the restriction.

Overall, with probability at least $1-\eps/50$ we are left with the XOR of a small-junta, on at most $t + C\log(n/\eps)$ variables, and an XOR of at least $4^b$ non-constant functions on at most $16$ variables (i.e., a degree $16$ polynomial). Moreover, the restricted function has expected value at most $\eps/4$ in absolute value under the uniform distribution.
Using Claim~\ref{lemma:low-deg} we get that Viola's~\cite{Viola08} or Lovett's~\cite{Lovett08} PRG for low-degree polynomials $\eps/4$-fools the remaining function. Combining all estimates we get that the expected value of the restricted function under our distribution is at most $3\eps/4$ in absolute value which completes the proof.
\end{proof}

\subsection{A Thought Experiment}\label{sec:4.2}
We are ready to describe the pseudo-random restriction process in full detail.
We start by describing a process that iteratively ``looks'' at the restricted functions in order to decide which pseudorandom restriction to apply next: the one described in Lemma~\ref{claim:assigning-0.9999} or the one from Lemma~\ref{lemma:aggressive-real}.
This ultimately defines a decision tree of random restrictions.
We then show in Section~\ref{sec:4.3} how to transform the adaptive process into a non-adaptive pseudorandom generator that (by definition) does not depend on the function it tries to fool.
Namely, we would generate a pseudorandom string that  fools the function no matter what path was taken in the decision tree.


We start with $m \le n$ blocks of length $b$.
We assume that $m \le 16^b$ (if not set $b= \log_{16}(m)$).
\begin{algorithm}[H]
\caption{an ``adaptive pseudorandom generator''} 
\begin{algorithmic}[1]
\For{$i=0,1,\ldots$} 
\State Let $b_i = b/2^i$.
\If{$b_i \le C\log \log(n/\eps)$} 
apply CHRT's PRG on the remaining coordinates, and Halt!
\EndIf
\If{more than $16^{b_i}$ of the restricted functions are non-constant and depend on at most $b_i$ variables} 
apply $\GMany(b_i,10\log(n/\eps),n,\eps/2)$ from Lemma~\ref{lemma:Gb} on the remaining variables, and Halt!
\Else{ apply the pseudorandom restriction from Lemma~\ref{claim:assigning-0.9999} on the remaining variables.}
\EndIf
\EndFor
\end{algorithmic}\label{alg:adaptive}
\end{algorithm}

Next, we show that the process yields a pseudorandom string fooling $f = \prod_{i=1}^m{f_i(x)}$.
First, note that the process either stops at Step 3 or at Step 4. In both cases we assign all the variables according to some pseudorandom generator, hence all the variables will be assigned by the end of the process.

For $i=0,1, \ldots, $.
Let $T_i$ be the set of coordinates that remain alive at the beginning of the $i$-th iteration.
Denote by $f_j^{(i)}$ the $j$-th function under the restriction at the beginning of the $i$-th iteration. 
Define $\V[f_j^{(i)}]$ to be the set of variables that affect the output of $f_j^{(i)}$.
For example if $f_j^{(i)}$ is a constant function, then $\V[f_j^{(i)}] = \emptyset$.

Let $\Good_i = \{j:  1 \le |\V[f_j^{(i)}]| \le b_i\}$ be the set of functions that depend on some but not more than $b_i$ variables, $\Bad_i = \{j:  |\V[f_j^{(i)}]| > b_i\}$ be the set of functions that depend on more than $b_i$ variables
%. Let $\Good_{\le i} = \bigcap_{i'\le i}\Good_{i'}$, $\Bad_{\le i}= \bigcup_{i'\le i} \Bad_{i'}$ 
and  $\VBad_{i} = \bigcup_{j\in \Bad_{i} }{\V[f_j^{(i)}]}$.

\begin{claim}\label{claim:VBad}
Let $b_i > C\log \log(n/\eps)$.
Suppose $|\VBad_{i}| \le 10 \log(n/\eps)$ and $|\Good_{i}| \le 16^{b_{i}}$.
Then, with probability at least $1-(\eps/n)$ 
we have $|\VBad_{(i+1)}| \le 10 \log(n/\eps)$.
\end{claim}
\begin{proof}
Under the assumptions we reach Step 5 in Algorithm~\ref{alg:adaptive}. We show that:
\begin{enumerate}
	\item With probability at least $1-\frac{1}{2}(\eps/n)$, at most $5\log(n/\eps)$ of the variables in $\VBad_{i}$ remain alive in Step 5.
	\item With probability at least $1-\frac{1}{2}(\eps/n)$, at most $5 \log(n/\eps)$ new variables are added to $\VBad_{(i+1)}$.
\end{enumerate}
Both claims rely on the fact that  any set of $k \le 5 \log(n/\eps)$ variables remain alive under the pseudorandom restriction in Lemma~\ref{claim:assigning-0.9999} with probability at most $2\cdot 0.0001^k$.

%Both claims rely on the fact that for any set of $k \le 5 \log(n/\eps)$ variables, the probability that all of them survive the pseudorandom restriction in Lemma~\ref{claim:assigning-0.9999} is at most $2\cdot 0.0001^k$.

This  first item follows since the probability that more than $5\log(n/\eps)$ variables in $\VBad_i$ survive is at most 
$$\binom{10\log(n/\eps)}{5\log(n/\eps)} \cdot 2\cdot0.0001^{5\log(n/\eps)} \le \tfrac{1}{2} (\eps/n).$$

As for the second item, we start with the case where $b_i \le 2\log(n/\eps)$.
Assume that more than $5\log(n/\eps)$ new variables were added to $\VBad_{(i+1)}$. 
This implies that there is a set of $k = \lceil{5\log(n/\eps)/(b_i/2)\rceil}$ good functions in step $i$ that are accountable to at least $5\log(n/\eps)$ bad variables in step $i+1$.
The latter event happens with probability at most 
$$\binom{16^{b_i}}{k} \cdot \binom{k b_i}{5\log(n/\eps)} \cdot 2\cdot 0.0001^{5\log(n/\eps)} \le 32^{b_i k} \cdot 2 \cdot 0.0001^{5\log(n/\eps)} \le \tfrac{1}{2} (\eps/n)$$
(where we used $k b_i \le b_i+10\log(n/\eps) \le 12\log(n/\eps)$)
which finishes the case $b_i \le 2\log(n/\eps)$.

In the case where $b_i > 2\log(n/\eps)$, we show that with high probability all good functions remain good.
For each individual function, using Markov's inequality
\begin{align*}\Pr[ |\V[f^{(i+1)}_j]| \ge b_i/2] 
&\le \frac{\E\left[\binom{|\V[f^{(i+1)}_j]|}{ \log (n/\eps)}\right]}{\binom{b_i/2}{\log(n/\eps)}}
\le 2\cdot 0.0001^{\log n/\eps} \cdot \frac{\binom{b_i}{\log(n/\eps)}}{\binom{b_i/2}{\log(n/\eps)}}\\
&\le 2\cdot  0.0001^{\log n/\eps} \cdot \frac{(e \cdot b_i/ \log(n/\eps))^{\log(n/\eps)}}{((b_i/2)/ \log(n/\eps))^{\log(n/\eps)}} \\
&= 2\cdot(0.0001 \cdot e\cdot 2)^{\log(n/\eps)} \le \tfrac{1}{2} (\eps/n^2).\end{align*}
Thus, we can apply a union bound and show that all good functions remain good with  probability at least $1-\frac{1}{2}(\eps/n)$.
\end{proof}


Say the process finished. 
We shall assume that $|\VBad_i| \le 10 \log(n/\eps)$ for every iteration $i$ until the process stopped. By Claim~\ref{claim:VBad} this happens with probability at least $1-\log(b)\cdot(\eps/n) \ge 1-\eps/2$ by applying a union bound on the at most $\log(b)$ iterations.
We wish to show that we constructed a pseudorandom string fooling $f$. We consider two cases:
\begin{enumerate}
	\item 
	We stopped on Step~3 at some iteration $i$. If $i=0$ then $m \le 16^{b_0} \le \poly\log(n/\eps)$ and at most $\poly\log(n/\eps)$ variables remain that affect the functions $f_{j}^{(i)}$.
	 %In this case, by Claim~\ref{claim:VBad}, with high probability, 
	 Otherwise, since $|\Good_{(i-1)}| \le 16^{2b_i} \le \poly\log(n/\eps)$ and $|\VBad_{(i-1)}| \le 10 \log(n/\eps)$, at most $\poly\log(n/\eps)$ variables remain that affect the functions $f_{j}^{(i-1)}$, and thus at most $\poly\log(n/\eps)$ variables remain that affect the functions $f_{j}^{(i)}$.
%	Since $|\VBad_i| \le 10\log(n/\eps)$, and there are at most $16^{2C\log \log(n/\eps)} \le \poly\log(n/\eps)$ ``good'' functions that depend on at most $C\log\log(n/\eps)$ variables. 
%	 Overall, only $\poly\log(n/\eps)$ variables remain that affect the functions $f_{j}^{(i)}$, and 
Thus, we can write $\prod_{j=1}^{m} f_j^{(i)}$ as a ROBP of width $2w$ and length $\poly\log(n/\eps)$, which is $(\eps/2)$-fooled by the pseudorandom generator from Theorem~\ref{thm:CHRT} using  $\tildeO(\log(n/\eps))$ random bits.
	 
\item We stopped at Step~4 at some iteration $i$.
%We wish to bound $|\Good_{i}|$ from above. 
%Let $G = \Good_{i} \cap \Good_{(i-1)}$.
%Since $\Good_{(i-1)} \le 16^{2b_i}$ we have $|G| \le 16^{2b_i}$.
%All functions in $G$ are non-constant and have at most $b_i$ affecting variables.
%Other than $G$ there are at most $|\VBad_{(i-1)}| + |\VBad_{i}| \le 20\log(n/\eps)$ alive variables.
Certainly, $|\Good_{i}|\le  |\Good_{(i-1)}|+|\Bad_{(i-1)}| \le 16^{2b_i} + 10 \log(n/\eps) \le 2\cdot 16^{2b_i}$.
%Note that there are at most $16^{2 b_i}$ good functions in the beginning of the $i$-th iteration, 
%otherwise we would have stopped earlier.
Thus, we are in the case that was handled in Section~\ref{sec:4.1}, with $t \le 10 \log(n/\eps)$. Indeed, Lemma~\ref{lemma:Gb} guarantees that $\GMany(b_i,10\log(n/\eps), n, \eps/2)$ fools the remaining function with error at most $\eps/2$ using $O(\log(n/\eps))$ random bits.
\end{enumerate}


\subsection{The Actual Generator}\label{sec:4.3}
Algorithm~\ref{alg:adaptive} described the pseudo-random generator as if we knew whether or not the condition in step 3 holds. 
However, a pseudorandom generator cannot depend on the function it tries to fool.
To overcome this issue, we use the following general observation regarding pseudorandom generators.

\begin{claim}\label{claim:XOR-of-PRGs}
Say there are two families of functions $\mathcal{F}_1$ and $\mathcal{F}_2$ that are both closed under shifts (i.e., closed under XORing a constant string to the input). Say that $\D_1$ is an $\eps$-PRG for $\mathcal{F}_1$ and $\D_2$ is an $\eps$-PRG for $\mathcal{F}_2$  then $\D_1 \oplus \D_2$ is an $\eps$-PRG for $\mathcal{F}_1 \cup \mathcal{F}_2$.\end{claim}
\begin{proof}
%
Let $f \in \mathcal{F}_1 \cup \mathcal{F}_2$, we show that $\D_1 \oplus \D_2$ fools $f$.
By symmetry assume $f\in \mathcal{F}_1$.
\begin{align*}
\abs{\E_{\substack{x_1\sim \D_1\\ x_2 \sim \D_2}}[f(x_1 \oplus x_2)] -\E_{\substack{z\sim U}}[f(z)]}
&=\abs{\E_{\substack{x_1\sim \D_1\\ x_2 \sim \D_2}}[f(x_1 \oplus x_2)] -\E_{\substack{x_1\sim U\\ x_2 \sim \D_2}}[f(x_1 \oplus x_2)]}\\ 
&\le  \E_{x_2 \sim \D_2} 
\abs{\E_{x_1 \sim \D_1}[f(x_1 \oplus x_2)] -\E_{x_1\sim U}[f(x_1 \oplus x_2)]}\\
&=  \E_{x_2 \sim \D_2} \abs{\E_{x_1 \sim \D_1}[f_{x_2}(x_1)] -\E_{x_1\sim U}[f_{x_2}(x_1)]}\end{align*}
where $f_{y}(x):= f(x\oplus y)$. Since $\mathcal{F}_1$ is closed under shifts, we have that $f_{x_2} \in \mathcal{F}_1$ thus $\D_1$ $\eps$-fools $f_{x_2}$ and we get
$
\E_{x_2 \sim \D_2} \abs{\E_{x_1 \sim \D_1}[f_{x_2}(x_1)] -\E_{x_1\sim U}[f_{x_2}(x_1)]} \le \eps\;.$
\end{proof}

The actual generator would proceed as follows.
\begin{algorithm}[H]
\caption{The Pseudorandom Generator $\GXOR(T, w, b,\eps)$} % (A,m)}
\begin{algorithmic}[1]
\Require a set $T\subseteq [n]$ of the ``live'' coordinates, a width $w$,  an integer $b$, a parameter $\eps\in (0,1)$.
\If{$b \le C\log\log(n/\eps)$} \Return $\mathbf{CHRT}(n, n', 2w,  \eps)|_{T}$ for $n' = 2\cdot 16^{2b}\cdot b +10\log(n/\eps)$
\EndIf
\State Let $x := \GMany(b,t,n,\eps)|_{T}$ for $t = 10 \log(n/\eps)$.
\State Pick $T' \subseteq T$, $y\in \pmone^{T\setminus T'}$ according to Claim~\ref{claim:assigning-0.9999}
\State Let $z  := \GXOR(T', w, b/2,\eps/2)$.
\State \Return  $x \oplus \Sel_{T'}(z,y)$.
\end{algorithmic}
\end{algorithm}


\begin{claim}[Proof of Correctness]\label{claim:correctness}
Let $T \subseteq [n]$.
Suppose $f_1, \ldots, f_m$ are functions on disjoint sets of $T$.
Suppose each function depends on at most $b$ variables except for a total of at most $10 \log(n/\eps)$ variables, and the number of non-constant functions is at most $2 \cdot 16^{2b}$.
Then, $\GXOR(T,w, b,\eps)$ fools $f = \prod_{i=1}^{m} f_i$ with error $\eps$.
\end{claim}
\begin{proof}
We prove the claim by induction on $b$.
	If $b \le C \log \log(n/\eps)$ then Theorem~\ref{thm:CHRT} implies correctness.
	If $b > C \log \log(n/\eps)$ then we
	consider the following two cases:
	\begin{enumerate}
%	{\color{red}
		\item If there are more than $16^b$ good functions, then $x = \GMany(b,t,n,\eps)|_{T}$ fools $\prod_{i=1}^{m}f_i$ with error $\eps$.
%	}
		\item Otherwise, there are at most $16^b$ good functions and we apply Step~3. According to Claim~\ref{claim:assigning-0.9999}, the average acceptance probability of $f_{T'|y}$ is $\eps/4$ close to that of $f$. 
	Furthermore, with  probability at least $1-\eps/4$ all functions $(f_1)_{T'|y}, \ldots, (f_m)_{T'|y}$ depend on at most $b/2$ variables except for at most $10 \log(n/\eps)$ variables (by Claim~\ref{claim:VBad}).
	In such a case, the number of non-constant functions among $(f_1)_{T'|y}, \ldots, (f_m)_{T'|y}$ is at most $16^{b} + 10\log(n/\eps) \le 2\cdot 16^{b}$.
	Using induction, $z = \GXOR(T',w,b/2,\eps/2)$ fools $f_{T'|y}$ with error $\eps/2$, and we get that $\Sel_{T'}(z,y)$ fools $f$ with error $\eps$.

	\end{enumerate}
	Since we have a pseudorandom generator fooling the function in each case,
	 Claim~\ref{claim:XOR-of-PRGs} shows that $x \oplus \Sel_{T'}(z,y)$ fools $f$ with error $\eps$.
	\end{proof}
	
	


\begin{claim}[Seed Length]\label{claim:seed-length}
	The amount of random bits used to calculate $\GXOR([n],w, b,\eps)$ is at most $O(\log(b) + \log\log(n/\eps))^{2w+2} \cdot \log(n/\eps)$.
\end{claim}
\begin{proof}
Unwrapping the recursive calls in the evaluation of $\GXOR([n],w, b,\eps)$ we see that there are at most $\log(b)$ recursive calls to the procedure and that the error parameters are at least $\eps/2^{\log(b)} \ge \eps/n$ in all of them.

We apply the generator from Theorem~\ref{thm:CHRT} only once during these recursive calls, on a ROBP of width-$w$ and length $\poly\log(n/\eps)$. Thus, the application of Theorem~\ref{thm:CHRT} uses at most  $O(\log\log(n/\eps)^{w+2} \log(n/\eps))$ random bits.

The partial assignment from Claim~\ref{claim:assigning-0.9999} uses  at most $O\big(\log(b) + \log\log(n/\eps)\big)^{2w+1} \cdot \log(n/\eps)$ each time we invoke it, and we invoke it at most $\log(b)$ times.

The generator $\GMany$ uses $O(\log (n/\eps))$ random bits each time we invoke it, and we invoke it at most $\log(b)$ times.
\end{proof}

Claims~\ref{claim:correctness} and~\ref{claim:seed-length} completes the proof Theorem~\ref{thm:GXOR} with $\GXOR([n], w, b, \eps)$ as the generator.



\subsection{Pseudorandom generator for read-once polynomials}

\begin{thm}\label{thm:read-once poly}
There exists an explicit $\eps$-PRG for the class of read-once polynomials on $n$ variables with seed-length $O((\log \log(n/\eps))^6 \cdot \log(n/\eps))$.
%$\GXOR([n],2, \log(4n/\eps),\eps/4n)$ fools any read-once polynomial with error at most $\eps$. Its seed length is $O((\log \log(n/\eps))^6 \cdot \log(n/\eps))$.
\end{thm}
\begin{proof}
We show that $\GXOR([n],2, \log(8n/\eps),\eps/8n)$ fools any read-once polynomial with error at most $\eps$. Its seed length is $O((\log \log(n/\eps))^6 \cdot \log(n/\eps))$.

	A read-once polynomial can be written as the XOR of AND functions on disjoint variables, i.e., as the XOR of width-$2$ ROBPs on disjoint variables.
	It remains to show that these ROBPs are short.
	Rather, we show that any PRG that $(\eps/8n)$-fools read-once polynomials of degree at most $b = \log(8n/\eps)$ also $\eps$-fools all read-once polynomials.
Let	$$f(x) = \sum_{i=1}^{m} \prod_{j\in B_i} x_j$$ be a read-once polynomial over $\F_2$, where $B_1, \ldots, B_m$ are disjoint subsets of $[n]$.
Without loss of generality let $B_1, \ldots, B_\ell$ be the blocks of length bigger than $b$. 
Let $$f'(x) = \sum_{i=\ell+1}^{m} \prod_{j\in B_i} x_j,$$ be the sum over monomials of degree at most $b$ of $f$.
Let $\D = \GXOR([n],2, \log(8n/\eps),\eps/8n)$. By triangle inequality
\begin{align}\nonumber\abs{\E_{x\sim \D}[f(x)] -\E_{x\sim U_n}[f(x)]} &\le \Pr_{x\sim \D}[f(x) \neq f'(x)] + \Pr_{x\sim U_n} [f(x) \neq f'(x)] + 
\abs{\E_{x\sim \D}[f'(x)] -
\E_{x\sim U_n}[f'(x)]}\\
&\le \sum_{i=1}^{\ell} \Pr_{x\sim \D}[\wedge_{j\in B_i}(x_j=1)] + \sum_{i=1}^{\ell} \Pr_{x\sim U_n}[\wedge_{j\in B_i}(x_j=1)] + \eps/8n\label{eq:read-once polys}
\end{align}
For $i\in \{1,\ldots, \ell\}$, since $|B_i|\ge b$, we have
$\Pr_{x\sim U_n}[\wedge_{j\in B_i}(x_j=1)] \le 2^{-b} \le \eps/8n$. 
As for the distribution $\D$, by monotonicity 
$$\Pr_{x\sim \D}[\wedge_{j\in B_i}(x_j=1)] \le
\Pr_{x\sim \D}[\wedge_{j\in B'_i}(x_j=1)]$$ where $B'_i$ is any arbitrary subset of exactly $b$ variables from $B_i$. Since $\D$ fools degree-$b$ read-once polynomials with error at most $\eps/8n$, and $\wedge_{j\in B'_i}(x_j=1)$ is such a polynomial, we get that $\Pr_{x\sim \D}[\wedge_{j\in B'_i}(x_j=1)]$  is at most $2^{-b} + \eps/8n \le \eps/4n$.
Plugging both bounds into Eq.~\eqref{eq:read-once polys} we get $\abs{\E_{x\sim \D}[f(x)] -\E_{x\sim U_n}[f(x)]}\le (\eps/4n)\cdot \ell + (\eps/8n)\cdot \ell + \eps/8n \le \eps$.
\end{proof}



%%%%%%%%%%%%%%%%%%%%%%%%%%%%%%%%%%%%%%%%%%%%%%%%%%%%%%%%%%%
% !TEX root = MRT.tex

\newcommand{\mE}{{\mathcal{E}}}
\newcommand{\BRRY}{{\mathbf{BRRY}}}
\newcommand{\Col}{{\mathsf{Col}}}
\newcommand{\FCol}{{\mathsf{FCol}}}

\section{Pseudorandom generators for width-3 ROBPs}

In this section, we construct pseudorandom generators fooling width-3 ROBPs with seed-length $\tilde{O}(\log n)$. For ordered width-3 ROBPs we can guarantee error $1/\poly\log(n)$ using seed-length $\tilde{O}(\log n)$, and for unordered width-3  ROBPs we can guarantee error $1/\poly\log\log(n)$ for the same seed-length:

\begin{theorem}[Main Theorem]\label{thm:main-3ROBP-ordered}
For any $\epsilon > 0$, there exists an explicit PRG that $\delta$-fools ordered 3ROBPs with seed-length $\tilde{O}(\log(n/\delta)) + O((\log(1/\delta)) \cdot (\log n))$. 
\end{theorem}

Note that in comparison, even for constant $\delta > 0$, the best previous generators had seed-length $O(\log^2 n)$ for ordered 3ROBPs. We also get similar improvements for unordered 3ROBPs but with worse dependence on the error $\delta$. 

\begin{theorem}\label{thm:main-3ROBP-unordered}
For any $\epsilon > 0$, there exists an explicit PRG that $\delta$-fools ordered 3ROBPs with seed-length $\tilde{O}(\log(n/\delta)) + O( \poly(1/\delta) \cdot (\log n))$
\end{theorem}



\subsection{Proof overview}


We heavily rely on the pseudorandom restriction from Theorem~\ref{thm:main_two_steps} that assigns $p = 1/\poly \log \log (n)$ of the variables while changing the acceptance probability by at most $\poly(\eps/n)$. As a first step we assign a constant fraction of the coordinates.

\paragraph{Assigning 0.9999 of the coordinates.} The first step is rather simple: we apply iteratively the pseudorandom restriction from Theorem~\ref{thm:main_two_steps} $O(1/p)$ times to get the following analog result to Claim~\ref{claim:assigning-0.9999}.
The proof is the same as that of Claim~\ref{claim:assigning-0.9999} and is omitted.

\begin{claim}\label{claim:assigning}
For all constants $\alpha \in (0,1)$, there is a pseudorandom restriction that leaves each variable unfixed with probability at most $\alpha$, using at most $\log(n/\eps) \cdot \poly(\log\log(n/\eps))$ random bits, and changing the acceptance probability of width-3 ROBPs by at most $\poly(\eps/n)$.

Furthermore, any fixed set $S\subseteq[n]$ of  $k \le 5\log(n/\eps)$ variables remains alive with probability at most $2 \alpha^k$.\end{claim}

Let $B$ be a 3ROBP of length-$n$. First, we claim that after applying the pseudorandom restriction $\rho$ in Claim~\ref{claim:assigning}, with high probability (at least $1-\poly(\eps/n)$), $B|_{\rho}$ has a simpler structure in that there will be several width two layers in $B|_{\rho}$ and furthermore, between any two width two layers the subprogram has $O(\log(n/\eps))$ \emph{colliding layers}. Concretely, we use the following definitions. 

\begin{definition}
Given a ROBP $B$, we call a layer of edges {\sf colliding} if either the edges marked by $0$ and the edges marked by $1$ collide.

We call a ROBP $B$ a $(w, \ell,m)$-ROBP if $B$ can be written as $D_1 \circ D_2 \circ \ldots \circ D_m$, with each $D_i$ being a width $w$ ROBP with the first and last layers having at most two vertices and each $D_i$ having at most $\ell$ colliding layers.
\end{definition}

We show that after applying the pseudorandom restriction $\rho$ in Claim~\ref{claim:assigning}, with high probability the restricting ROBP $B|_\rho$ is a $(3,O(\log(n/\epsilon)),m)$-ROBPs. Now, similar to Section~\ref{sec:assign_all}, we wish to iteratively apply Claim~\ref{claim:assigning}, making the ROBP simpler in each step. We will have one progress measures on the restricted ROBP: the maximal number of colliding layers in a subprogram (denoted $\ell$). We show that the number of colliding layers reduces by a constant-factor in each iteration. To do so, we show a structural result on $(3,\ell,m)$-ROBPs that such ROBPs can be approximated by $(3,\ell, C^\ell)$-ROBPs for some constant $C$. This allows us to not worry about the number of sub-programs and use the number of colliding layers as a progress measure. Applying the restriction and the structure result $O(\log\log n)$ times, we end up with a ROBP where $\ell = O(\log \log n)$. We also show that ROBPs with few colliding layers are fooled by the INW generator. This follows from the results of \cite{BravermanRRY10}

%This should be thought of as the starting point. 


\ignore{Similarly to Section~\ref{sec:assign_all}, we wish to iteratively apply Claim~\ref{claim:assigning}, making the ROBP simpler in each step. We will have two progress measures on the restricted ROBP $D_1 \circ D_2 \circ \ldots \circ D_m$: the maximal number of colliding layers in a subprogram (denoted $\ell$), and the number of subprograms (denoted $m$). 
We shall show that the former reduces by a factor $2$ in each iteration, and the latter by a square-root (i.e., from $m$ to $\sqrt{m}$). 
This will allow us to reduce $m$ and $\ell$ to $\poly(1/\eps)$ and $O(\log(1/\eps)$ respectively, in $O(\log \log n)$ iterations.
In order to carry the argument, we will introduce a new error term in each application of the pseudo-random restriction. Luckily, the error term itself can be computed by the AND of 3ROBPs with at most $\ell$ colliding layers, which will allow us to fool it as well. In every step, we will show that the probabilities that the error terms equal $1$ remain small, for almost all restrictions.}

%\paragraph{Organization.} In Section~\ref{sec:ell-col-ROBPs}, we state some useful claims about ROBPs with at most $\ell$-colliding layers.  In Section~\ref{sec:negligible}, we show that PRGs fooling ROBPs with all vertices having high probability of reachability is enough to fool ROBPs with at most $\ell$-colliding layers. A similar reduction was used in \cite{CHRT17}, however here we are able to replace the length of the program with the number of colliding layers. In Section~\ref{sec:generator} we present the main details of the analysis of the generator, showing that after each application of the pseudorandom restriction, we can reduce the number of subprograms and the number of non-colliding layers in each subprogram. 

\subsection{Reducing the length of $(3,\ell,m)$-ROBPs}

Here we show that $(3,\ell,m)$-ROBPs can be approximated by $(3,\ell,C^\ell)$-ROBPs for some constant $C$. Another subtle aspect is that we need the approximation to work not just under the uniform distribution but also under the pseudo-random distribution. Fortunately, we are able to do so by arguing that the \emph{error} function detecting when our approximation is wrong is itself computable by a width $3$-ROBP with few colliding layers. 

\begin{lemma}[Main Structural Result]\label{lem:structcolliding}
For any $c > 0$, there exists  $C \geq 1$ such that the following holds. Any $(3,\ell,m)$-ROBP $B$ can be written as $B' + E$ where $B'$ is a $(3,\ell,C^\ell)$-ROBP and for any $x$, $|E(x)| \leq F(x) = \wedge_{i=1}^{m'} (\neg F_i(x))$ where $F_i$ are $(3,\ell,1)$-ROBPs on disjoint variables with $m' \leq C^\ell$ and $Pr[F(x) = 1] < 2^{-2^{c\ell}}$. 
\end{lemma}

For any vertex $v$ in a ROBP, we denote by $p_v$ the probability to reach $v$ under a uniform random assignment to the inputs.

\begin{claim}\label{claim:pv-large}
In a ROBP with width $w$ and at most $\ell$ colliding layers, every vertex whose $p_v>0$ has $p_v \ge 2^{-(\ell+1)\cdot(w-1)}$.
\end{claim}
We remark that this bound is exactly tight.
\begin{proof}
	We prove by induction (on the length of the program) that any program with width at most $w$,  exactly $\ell$ colliding layers and exactly $t$ reachable states in the last layer, has $p_v \ge 2^{-\ell\cdot(w-1) -(t-1)}$ for any reachable vertex $v$.
	Without loss of generality all nodes in the program are reachable (otherwise, we remove vertices that aren't reachable).
	
	Consider a program $B$ of length $n$ with  parameters $(t, \ell,w)$.
	Removing the last layer gives a program $B'$ of length $n-1$ with parameters $(t',\ell',w)$. By the induction hypothesis for any $v'$ in the last layer of $B'$ we have $p(v')\ge \delta$ for $\delta := 2^{-\ell'\cdot(w-1) -(t'-1)}$.
	
	We perform a case analysis. The following simple bound will be used in all cases. Let $v$ be a vertex in the last layer of $B$. Assume that $e$ edges go into $v$ from vertices in the second to last layer. Then, $p_v \ge \frac{1}{2} \cdot \delta \cdot e$. In particular, since we assumed all vertices are reachable, any vertex in the last layer have $p_v \ge \delta/2$. 

	If $\ell'=\ell$ and $t' = t$, then the last layer of edges in $B$ is regular, i.e., any node in the last layer in $B$ has exactly two ingoing edges. In this case any vertex $v$ in the last layer has  $p_v \ge \frac{1}{2} \cdot \delta\cdot 2= \delta = 2^{-\ell\cdot(w-1) - (t-1)}$.
	
	If $\ell'=\ell$, then $t' \le t$, since there are no collisions in the last layer of edges. Since we already handled the case $t'=t$, we may assume $t'\le t-1$.
	For any vertex $v$ in the last layer we have 
	$p_v \ge \delta/2 
	 \ge \frac{1}{2} \cdot 2^{-\ell'(w-1)-(t'-1)}
	 \ge \frac{1}{2} \cdot 2^{-\ell(w-1)-(t-2)} 
	 = 2^{-\ell(w-1)-(t-1)}$.
	
	If $\ell'<\ell$, then we consider two sub-cases:
	if $t = 1$ then only one vertex is reachable in the last layer and its $p_v$ equals $1$.
	Otherwise, $t\ge 2$ and $t'\le w$ thus $t' \le t+(w-2)$ and 
	 for any  vertex $v$ in the last layer we have  $p_v \ge \delta/2
	 \ge \frac{1}{2} \cdot 2^{-\ell'(w-1)-(t'-1)}
	 \ge \frac{1}{2} \cdot 2^{-(\ell-1)(w-1)-(t + (w-2)-1)} 
	 = 2^{-\ell(w-1)-(t-1)}$.
\end{proof}


\begin{claim}[XOR or colliding]\label{claim:XOR or Colliding}
Let $B$ be a 3ROBP with width-2 at the start and finish. 
Let $v_{1,1}$ and $v_{1,2}$ be the two start nodes.
Then, either $B$ computes the XOR of some of the input bits or there exists a string on which the two paths from $v_{1,1}$ and $v_{1,2}$ collide.
\end{claim}
\begin{proof}
	If $B$ can be computed by a 2ROBP then either this 2ROBP has some collision, or it computes the XOR of some of the input bits. Both cases satisfy the conclusion of the claim.
		
	For the rest of the proof assume that $B$ cannot be computed by a 2ROBP.
	Let $V_1, \ldots, V_{n+1}$ be the layers of vertices in $B$.
	We say that two states $u,v \in V_i$ are equivalent if the Boolean functions that are computed in the subprograms starting from $u$ and $v$ (resp.) are equal.
	Without loss of generality, any vertex in $B$ is reachable and there are no two states in $B$ that are equivalent.
	Let $i$ denote the index of the last layer in $B$ with width $3$.
	Since $B$ has width-2 at the end, $i<n+1$.
	
	There are  six edges between $V_{i}$ and $V_{i+1}$: three edges marked with $x_{i}=0$ and three edges marked with $x_{i}=1$.
	Since $|V_{i+1}|=2$, by Pigeon-hole principle, there are two edges marked with $x_i = 0$ going to some vertex $v \in V_{i+1}$, and two edges marked with $x_i = 1$ going to some vertex $v'\in V_{i+1}$ ($v'$ is not necessarily different from $v$). 
	These two pairs of edges cannot be starting from the same two nodes in $V_{i}$ since then the two nodes will be equivalent. 
	By renaming the nodes in $V_i$, we can assume that the two edges from $v_{i,1}, v_{i,2} \in V_i$ marked with $0$ go to $v\in V_{i+1}$ 
	and the two edges from $v_{i,2}, v_{i,3}\in V_i$ marked with $1$ go to $v'\in V_{i+1}$. 
	
	Since $v_{i,2}$ is reachable, there is an input $(x_1, \ldots, x_{i-1})$ that leads from $v_{1,1}$ or $v_{1,2}$ to $v_{i,2}$.
	Without loss of generality, we assume that $v_{i,2}$ is reachable from  $v_{1,1}$.
	Let $v'_i \in V_i$ be the vertex reached by following the input $(x_1, \ldots, x_{i-1})$ starting from the other start vertex $v_{1,2}$.
	If $v'_i = v_{i,2}$, then we already collided. If $v'_i = v_{i,1}$ then for the choice $x_i = 0$ the two paths defined by $(x_1, \ldots, x_i)$  starting from $v_{1,1}$ and $v_{1,2}$ collide on $v \in V_{i+1}$.
	Similarly, if $v'_i = v_{i,3}$, then for the choice $x_i = 1$ the two paths collide on $v' \in V_{i+1}$.
	\end{proof}


\begin{claim}[Decomposition]\label{claim:decomposition}
Let $B = D_1  \circ D_2  \circ \ldots \circ D_m$ be a ROBP where each $D_i$ is a 3ROBP with the first and last state spaces having width at most 2 and at most $\ell$ colliding layers.
Then, $B$ can be written as $B = D'_{1} \circ \ldots \circ D'_{t}$, for $t\le m$, where each subprogram $D'_{i}$ is a 3ROBP with the first and last state spaces having width at most 2 and at most $\ell$ colliding layers, and each subprogram $D'_i$, except for maybe $D'_1$, has a possible collision.
\end{claim}
\begin{proof}
	Recall that according to Claim~\ref{claim:XOR or Colliding} each $D_i$ is either ``XOR or colliding''.
	Apply induction on $m$.
	If $m=1$ we take $D'_1 = D_1$.
	For $m\ge 2$, 
	let $D'_1 \circ \ldots \circ D'_t$ be a decomposition for $D_1\circ \ldots \circ D_{m-1}$.
	We wish to show how to decompose $D_1\circ \ldots \circ D_{m}$.
	If $D_{m}$ computes an XOR function, then take $D'_{t} := D'_{t} \circ D_{m}$. Note that we haven't introduced any new colliding layers to $D'_t$ thus the decomposition is still valid. Otherwise, $D_{m}$ is a colliding 3ROBP. We set $D'_{t+1} := D_{m}$ and take the decomposition to be $D'_1 \circ \ldots \circ D'_{t+1}$.
	\end{proof}

\begin{claim}\label{claim:checkcollision}
Let $B$ be a 3ROBP with width-2 at the start, let $v_{1,1}, v_{1,2}$ be the two start nodes.
Suppose there are at most $\ell$ colliding layers in $B$. 
Assume there exists a string on which the two paths from $v_{1,1}$ and $v_{1,2}$ collide.
Let $u$ be the first vertex on which a collision can occur, and let $E$ be the event that a collision happened on $u$. Then, the event $E$ can be computed by another width-$3$ ROBP with at most $\ell$-colliding layers.
\end{claim}

\begin{proof}
To simulate whether the paths starting from $v_{1,1}$ and $v_{1,2}$ collide at $u$, we consider the ROBP that keeps the {\bf unordered} pair corresponding to the states of the two paths during the computation. In each layer until $u$, we have only states corresponding to $\{0,1\}, \{0,2\}$ or $\{1,2\}$. 
When we reach the layer of $u$ we have two states: ``accept'' (corresponding to a collision on $u$) and ``reject'' (corresponding to anything else).
Observe that any permutation layer in the original program defines a permutation layer in the new branching program (as a permutation over a finite set also defines a permutation over unordered pairs from this set). Thus, there are at most $\ell$ colliding layers.
\end{proof}

We are now ready to prove the main structural lemma Lemma~\ref{lem:structcolliding}. In the following, we consider branching programs with two initial nodes $v_{1,1}, v_{1,2}$. We interpret the value of the program on input $x$ as its average value starting for $v_{1,1}$ and $v_{1,2}$. That is, the program can get value $1, 0$ or $-1$ depending on whether the two paths from $v_{1,1}$ and $v_{1,2}$ accept or not.

{\color{red}Throughout this section we think of the error terms as $\{0,1\}$-indicators (instead of the usual $\pmone$-notation for other Boolean functions). We shall use $A \wedge B$ and $\bar{A}$ to denote the standard AND and negation of these Boolean values.
}

\begin{lemma}\label{lemma:short programs plus error term}
	Let $B =  D_1 \circ D_2  \circ \ldots \circ D_m $ be a ROBP where each $D_i$ is a 3ROBP with the first and last state spaces having width at most 2.
	Then, for any $j\in \{2, \ldots m\}$ we can write $B(x)$ as the sum of $(D_j \circ  \ldots \circ D_m)(x)$ and an error term $E(x)$, that is bounded in absolute value by $\bar{\FCol_j(x)} \wedge  \ldots \wedge \bar{\FCol_m(x)}$ where $\FCol_{i}(x)$ denotes the event that the two paths in $D_{i}$ collide on input $x$ at the first vertex on which it is possible to collide in $D_i$.
\end{lemma}
\begin{proof}
For $j=2, \ldots, m$, let $v_{j,1}$ and $v_{j,2}$ be the two nodes at the first layer of the subprogram $D_j$.
If $D_1$ has two nodes at the first layer, then denote them by $v_{1,1}$ and $v_{1,2}$, otherwise denote the single node by $v_{1,1}$. 
Let $x$ be an input to the branching program $B$.
Let $v^{*}_j \in \{v_{j,1}, v_{j,2}\}$ be the vertex in the path defined by $x$ from $v_{1,1}$ right after the end of $D_1 \circ \ldots \circ D_{j-1}$.
Let $v'_j$ be the other vertex in the layer of $v^{*}_j$.
If the two paths defined by $x$ from $v^{*}_j$ and $v'_j$ collide at some point, then the value of $B(x)$ equals the value of $(D_j \circ \ldots \circ D_m)(x)$.
If the two paths do not collide, then $(D_j \circ \ldots \circ D_m)(x) = 0$, since it is the average of two paths with different outcomes, thus $E(x)= B(x) - (D_j \circ \ldots \circ D_m)(x)$ is at most $1$ in absolute value. 
Furthermore, in such a case, for all $i = j, \ldots, m$ it holds that both paths in the subprogram $D_i$ starting from $v_{i,1}$ and $v_{i,2}$ on input $x$ do not collide, i.e., $\FCol_i(x) = 0$.
Overall, we got that $B(x) = E(x) + (D_j \circ \ldots \circ D_m)(x)$, and whenever $E(x)\neq 0$, it holds that $\bar{\FCol_j(x)} \wedge  \ldots \wedge \bar{\FCol_m(x)} =1$.
\end{proof}

\begin{proof}[Proof of Lemma~\ref{lem:structcolliding}]
Fix a constant $C$ to be chosen later. Let $B$ be a $(3,\ell,m)$-ROBP. Let $B = D_1' \circ D_2' \circ \cdots D_t'$ for $t \leq m$ be the decomposition as guaranteed by Claim~\ref{claim:decomposition}. If $t \leq C^\ell$, there is nothing to prove. Suppose that $t > C^\ell$. Let $j = t - C^\ell > 1$. Let $B' = D_j' \circ D_{j+1}' \circ \cdots \circ D_t'$ and let $F(x) = \bar{\FCol_j(x)} \wedge  \ldots \wedge \bar{\FCol_m(x)}$ where $\FCol_{i}(x)$ denotes the event that the two paths in $D_{i}'$ collide on input $x$ at the first vertex on which it is possible to collide in $D_i'$. Then, by the previous claim, we can write $B = B' + E$ where for any input $x$, $|E(x)| \leq F(x)$. We will argue that this gives the desired decomposition. 

Fix $i \in \{j, j+1,\ldots, t\}$. By Claim~\ref{claim:checkcollision} $\FCol_i(x)$ is a $(3,\ell,1)$-ROBP. Further, as each $D_i'$ has a possible collision, each $\FCol_{i}$ has an accepting input. Since $\FCol_i$ is also a $(3,\ell,1)$-ROBP, by Claim~\ref{claim:pv-large}, for a uniformly random $x$ we get $\Pr[\FCol_{i}(x) = 1] \geq 2^{-2(\ell+1)}$. Therefore,
\begin{align*}
\Pr[F(x) = 1] = \prod_{i=j}^{t} (1 - \Pr[\FCol_i(x) = 1]) \leq (1 - 2^{-2(\ell+1)})^{C^\ell} \leq 2^{-2^{c\ell}}
\end{align*}
for $C = 8 \cdot 2^c$. This proves the claim. 
\end{proof}

\subsection{PRGs for ROBPs with few colliding layers}
In this section we show that we can $\epsilon$-fool PRGs with at most $\ell$-colliding layers with $\tilde{O}((\log(\ell/\epsilon)) (\log n))$ seed-length. 

\begin{theorem}\label{thm:foolfewcollisions}
For any $\epsilon > 0$, there is an explicit PRG that $\eps$-fools ordered width $w$-ROBPs with length $n$ and at most $\ell$ colliding layers using seed length
$$O((\log \log n + \log(1/\eps) + \log(\ell) + w) \log n.$$
\end{theorem}

The above relies on the PRGs for regular branching programs and generalizations of them due to Braverman, Rao, Raz, and Yehudayoff \cite{BravermanRRY10}. In the following, we call a read-once branching program $B$ an $(n,w,\delta)$-ROBP if $B$ is of length-$n$, width-$w$ and for all reachable vertices $v$ in $B$ we have $p_v(B) \ge \delta$ where  $$p_v(B):= \Pr_{x \sim U_n}[\text{reaching $v$ on the walk on $B$ defined by $x$}].$$

We start by quoting a result by Braverman et al. \cite{BravermanRRY10}.

\begin{theorem}[\cite{BravermanRRY10}]\label{thm:BRRY}
There is an explicit PRG that  $\eps$-fools all $(n,w,\delta)$-ROBPs, using seed length
$$O(\log \log n + \log(1/\eps) + \log(1/\delta) + \log(w)) \cdot \log n.$$
\end{theorem}

Next, we reduce the task of fooling ROBPs with at most $\ell$-colliding layers to the task of fooling ROBPs with no negligible vertices.
The reduction is similar to that in \cite{CHRT17}. The main difference is that we simulate a ROBP with width $w$ by a ROBP of width $w+1$ that has no negligible vertices by adding a new sink state that should be thought of as ``immediate reject''. 
This change seems essential in our case, and the reduction from \cite{CHRT17} does not seem to satisfy the necessary properties here.



\begin{lemma}\label{lemma:negligible}
Let $\delta \le 2^{-(w-1)}$.
Let $\D$ be a distribution on $\pmone^n$ that $\eps$-fools all $(n,w+1,\delta)$-ROBPs.
Then, $\D$ also fools width-$w$ ROBPs with at most $\ell$ colliding layers with error at most $(\ell w +1)\cdot \eps +  (2^{w} w \ell) \cdot \delta$.
\end{lemma}
\begin{proof}
	Let $\D$ be a distribution on $\pmone^n$ that $\eps$-fools all $(n,w+1,\delta)$-ROBPs.
The first observation is that any distribution $\D$ that fools all $(n,w+1,\delta)$-ROBPs also fools prefixes of these programs. The reason is simple because to simulate the prefix of length-$k$ of a $(n,w+1,\delta)$-program $B$, one can simply reroute the last $n-k$ layers of edges in $B$ so that they would ``do nothing'', i.e. that they would be the identity transformation regardless of the values of $x_{k+1}, \ldots, x_n$.

Let $B$ be a length $n$ width-$w$ ROBP with at most $\ell$ colliding layers.
Next, we introduce $B'$, an $(n,w+1,\delta)$-ROBP, that would help bound the difference between 
$$
B(U_n) := \Pr_{x\sim U_n} [B(x)=1] \quad\text{and}\quad 
B(\D) := \Pr_{x\sim \D} [B(x)=1]\;,
$$ 
where $U_n$ is the uniform distribution over $\pmone^n$.
Let $B'$ be the the following modified version of $B$.
To construct $B'$ we consider a sequence of $\ell+1$ branching programs $B_0, \ldots, B_\ell$ where $B_0 = B$ and $B' = B_\ell$.
We initiate $B_0$ with $B$.
For $i=1, \ldots, n$ we take $B_i$ to be $B_{i-1}$ except we may reroute some of the edges in the $i$-th layer (of edges).
We explain the rerouting procedure. Let $i_1, \ldots, i_\ell$ be the colliding layers in $B$.
For $j=1,\ldots, \ell$ we calculate the probability to reach vertices in layer $V_{i_j}$ of $B_{j-1}$. If some vertex $v$ in the $i_j$-th layer has probability less than $2^{w-1} \cdot \delta$, then we reroute the two edges going from the vertex $v$ to go to ``immediate reject''. We denote by $\Vsmall$ the set of vertices for which we rerouted the outgoing edges from them.

First, we claim that any reachable vertex $v$ in $B_\ell$ has $p_v  \ge \delta$. 
Let $i_{\ell+1} = n+1$ for convenience. 
We apply induction and show that for $j=0,1, \ldots, \ell$ any vertex reachable by $B_{j}$ in layers $1,\ldots, i_{j+1}$ has $p_v \ge \delta$.
The base case holds because up to layer $i_{1}$ the branching program has no colliding layers and we may apply Claim~\ref{claim:pv-large} to get that $p_v \ge 2^{-(w-1)} \ge \delta$.
%This obviously hold for the starting vertex in $B_0$ which has $p_v = 1$.
To apply induction assume the claim holds for $B_{j-1}$ and show that it holds for $B_{j}$.
The claim obviously holds for all vertices in layers $1, \ldots, i_{j}$ in $B_j$ since we didn't change any edge in those layers going from  $B_{j-1}$ to $B_j$.

Let $v$ be a reachable vertex in layer $i$ where $i_j < i \le i_{j+1}$ in $B_j$. It means that there is a vertex in $v'$ with $p(v') \ge 2^{w-1}\cdot \delta$ in the $i_{j}$-th layer of $B_j$ (and also in $B_{j-1}$) and a path going from $v'$ to $v$. 
Looking at the subprogram from $v'$ to $v$ we note that this is a subprogram with no colliding edges (only the first layer has the potential to be colliding, but in a ROBP the first layer can never be colliding has there is only one $0$-edge and only one $1$-edge). By Claim~\ref{claim:pv-large} the probability to get from $v'$ to $v$ is at least $2^{-(w-1)}$.
Thus, the probability to reach $v$ is at least $p(v') \cdot \Pr[\text{reach $v$}|\text{reached $v'$}] \ge 2^{w-1} \cdot \delta \cdot 2^{-(w-1)} = \delta$.

To bound $|B(U_n)-B(\D)|$ we use the triangle inequality:
\begin{equation}\label{eq:hybrid}|B(U_n)-B(\D)|  \le |B(U_n)-B'(U_n)| + |B'(U_n)-B'(\D)| + |B'(\D)-B(\D)|\end{equation}
and bound each of the three terms separately:
\begin{enumerate}
\item	
The first term is bounded by the probability of reaching one of the nodes in $\Vsmall$ in $B'$ when taking a uniform random walk. This follows since if the path defined by $x$ didn't pass through $\Vsmall$ then we would end up with the same node in both $B$ and $B'$ (since no rerouting affected the path).
By union bound, the probability to pass through $\Vsmall$ is at most $|\Vsmall| \cdot 2^{w-1} \cdot \delta$.
\item The second term is at most $\eps$ since the program $B'$ has all $p_v \ge \delta$.
\item Similarly to the first term, the third term is bounded by the probability of reaching one of the nodes in $\Vsmall$ in $B'$ when taking a walk sampled by $\D$.
	\begin{align*}
		|B'(\D) - B(\D)|
		&\le \Pr_{x\sim \D}[\text{reaching $\Vsmall$ on the walk on $B'$ defined by $x$}] \\
		&\le \sum_{v\in \Vsmall} \Pr_{x\sim \D}[\text{reaching $v$ on the walk on $B'$ defined by $x$}]
	\end{align*}
	However since $\D$ is pseudorandom for prefixes of $B'$, for each $v \in \Vsmall$ the probability of reaching $v$ when walking according to $\D$ is  $\eps$-close to the probability of reaching $v$ when walking according to $U_n$.
	\begin{align*}
		|B'(\D) - B(\D)|
		&\le \sum_{v\in \Vsmall} \Pr_{x\sim U_n}[\text{reaching $v$ on the walk on $B'$ defined by $x$}] + \eps\\
		&= \sum_{v\in \Vsmall} \left(p_v(B') + \eps\right)
		\le |\Vsmall| \cdot (\eps + 2^{w-1} \delta)
	\end{align*}
\end{enumerate}

Summing the upper bound on the three terms in Eq.~\eqref{eq:hybrid} gives:
\[|B(U_n)-B(\D)| \le |\Vsmall|\cdot (\eps +  2^{w} \delta) + \eps  \le \ell w \cdot (\eps +  2^{w} \delta) + \eps.\;\qedhere\]
\end{proof}


\begin{proof}[Proof of Theorem~\ref{thm:foolfewcollisions}]
	Take the BRRY-generator. %(which is the INW-generator).
	Take $\eps' = \eps/(2(\ell w+1))$ and $\delta = \eps'/2^{w}$. 
	Apply Lemma~\ref{lemma:negligible} and Theorem~\ref{thm:BRRY} with $\delta$ and $\eps'$.
	Then, the error of the generator guaranteed by Theorem~\ref{thm:BRRY}  on the class of ROBPs with width $w$ length $n$ and at most $\ell$ colliding layers is at most 
	$(\ell w+1) \cdot \eps' + (2^{w} \cdot w\cdot \ell) \cdot \delta \le \eps/2 + \eps/2 = \eps$, 
	and the seed length is 
	$$O(\log \log n + \log(1/\eps') + \log(1/\delta) + \log(w)) \cdot \log(n)$$
	which is at most $O(\log \log n + \log(1/\eps) + \log(\ell) + w)\cdot \log(n)$.
	\end{proof}
	
\subsection{Proof of Theorem~\ref{thm:main-3ROBP-ordered}}
We are now ready to prove our main result on fooling width 3 ROBPs. Our generator is obtained by applying Claim~\ref{claim:assigning} iteratively for $O(\log \log n)$ times and then using a PRG for fooling width 3 ROBPs with at most $O(\log(n/\epsilon))$ colliding layers as in Theorem~\ref{thm:foolfewcollisions}. The intuition is as follows. 

Let $B$ be a 3ROBP and let $\rho_0$ be a pseudorandom assignment as in Claim~\ref{claim:assigning}. We first show that with probability at least $1-\epsilon/n$ over $\rho_0$,  $B^0 = B|_{\rho_0}$ is a $(3,\ell_0,m)$-ROBP for $\ell_0 = O(\log(n/\epsilon))$. Let $B^0 = D_1^0 \circ D_2^0 \circ \cdots \circ D_m^0$ where each $D_i^0$ has at most $\ell_0$ colliding layers and begins and ends with width two layers. Let $\rho_1$ be an independent pseudo-random assignment as in Claim~\ref{claim:assigning}. Then $B^1 \equiv B^0|_{\rho_1} = D_1^0|_{\rho_1} \circ D_2^0|_{\rho_2} \cdots \circ D_m^0|_{\rho_1}$ and it is easy to check that with probability at least $\epsilon + 2^{-O(\ell)}$, each $D_i^0|_{\rho_1}$ has at most $\ell_0/2$ colliding layers. Ideally, we would like to apply a union bound over the different $D_i^0$ and conclude that $B^1$ is a $(3,\ell_0/2,m)$-ROBP. However, $m$ could be much larger than $C^\ell$ to use this approach. Thus, by a union bound if $m \leq C^\ell$ for a constant, then choosing $\rho_1$ with appropriate parameters gives us that with probability at least $1 - 2^{-O(\ell)}$, $B^1$ is a $(3,\ell_0/2,m')$-ROBP. However, $m$ could be much larger than $C^\ell$ to begin with. Nevertheless, we know that we can always approximate $B^0$ with a $(3,\ell_0,C^\ell)$-ROBP by Lemma~\ref{lem:structcolliding}. This approximation allows us to apply the union bound and conclude that the number of colliding layers in each block decreases by a factor of $2$. We iterate this approach until the number of colliding layers is $O(\log \log n)$ when we use the PRG from Theorem~\ref{thm:foolfewcollisions}.

To carry the induction forward as outlined above, we need the following technical lemma. 

\begin{claim}\label{claim:simplifyerror}
For all constants $c$ and $C > 8$, there exists $\alpha \in (0,1)$ such that the following holds. Let $\ell \leq \log(n/\epsilon)$, and let $F(x) = \wedge_{i=1}^m (\neg F_i(x))$ where $F_i$ are $(3,\ell,1)$-ROBPs on disjoint variables with $m \leq C^\ell$ and $Pr[F(x) = 1] < 2^{-2^{c \ell}}$. Then, for a sufficiently small constant parameter $\alpha$, and $\rho$ a pseudorandom assignment as in Claim~\ref{claim:assigning}, with probability at least $1 - C^{\ell/2}$, we can write $F_\rho(x) = \wedge_{i=1}^{m'} (\neg F_i'(x))$ where $F_i'$ are $(3,\ell/2,1)$-ROBPs on disjoint variables with $m' \leq C^{\ell/2}$ and $Pr[F(x) = 1] < 2^{-2^{c \ell/2}} + n^2 \epsilon$. 
\end{claim} 
\begin{proof}
	First, we show that with high probability, each $F_i$ has at most $\ell/2$ colliding layers under the pseudo-random restriction. 
	To see it, note that any colliding layer that is restricted can be either:
	\begin{itemize}
		\item Assigned to a value that reduces the width of the original program to $2$, and thus the width of $F_i$ to 1, in which case every previous layer in $F_i$ is not affecting the value of $F_i$.
		\item Assigned to a value that applies a permutation on the states of the program, thus reducing the number of colliding layers.
	\end{itemize}  
	In either case, if $k$ colliding layers are unassigned, then $F_i|_{\rho}$ can be a computed by a 3ROBP with at most $k$ colliding layers.
	%Since the selection is done using a $\poly(\eps/n)$-biased distribution, 
	By Claim~\ref{claim:assigning} the probability that less than $\ell/2$ colliding layers are unassigned is at least $1- 2 \alpha^{-\ell}$.
	Taking a union bound over the $C^{\ell}$ functions $\{F_i\}_{i=1}^{m}$ we get that with probability at least $1-2 (C \alpha)^{\ell}$ all functions $\{F_i|_{\rho}\}_{i=1}^{m}$ can be computed by 3ROBPs with at most $\ell/2$ colliding layers.
	
	We move to show that with high probability at least $C^{\ell/2}$ of the functions $F_i|_{\rho}$ are non-zero.
	We apply the second moment method. 
	Denote by $p_i = \Pr_{z\sim U}[F_i(z)]$ for $i=1, \ldots, m$.
	Let $A_1, \ldots, A_m$ be the events that $\{(F_i)|_{\rho}(y) =1\}_{i=1}^{m}$ respectively, where $\rho$ is the pseudo-random restriction from Claim~\ref{claim:assigning} and $y \sim U_{\rho^{-1}(*)}$.
	By Claim~\ref{claim:assigning}
	$$\Pr[A_i] = \Pr_{\rho,y}[(F_i)|_{\rho}(y) =1] = \E_{z\sim U}[F_i(z)] \pm \poly(\eps/n) = p_i \pm \poly(\eps/n),$$
	and by the next lemma, whose proof is deferred to Appendix~\ref{app:composition 3ROBPs}, we get
	\begin{align*}\Pr[A_i \wedge A_j] &= \Pr_{\rho,y}[(F_i)|_{\rho}(y) \wedge (F_j)|_{\rho}(y) =1]\\
	&= \E_{z\sim U}[F_i(z)\wedge F_j(z)] \pm \poly(\eps/n) = p_i p_j \pm \poly(\eps/n).
	\end{align*}	

\begin{lemma}\label{lemma:fooling H of width-3}
Let $f_1, \ldots, f_k$ be 3ROBPs on disjoint sets of variables of $[n]$. 
Let $H:\pmone^k \to \pmone$ be any Boolean function.
Then, $f = H(f_1, f_2, \ldots, f_k)$ is $\eps\cdot ((n+k)/k)^k$-fooled by the pseudorandom partial assignment in Claim~\ref{claim:assigning}.
\end{lemma}
	Thus, the covariance of the two events is at most $\eps' := \poly(\eps/n)$.	
	We get that the probability that at most $\sum_{i}{p_i/2}$ of the events $A_1, \ldots, A_m$ occur is at most 
	$(\sum_{i=1}^{m} {p_i} + \eps' m^2)/(\sum_{i=1}^{m}p_i/2 - \eps' m)^2 \le O(1/\sum_{i=1}^{m}p_i) \le  2^{2(\ell+1)}/m$.
	In the complement event, at least $\sum_{i}{p_i/2}$ of the events $A_1, \ldots, A_m$ occur, and in particular at least $\sum_{i}{p_i/2} \ge m \cdot 2^{-2(\ell+1)-1} \ge C^{\ell/2}$ (using $\ell \ge 24$) of the restricted functions $\{F_i|_\rho\}_{i=1}^m$ are non-zero. 

Suppose that at least $C^{\ell/2}$ of the restricted functions $\{F_i|_\rho\}_{i=1}^m$ are non-zero, and that all restricted functions has at most $\ell/2$ colliding layers. By the above analysis this happens with probability at least $1-C^{-\ell/2}$. Under this assumption, we can reduce the number of functions to be exactly $m' = C^{\ell/2}$, resulting in an upper bound on $E|_{\rho}$ which we denote by $\mE(x) \triangleq \bar{F'_{1}(x)} \wedge \ldots \wedge \bar{F'_{m'}(x)} $.
%It remains to show that the probability that $\mE(x)=1$ is exponentially small in $\ell$. Indeed, using  Claim~\ref{claim:pv-large}, $\mE(x)=1$ with probability at most $(1-2^{-2(\ell/2+1)})^{20^{\ell/2}} \ll 2^{-\ell}$.
%
%	It remains to show that we can reduce the number of  functions in case there are more than $20^{\ell/2}$ functions that are non-zero after the restriction without increasing the error by too much.
%	We simply note that whenever more than $20^{\ell/2}$ functions remained non-zero we can simply take a subset of $20^{\ell/2}$ non-zero of them. In this case the probability of $\vee_{i} (\FCol_i)|_{\rho}(y) =1$ is still greater than $1-(1-2^{-2(\ell/2+1)})^{20^{\ell/2}} \gg 1-2^{-\ell}$ using Claim~\ref{claim:pv-large}. 
%	Thus, the error probability remains exponentially small in $\ell$.
	\end{proof}

We are now ready to prove the main Theorem of this section, Theorem~\ref{thm:main-3ROBP-ordered}.
\begin{proof}[Proof of Theorem~\ref{thm:main-3ROBP-ordered}]
Fix a constant $c \geq 1$, and let $C$ be the constant from Lemma~\ref{lem:structcolliding} applied to $c$. Let $\alpha \in (0,1)$ be a constant to be chosen later. Let $\ell_0 = \lceil\log_2 (n/\epsilon) \rceil$. Let $k$ be a parameter to be chosen later and let $\ell_i = \ell_0/2^i$ for $1 \leq i \leq k$. 

Our generator is as follows. First choose $\rho_0, \rho_1,\ldots,\rho_k$ independent pseudo-random restrictions as in Claim~\ref{claim:assigning} with parameter $\alpha$. After iteratively applying the restrictions $\rho_0,\rho_1,\ldots,\rho_k$, we set the remaining bits using the generator from Theorem~\ref{thm:foolfewcollisions} for a parameter $\ell = \ell_k \cdot C^{\ell_k}$ and error parameter $\epsilon'$ to be chosen later. Let $Y$ be the output distribution of the generator. 

Let $B^0 = B|_{\rho_0}$. We first claim that $B^0$ is a $(3,\ell_0,m)$-ROBP with high probability. In the following, let $bias(f)$ denote the expectation of a function $f$ under a uniformly random input. In the following let $X$ be uniformly random over $\{0,1\}^n$. 
\begin{claim}\label{claim:main1}
With probability at least $1-\epsilon/n$, $B|_{\rho_0}$ is a $(3,\ell_0,m)$-ROBP and $E_{\rho_0,X}[B|_{\rho_0}(X)] = E_X[B(X)] \pm \epsilon/n$. 
\end{claim}

For $0 \leq i \leq k$, let $\rho^i = \rho_0 \circ \cdots \rho_i$. We will show the following claim by induction on $i$.
\begin{claim}\label{claim:main2}
For $0 \leq i \leq k$, with probability at least $1- (i+1) 2^{-c \ell_i}$, $B|_{\rho^i}$ can be written as $B^i + E^0 + E^1 + \cdots + E^i$ where $B^i$ is a $(3,\ell_i,C^{\ell_i})$-ROBP and the error terms $E^j$ for $0 \leq j \leq i$ satisfy:
\begin{enumerate}
\item $|E^j(x)| \leq F^j(x)$ with $F^j(x) = \wedge_{h=1}^{m_j} (\neg F_h^j(x))$ where $F_h^i$ are $(3,\ell_i,1)$-ROBPs on disjoint sets of variables and $m_j \leq C^{\ell_i}$. 
\item $Pr[F^j(x) = 1] \leq 2^{-2^{c\ell_i}} + n^2 i \epsilon$.
\end{enumerate}

Furthermore, $E_{\rho^i,X}[B|_{\rho^i}(X)] = E_X[B(X)] \pm (i+1) \epsilon/n$. 
\end{claim}

A crucial point in the above is that the functions $F^0,\ldots,F^i$ bounding the error terms are a conjunctions of negations of $(3,\ell_i,C^{\ell_i})$-ROBPs and there are at most $C^{\ell_i}$ in each of them. 
\begin{proof}
For $i= 0$, the claim follows immediately by applying Lemma~\ref{lem:structcolliding} to $B|_{\rho_0}$. Now, suppose the claim is true for $i$. Suppose, we can write $B|_{\rho^i} = B^i + {\cal E}^i$, where ${\cal E}^i = E^0 + E^1 + \cdots + E^i$ as in the claim. By the induction hypothesis, this happens with probability at least $1 - i \cdot 2^{-c \ell_i}$. 

Clearly, $B|_{\rho^{i+1}} = B^i_{\rho_{i+1}} + {\cal{E}}^i|_{\rho_{i+1}}$. Let $B^i = D_0 \circ D_1 \circ \cdots \circ D_{m'}$ be a decomposition where each $D_j$ has at most $\ell_i$ colliding layers, starts and ends with width two layers and $m' \leq C^{\ell_i}$.  

Now, observe that as each $D_0$ has at most $\ell_i$ colliding layers, the probability that at least $\ell_i/2$ of these colliding layers are unfixed under $\rho_{i+1}$ is at most $ 2^{\ell_i} \alpha^{\ell_i/2}$ by Claim~\ref{claim:assigning}. Thus, by a union bound over $0 \leq j \leq m'$, with probability at least $1 - 2^{\ell_i} \alpha^{\ell_i/2} \cdot C^{\ell_i}$, over $\rho^{i+1}$, $B^i_{\rho_{i+1}}$ is a $(3,\ell_i/2,C^{\ell_i})$-ROBP.  Now, conditioning on this event, by Lemma~\ref{lem:structcolliding}, we can write $B^i_{\rho_{i+1}}$ as $B^{i+1} + E^{i+1}$, where $B^{i+1}$ is a $(3,\ell_{i+1}, C^{\ell_{i+1}})$-ROBP and $E^{i+1}$ satisfies the conditions of the claim. Thus, with probability at least $1 - i \cdot 2^{-c \ell_i} - 2^{-c \ell_{i+1}} \geq 1 - (i+1) 2^{-c \ell_{i+1}}$, 
\begin{align*}
B|_{\rho^{i+1}} &= B^i|_{\rho_{i+1}} + {\cal E}^i|_{\rho_{i+1}} \\
&= B^{i+1} + {\cal E}^i|_{\rho_{i+1}} + E^{i+1},
\end{align*}
where $B^{i+1}$, and $E^{i+1}$ satisfy the conditions of the claim. 

We just need to argue that ${\cal E}^i|_{\rho_{i+1}}$ can be written in the requisite form. To this end, note that for $0 \leq j \leq i$, $|E^j|_{\rho_{i+1}}| \leq F^j|_{\rho_{i+1}}$. By the induction hypothesis, we can write $F^j = \wedge_{h=1}^{m_j} (\neg F_h^j(x))$ where $F_h^i$ are $(3,\ell_i,1)$-ROBPs on disjoint sets of variables and $m_j \leq C^{\ell_i}$. We can now apply Claim~\ref{claim:simplifyerror} to conclude that with probability at least $1 - O(\alpha)^{\ell_j}$, we can write $F^j|_{\rho_{i+1}} = \wedge_{h=1}^{m_j'} (\neg H_h^j(x))$ where $H_h^j$ are $(3,\ell_i/2,1)$-ROBPs on disjoint sets of variables and $m_j' \leq C^{\ell_i/2}$. This satisfies the constraints of the claim. 

Adding up the failure probabilities over the choice of $\rho_{i+1}$, we get the desired decomposition for $i+1$ with probability at least 

$$1 - 2^{\ell_i} \alpha^{\ell_i/2} \cdot C^{\ell_i} - \sum_{j=0}^i O(\alpha)^{\ell_j} \geq 1 - 2^{-c \ell_j},$$
for $\alpha$ a sufficiently small constant. 


The furthermore part follows immediately from Claim~\ref{claim:assigning}. The claim now follows by induction. 
\end{proof}

We are now ready to prove the theorem. By the above claim, we have that with probability at least $1 - (k+1) 2^{-c \ell_k}$ over the choice of $\rho_0, \rho_1,\ldots,\rho_k$, we can write
$$B|_{\rho^k} = B^k + E^0 + \cdots + E^k,$$
where $B^k$ is a $(3, \ell_k,C^{\ell_k})$-ROBP and $E^0,\ldots,E^k$ can be bounded by functions $F^0,\ldots,F^k$ as a conjunction of negations of $C^{\ell_k}$ $(3,\ell_k,1)$-ROBPs. 

Note that each such $F^j$ can be written as a width $4$ ROBP, say $H^j$, by adding an additional layer to compute the conjunction and that the number of collisions in the width $4$ ROBP is at most $\ell_k \cdot C^{\ell_k}$. Therefore, if we let $Y$ be the output distribution of the generator from Theorem~\ref{thm:foolfewcollisions} with $\ell = \ell_k \cdot C^{\ell_k}$ and error parameter $\epsilon'$, we get that for all $0 \leq j \leq k$, and $X$ uniformly random over $\{0,1\}^n$, 
\begin{align*}
E[B^k(X)] &= E[B^k(Y)] \pm \epsilon'\\
E[|E^j(Y)|] &\leq E[H^j(Y)] \leq E[H^j(X)] + \epsilon' \leq 2^{-2^{-c\ell_k}} + n^2 k \epsilon + \epsilon'.
\end{align*}

Combining the above inequalities we get that with probability at least $1 - (k+1) 2^{-c \ell_k}$ over the choice of $\rho_0, \rho_1,\ldots,\rho_k$
$$E_X [B|_{\rho^k}(X)] = E_Y [B|_{\rho^k}(Y)] + \epsilon' + k(2^{-2^{-c\ell_k}} + n^2 k \epsilon + \epsilon').$$
Finally, as we also have that 
$$E_{\rho_0,\ldots,\rho_k}[ B|_{\rho^k}(X)] = E[B(X)] \pm (k+1) \cdot \epsilon/n,$$
we get
$$E_{\rho_0,\ldots,\rho_k}[B|_{\rho^k}(Y)] = E[B(X)] \pm \left( (k+1) \cdot \epsilon/n + (k+1) 2^{-c \ell_k} + \epsilon' + k(2^{-2^{-c\ell_k}} + n^2 k \epsilon + \epsilon')\right).$$

Note that if we set $k = \log(\ell_0/(\log(1/\delta))) = O(\log \log n)$, so that $\ell_k = \ell_0/2^k = \log(1/\delta)$, $\epsilon = \delta/n^3$ and $\epsilon' = \delta/k$, the above error bound becomes
$$E_{\rho_0,\ldots,\rho_k}[B|_{\rho^k}(Y)] = E[B(X)] \pm O(\log \log n) \delta^2 E[B(X)] \pm O(\delta).$$

Finally, we estimate the seed-length of our generator. Choosing the random restrictions takes $\tilde{O}(\log(n/\epsilon)) = \tilde{O}(\log(n/\delta))$ random bits. Sampling $Y$ as per Theorem~\ref{thm:foolfewcollisions} needs seed-length
$$O(((\log \log n) + \log(1/\epsilon') + \log(L))(\log n)) = O((\log \log n) + \log(1/\delta)) \cdot (\log n).$$

Thus, the final seed-length is $\tilde{O}(\log(n/\delta)) + O(\log(1/\delta) (\log n))$. The theorem follows. 
\end{proof}

\ignore{
\subsection{The Generator}\label{sec:generator}

In this section, we present the generator and its analysis.
Throughout the section, we deal with ROBPs of the form
$D_1 \circ D_2  \circ \ldots \circ D_m$ where each $D_i$ is a 3ROBP with the first and last state spaces having width at most 2, and each $D_i$ has at most $\ell$ colliding layers. 
The next lemma shows that we can approximate any such ROBP by a shorter ROBP plus an error term, which  in itself can be computed by the AND of 3ROBPs with at most $\ell$ colliding layers.




The next lemma shows that in ROBPs composed of smaller subprograms with at most $\ell$ non-permuting layers, the maximal number of non-permuting layers decrease from  $\ell$ to  $\ell	/2$ under the pseudo-random restriction from Claim~\ref{claim:assigning} with high probability.

\begin{lemma}\label{lemma:ell reduces}
	Let $\ell \le 10\log(n/\eps)$.
	Let $B = D_1 \circ D_2 \circ \ldots \circ D_m$ be a ROBP where each $D_i$ is a 3ROBP with the first and last state spaces having width at most 2, and at most $\ell$ colliding layers.
	Suppose $m \le 20^{\ell}$.
	Let $\rho$ be the pseudo-random restriction from Claim~\ref{claim:assigning}.
	Then, with probability at least $1-2^{-\ell}$ the restricted ROBP $B|_{\rho}$ can be written as $D'_{1} \circ \ldots \circ D'_{m'}$  where each $D'_i$ is a 3ROBP with the first and last state spaces having width at most 2, and at most $\ell/2$ colliding layers.
\end{lemma}

\begin{proof}
	It suffices to show that with high probability under the pseudo-random restriction, each $D_i$ can be written as $D'_{i,1} \circ \ldots \circ D'_{i,r_i}$ where each $D'_{i,j}$ is a 3ROBP with the first and last state spaces having width at most 2, and at most $\ell/2$ colliding layers.
To see it, note that any colliding layer that is restricted can be either:
	\begin{itemize}
		\item assigned a value that reduces the width to $2$, breaking the subprogram $D_i$ into two subprograms. This step can make the layer before the current layer become colliding, but all other non-colliding layers remain non-colliding.
		\item assigned a value that applies a permutation on the states of the program, thus reducing the number of colliding layers.
	\end{itemize} 
		In either cases, if less than $\ell/2$ colliding layers are unassigned, then $D_i$ can be written as $D'_{i,1} \circ \ldots \circ D'_{i,r_i}$ with each $D'_{i,j}$ having at most $\ell/2$ colliding layers.
	By Claim~\ref{claim:assigning}, the probability that less than $\ell/2$ colliding layers are unassigned is at least $1-40^{-\ell}$.

%	In either cases, if among every consecutive $\ell/2$ colliding layers, at least one is assigned under the pseudo-random restriction, then $D_i$ can be written as $D'_{i,1} \circ \ldots \circ D'_{i,r_i}$ with each $D'_{i,j}$ having at most $\ell/2$ colliding layers.
%	Claim~\ref{claim:assigning} guarantees that the probability that any set of $\ell/2$ bits are unassigned is at most $0.0002^{\ell}$.
%	 Thus, with probability at most $40^{-\ell}$, there are $\ell/2$ consecutive colliding layers that are unassigned by the pseudo-random restriction.
	
	Taking a union bound over all $m\le 20^{\ell}$ subprograms $D_i$, shows that with probability at least $1-2^{-\ell}$, the restricted function can be computed by 
	$D'_{1,1} \circ \ldots \circ D'_{1,r_1} \circ \ldots \circ D'_{m,1} \circ \ldots \circ D'_{m,r_m}$ 
	where each $D'_{i,j}$ having at most $\ell/2$ colliding layers.
\end{proof}


The next lemma shows that the error term simplifies under a pseudo-random restriction, while maintaining a small probability of acceptance.

\begin{lemma}\label{lemma:error under restriction}
Let $\ell \ge 24$.
	Let $m= 20^{\ell}$. Let $E = \bar{\FCol_1} \wedge \ldots \wedge \bar{\FCol_m}$ where each $\FCol_i$ is a non-zero unordered-pairs-3ROBP with at most $\ell$ colliding layers.
	Let $\rho$ be the pseudo-random restriction that keeps each variable alive with probability at most $0.0001$ from 
	Claim~\ref{claim:assigning}.
	Then, with probability at least $1-2\cdot 2^{-\ell}$ the restricted function $E|_{\rho}(x)$ can be upper bounded by 
	$\mE(x) \triangleq \bar{\FCol'_{1}(x)} \wedge\ldots \wedge \bar{\FCol'_{\sqrt{m}}(x)}$, each $\FCol'_i$ is non-zero.
\end{lemma}
Observe that under the uniform distribution, the probability that $\mE(x)=1$ is doubly-exponentially small in $\ell$. Indeed, using  Claim~\ref{claim:pv-large}, $\mE(x)=1$ with probability at most $(1-2^{-2(\ell/2+1)})^{20^{\ell/2}}$.
% and has at most $\ell/2$ colliding layers, and $\Pr_{x}[\mE(x) =1] \ll 2^{-\ell}$.

	
	
We are now ready to prove the main lemma, which constitutes to one step of the generator.

\begin{lemma}[Main Lemma]\label{lemma:main}
Let $\ell, k \in \N$ and $m = 20^\ell$.
Let $f: \pmone^n \to \pmone$.
Let $B = D_1 \circ D_2 \circ \ldots \circ D_m$ be a 3ROBP with at most $\ell$ colliding layers in each chunk.
For $i= 1, \ldots, k$, let $E^{(i)} = \bar{\FCol^{(i)}_1} \wedge \ldots \wedge \bar{\FCol^{(i)}_{m}}$ be an error term with all $\FCol^{(i)}_j$ non-zero.
Suppose that whenever all $E^{(i)}(x)=0$ then $B(x) = f(x)$.
Then, with probability at least $1-(k+1)\cdot 2^{-\ell}$, under the restriction $\rho$ from Claim~\ref{claim:assigning},  there exist
$k^* \in \{k, k+1\}$ new error terms
$\mE^{(i)} = \bar{\FCol'^{(i)}_1} \wedge \ldots \wedge \bar{\FCol'^{(i)}_{\sqrt{m}}}$ for $i=1, \ldots, k^{*}$ and a program	
 $B' = D'_1 \circ D'_2 \circ \ldots \circ D'_{\sqrt{m}}$,
where all $\FCol'^{(i)}_j$ and $D'_j$ are 3ROBPs with at most $\ell/2$ colliding layers, and moreover
\begin{enumerate}
	\item 
	For all $i=1, \ldots, k^{*}$ and $j=1, \ldots, \sqrt{m}$, the 3ROBP $\FCol'^{(i)}_j$ is non-zero.
\item If for every $i=1,\ldots,k^{*}$ it holds that  $\mE^{(i)}(x) =0$, then $B'(x) = f|_{\rho}(x)$.
\end{enumerate}
\end{lemma}
\begin{proof}
We apply a random restriction $\rho$ from Claim~\ref{claim:assigning}.
We say that $\rho$ is good if 
\begin{itemize}
	\item $B|_{\rho}$ can be written as a $D'_1 \circ \ldots \circ D'_{m'}$ where each $D'_i$ is a 3ROBP with the first and last state spaces having width at most $2$ and at most $\ell/2$ colliding layers.
	\item For $i=1, \ldots, k$, the restricted error term $E^{(i)}|_{\rho}$ can be upper bounded by $\mE^{(i)} = \bar{\FCol'^{(i)}_1} \wedge \ldots \wedge \bar{\FCol'^{(i)}_{\sqrt{m}}}$ where each $\FCol'^{(i)}_{j}$ has at most $\ell/2$ colliding layers and is non-zero.
\end{itemize}
By Lemma~\ref{lemma:ell reduces} the first event happens with probability at least $1-2^{-\ell}$.
By Lemma~\ref{lemma:error under restriction} the second event happens with probability at least $1-k\cdot 2^{-\ell}$.
	
We prove the two items:
\begin{enumerate}
\item
We show how to approximate $B|_{\rho}$ by a shorter program and a possibly new error term. 
First, by Claim~\ref{claim:decomposition} we can assume without loss of generality that each $D'_{i}$, except for maybe $D'_1$, is colliding.
If $m'\le \sqrt{m}$ we take $B' = B|_{\rho}$ and we do not introduce any new error term. In other words, we take $k^{*} = k$. 
If $m'> \sqrt{m}$, let $j = m'-\sqrt{m}+1$ and let $k^{*} = k+1$.
By Lemma~\ref{lemma:short programs plus error term}, $B|_{\rho}$ can be approximated by a shorter program $B' := D'_{j} \circ \ldots \circ D'_{m'}$, where the error term is $\bar{\FCol_{j}} \wedge \ldots \wedge \bar{\FCol_{m'}}$.
The new error term $\mE^{(k+1)}$ is defined as  $\bar{\FCol_{m'-\sqrt{m}+1}} \wedge \ldots \wedge \bar{\FCol_{m'}}$.
Note that in this case since $j\ge 2$, all functions $\FCol_{j}(x), \ldots, \FCol_{m'}(x)$ are non-zero.
This proves Item~1.
\item Whenever $\mE^{(k+1)}(x)=0$ it holds that $B'(x) = B|_{\rho}(x)$. If $\mE^{(1)}(x) = \ldots \mE^{(k)}(x) = 0$, then $E^{(1)}|_{\rho}(x) =  \ldots E^{(k)}|_{\rho}(x) = 0$ making $B|_{\rho}(x) = f|_\rho(x)$. Thus, if both events happens simultaneously, we have  $B'(x) = B|_{\rho}(x) = f|_{\rho}(x)$.\qedhere
\end{enumerate}
\end{proof}

We show that applying the main lemma $O(\log \log n)$ times and then applying BRRY's~\cite{BravermanRRY10} generator on the remaining variables yields a PRG for 3ROBPs.

\begin{theorem}[Main Theorem]\label{thm:main-3ROBP-ordered}
Let $B$ be an ordered 3ROBP of length $n$. Let $0<\eps < o(1/\log\log(n))$.
Then, applying Claim~\ref{claim:assigning} $O(\log\log(n))$ times and then applying BRRY-generator with seed-length 
$O(\log(n)(\log \log (n) +  \log(1/\eps))$
 on the remaining variables yields a PRG $\eps$-fooling $B$.
\end{theorem}
\begin{proof}
We start by applying the pseudorandom assignment from Claim~\ref{claim:assigning} yielding $\rho_0$.
With  probability at least $1-\eps/n$, we can write $B|_{\rho_0}$ as a $D_1 \circ \ldots \circ D_m$ where each $D_i$ has at most $\ell \le 10 \log(n/\eps)$ colliding layers. \Anote{TODO: requires more details}

We repeat Lemma~\ref{lemma:main} for $k = \log(\ell/\log(1/\eps)) = O(\log \log n)$ iterations.
We get restrictions $\rho_1, \ldots, \rho_k$.
Then for $\rho = \rho_0 \circ \rho_1 \circ  \ldots \circ \rho_k$ it holds that $\E_{\rho,x}[B|_{\rho}(x)] = \E_{z}[B(z)] \pm \poly(\eps/n)$.
Furthermore, by Lemma~\ref{lemma:main}, with probability at least 
$$
1-2^{-\ell} - 2\cdot 2^{-\ell/2} - 3\cdot 2^{-\ell/4} - \ldots  - k\cdot 2^{-\ell/2^{k-1}} \ge 1-2k\cdot 2^{-\ell/2^{k-1}} \ge 1-\eps^2 \cdot O(\log \log n) \ge 1-\eps/10
$$
over the choice of $\rho$, we can write
$B|_{\rho}(x)$ as $B'(x) + E^{(1)}(x) + \ldots + E^{(k)}(x)$  where 
\begin{itemize}
	\item $B'$ is a 3ROBP of length $n$ with at most $20^{\ell/2^{k}}\cdot \ell/2^{k}$ colliding layers, 
	\item Each $E^{(1)}, \ldots, E^{(k)}$ is computed by the AND of $20^{\ell/2^{k}}$ non-zero 3ROBP with at most $\ell/2^{k}$
	 colliding layers. 
	%\item The probability that under the uniform distribution $E^{(i)}(x) = 1$ is at most $1/\exp(\exp(\Omega(\ell/2^{k}))) \ll \frac{\eps}{10k}$.
\end{itemize}
Call such a restriction {\sf good}.
Note that each $E^{(i)}$ can be computed by a width-4 ROBP with at most $20^{\ell/2^{k}} \cdot \ell/2^{k} \le \poly(1/\eps)$ colliding layers.
%Recall that $B'$ is a width-3 ROBP with at most $20^{\ell/2^{k}} \cdot \ell/2^{k} \le \poly(1/\eps)$ colliding layers.
We then apply Corollary~\ref{cor:BRRY} to $B'$ and $E^{(1)}, \ldots, E^{(k)}$ to see that the BRRY-generator $\eps/10k$-fools $B'$ and $E^{(1)}, \ldots, E^{(k)}$ using seed-length 
$$O(\log \log n + \log(10k/\eps) + \log(\poly(1/\eps)) + 4) \log n = O(\log\log n + \log(1/\eps))\cdot \log(n).$$

By Claim~\ref{claim:pv-large}, the probability that the error terms equal $1$ under the uniform distribution is at most $1/\exp(\exp(\Omega(\ell/2^{k}))) \ll \frac{\eps}{10k}$. Thus, for any good restriction $\rho$, we get that when $x$ is sampled from the BRRY-generator, the probability that $E^{(1)}(x) + \ldots + E^{(k)}(x) \neq 0$ is at most $\eps/5$.
Overall we get
%with probability at least $1-\eps/5$, we have that $B'(x) = B|_{\rho}(x)$.
%And
\begin{align*}
\E_{\rho, x\sim \BRRY}[B|_{\rho}(x)] 
&= \E_{\rho, x\sim \BRRY}[B|_{\rho}(x)| \rho\text{ is good}] \pm \eps/10\tag{most $\rho$'s are good}\\
&= \E_{\rho, x\sim \BRRY}[B'(x)| \rho\text{ is good}] \pm (\eps/5 + \eps/10)\tag{BRRY fools the error terms}\\
&= \E_{\rho, z\sim U}[B'(z)| \rho\text{ is good}] \pm (\eps/10k + \eps/5 + \eps/10)\tag{BRRY fools $B'$}\\
&= \E_{\rho, z\sim U}[B|_{\rho}(z)| \rho\text{ is good}] \pm (\eps/10 + \eps/10k + \eps/5 + \eps/10)\tag{the error terms are small under the uniform distribution}\\
&= \E_{\rho, z\sim U}[B|_{\rho}(z)] \pm (\eps/10 + \eps/10 + \eps/10k + \eps/5 + \eps/10)\tag{most $\rho$'s are good}\\
&= \E_{z\sim U}[B(z)] \pm \eps.\tag{$\rho$ maintains the acceptance probability of $B$}
\end{align*}

\end{proof}

}
\subsection{Pseudorandom generator for unordered 3ROBPs}
In this section, using the recent generator of CHHL~\cite{CHHL18}, 
and a Fourier bound from Steinke, Vadhan and Wan~\cite{SteinkeVW14}, 
we show that we can also handle unordered 3ROBPs, thus proving Theorem~\ref{thm:main-3ROBP-unordered}.
\begin{lemma}[Lemma~3.14~\cite{SteinkeVW14}]
	Let $\ell\in \N$ and let $B$ be a width-$w$ ROBP with at most $\ell$ colliding layers. 
	Then, for all $k=1, \ldots, n$ it holds that $L_{1,k}(f) \le O(w^{3}\cdot \ell)^{k}$.
\end{lemma}

\begin{theorem}[Theorem~4.5 \cite{CHHL18}] 
Let $F$ be a family of $n$-variate Boolean functions closed under restrictions. Assume that for all $f\in F$ for all $k=1,\ldots, n$, $L_{1,k}(f) \le a\cdot b^k$.
Then, for any $\eps>0$, there exists an explicit PRG which fools $F$ with error $\eps$, whose seed length is 
$O( \log(n/\eps) \cdot (\log\log(n) + \log(a/\eps)) \cdot b^2)$. 
\end{theorem}
	
\begin{corollary}\label{cor:CHHL}
There is an explicit PRG that $\eps$-fools unordered ROBPs with width $w$ length $n$ and at most $\ell$ colliding layers using seed length
$$O(\log(n/\eps) \cdot (\log \log n + \log(1/\eps)) \cdot w^6 \ell^2 )$$
\end{corollary}

\begin{proof}[Proof of Theorem~\ref{thm:main-3ROBP-unordered}]
The proof is essentially the same as that of Theorem~\ref{thm:main-3ROBP-ordered}, where instead of using the generator from Theorem~\ref{thm:foolfewcollisions} to set the bits after the random restriction, we use the generator from the above corollary. The final seed-length as a worse dependence on $\delta$ as we need to set $\ell = \poly(1/\delta)$ in the above corollary. 
\end{proof}
\ignore{
\begin{theorem}\label{thm:main-3ROBP-unordered}
Let $B$ be an unordered 3ROBP of length $n$. Let $0<\eps\le 1/o(\log\log(n))$.
Then, applying Claim~\ref{claim:assigning} $O(\log\log(n))$ times and then applying CHHL-generator with seed-length 
$O(\log(n) \log \log(n)\cdot \poly(1/\eps) )$
 on the remaining variables yields a PRG $\eps$-fooling $B$.
\end{theorem}
\begin{proof}
	The proof is similar to that of Theorem~\ref{thm:main-3ROBP-ordered}, replacing the BRRY-generator with the CHHL-generator.
\end{proof}}




%\begin{proposition}[Proposition~3.5\cite{SteinkeVW14}].
%Let $\ell,m \in \N$.
%Let $B = D_1 \circ D_2 \circ \ldots \circ D_m$ be a ROBP where each $D_i$ is a 3ROBP with the first and last state spaces having width at most 2, and at most $\ell$ colliding layers.
%	Then $L_{1,k}(B) \le m \cdot O(\ell)^k$ for all $k$.
%\end{proposition}
%
%\begin{proposition}
%	Let $B = D_1 \wedge \cdots \wedge D_m$ 
%	where each $D_i$ is a 3ROBP with the first and last state spaces having width at most 2, and at most $\ell$ colliding layers. How can we fool it?
%\end{proposition}





%%%%%%%%%%%%%%%%%%%%%%%%%%%%%%%%%%%%%%%%%%%%%%%%%%%
\section*{Acknowledgements}
%%%%%%%%%%%%%%%%%%%%%%%%%%%%%%%%%%%%%%%%%%%%%%%%%%%

\bibliographystyle{alphaabbr}
\bibliography{bibs.bib}


\appendix
\section{Proof of Theorem~\ref{thm:CHRTa}}
\label{app:CHRT}


In this section, we view the Boolean functions computed by branching programs as functions $B: \pmone^n \to \B$. For any set $T \subseteq [n]$, this changes the sum $\sum_{S \subseteq T}{|\hat{B}(S)|}$ by a factor of $2$, which we can afford.

Let $B$ be a ROBP of length $n$ and width $w$. Recall that $V_1, \ldots, V_{n+1}$ denote the layers of vertices in $B$.
For a vertex $v \in V_i$ in the branching program we denote by $B_{\to v}$ the sub-branching program ending in the $i$-th layer and having  $v$ the only accepting state.
We denote by $B_{v \to}$ the sub-branching program starting at $v$ and ending at $V_{n+1}$.
Observe that we may express the function computed by the branching program $B$ as a sum of products of these sub-programs, namely 
\begin{equation}\label{eq:bp-decomposition}
\forall{i\in[n]}: \forall{x\in \pmone^n}: B(x) = \sum_{v\in V_i}{B_{\to v}(x) \cdot B_{v \to}(x)}.	
\end{equation}

The main technical result from \cite{CHRT17} is the following theorem:
\begin{theorem}[\protect{\cite[Thm.~2]{CHRT17}}]\label{thm:main_fourier} Let $B$ be an ordered read-once, oblivious branching program of length $n$ and width $w$. Then, 
	$$\forall{k\in [n]}: \;\;\sum_{s: |s|=k} \abs{\widehat{B}(s)} 
	\le 
	O(\log n)^{wk}\;.$$
	\end{theorem}

We are ready to prove a corollary of this theorem, namely Theorem~\ref{thm:CHRTa}.

\begin{theorem}[Thm.~\ref{thm:CHRTa}, restated]
Let $B$ be a width-$w$ length-$n$ ROBP. Let $\eps>0$, $p \le 1/O(\log n)^w$, $k = O(\log(n/\eps))$, and $\D$ be a $\delta_T$-biased distribution over $[n]$ with marginals $p$, where $\delta_T \le p^{2k}$.
Then,
with probability at least $1-\eps$ over $T\sim \D$, 
\[L_1(\tilde{B}) =  \sum_{S \subseteq T}{ |\hat{B}(S)|} \le O((nw)^3/\eps).\]
\end{theorem}

\begin{claim}\label{claim:midlayers}
For all $\beta>0$, the following holds with probability at least $1-\frac{w^2 \cdot n^3}{\beta}$ over $T$:   for all  $v_0$ and $v$ and $1\le j\leq \min\{2k,n\}$:
\begin{equation}\label{eq:lowerlayers}\sum_{s \subseteq T, |s|=j}{ |\hat{B_{v_0 \to v}}(s)|} \le \frac{\beta}{2^j}.\end{equation}
\end{claim}
\begin{proof}
Fix $v_0$ and $v$. Letting $M$ denote the branching program $B_{v_0 \to v}$ we get
$
\sum_{s:|s|=j} \abs{ \widehat{M}(s) } \leq O(\log n)^{wj}
$ from Theorem~\ref{thm:main_fourier}.
Thus,
\begin{align*}
\E_T\left[ \sum_{s:|s|=j} | \widehat{M}(s)| \cdot \one_{\{s \subseteq T\}}\right] = \sum_{s:|s|=j} 
|\widehat{M}(s)| \cdot \Pr_{T}[s\subseteq T] \leq  O(\log n)^{wj} \cdot (p^j + \delta)\leq \frac{1}{2^j}.
\end{align*}
Finally, we conclude by applying the Markov inequality and a union bound, as there is a total of at most $w^2\cdot n^2$ branching programs $B_{v_0 \to v}$ and at most $n$ choices for $j$.
\end{proof}

Theorem~\ref{thm:CHRTa} follows from the next claim which uses Claim~\ref{claim:midlayers} with $\beta=(nw)^3 /\eps$ and $k =  O(\log (n/\eps))$ that ensure $\frac{w^2 \cdot n^3}{\beta}\leq \eps$ and  $\frac{\beta}{2^k}\leq \frac{\eps}{nw}$.
Indeed, with probability at least $1-\eps$, the spectral-norm of $\tilde{B}$ is at most $1+\sum_{j=1}^{k}{\frac{\beta}{2^j}} + (n-k) \cdot \frac{\eps}{nw} \le 2+\beta$.
%
\begin{claim}
Suppose that $T$ is such that the events in Claim~\ref{claim:midlayers} hold for $\beta, k$ such that $\beta/2^k \le \eps/(nw)$. 
Then for every $j$ such that $k\le j \le n$, 
\begin{equation}\label{eq:higherlayers}\sum_{s \subseteq T, |s|=j}{ |\hat{B}(s)|} \le \frac{\eps}{nw}.\end{equation}
\end{claim}
\begin{proof}
We  prove by induction on $j$ that Eq.~\eqref{eq:higherlayers} holds for all $B_{\to v}$, for any $\ell\in [n+1]$ and $v\in V_\ell$. Note that $B$ itself is of the form $B_{\to v}$ for $v$ being the accept node in the final layer (w.l.o.g. there exists only one such node).   
The case $k \le j \le 2k$ is handled by Claim~\ref{claim:midlayers}, since $\sum_{s\subseteq T:|s|=j}|\hat{B_{\to v}}(s)|  \le	 \frac{\beta}{2^j} \le \frac{\beta}{2^k} \le \frac{\eps}{(nw)^2}$. 
For $j > 2k$ we have: 
\begin{align*}
\sum_{s\subseteq T: |s|=j} |\widehat{B_{\to v}}(s)| 
&\le \sum_{i\in T \cap [\ell]} \sum_{v_0\in V_i} \;\;\sum_{\substack{s_0\subseteq T \cap \{1,\ldots,i-1\}:\\ |s_0|=j-k}}\;\;\sum_{\substack{s_1\subseteq T \cap \{i,\ldots, \ell \}:\\ |s_1|=k, i\in s_1}}\;\; \vert \widehat{B_{\to v_0}}(s_0)\cdot \widehat{B_{v_0\to v}}(s_1)\vert \tag{by \cref{eq:bp-decomposition}}\\
& \le \sum_{i\in T \cap [\ell]} \sum_{v_0\in V_i} \Big(\sum_{\substack{s_0\subseteq T \cap \{1,\ldots,i-1\}:\\ |s_0|=j-k}}\vert\widehat{B_{\to v_0}}(s_0)\vert\Big)\cdot \Big(\sum_{\substack{s_1\subseteq T \cap \{i,\ldots, \ell\}:\\ |s_1|=k, i\in s_1}} \vert \widehat{ B_{v_0\to v}}(s_1)\vert \Big) \\
&\le \sum_{i\in T \cap [\ell]} \sum_{v_0\in V_i} 
\frac{\eps}{nw}
\cdot 
\frac{\eps}{nw}
\le \frac{\eps}{nw} \tag{induction and Claim~\ref{claim:midlayers}}
\end{align*}
This completes the induction, and hence the claim follows.
\end{proof}


%%%%%%%%%%%%%%%%%%%%%%%%%%%%%%%%%%%%%%%%%%%%%%%%%%%%%%%%%%%

\section{Restatement of XOR-lemma for functions fooled by small-biased spaces}
\label{app:GMRTV}
In this section we show how Lemma~\ref{lemma:3.2} is a restatement of Thm~4.1 in \cite{GopalanMRTV12}. We recall the following equivalence between having sandwiching approximations with small spectral-norm and being fooled by every small-biased distribution.
\begin{lemma}[\cite{DETT10}]\label{lem:DETT}
Let $f: \pmone^n \to \R$ be a function. Then, the following hold for every $0 < \eps < \delta$:
\begin{itemize}
	\item If $f$ has $\delta$-sandwiching approximations of spectral-norm at most $\delta/\eps$, then for every $\eps$-biased distribution $D$ on $\pmone^n$, $|\E_{x\sim D}[f(x)] - \E[f]| \le \delta$.
	\item If for every $\eps$-biased distribution $D$ on $\pmone^n$, $|\E_{x\sim D}[f(x)] - \E[f]| \le \delta$, then $f$ has $(2\delta)$-sandwiching approximations of spectral-norm at most $1+\delta/\eps$.
\end{itemize}
\end{lemma}
We recall \cite[Thm.~4.1]{GopalanMRTV12}.
\begin{theorem}[\protect{\cite[Thm.~4.1]{GopalanMRTV12}}]\label{thm:GMRTV:4.1}
 	Let $F_1, \ldots, F_k : \pmone^n \to [0,1]$ be functions on disjoint input variables such that each $F_i$ has $\delta$-sandwiching approximation of spectral-norm at most $t$.
 	Let $H:[0,1]^k \to [0,1]$ be a multilinear function in its inputs.
 	Let $h: \pmone^n \to [0,1]$ be defined as $h(x) = H(F_1(x), \ldots, F_k(x))$. Then $h$ has $(16^k\delta)$-sandwiching approximations of spectral-norm at most $4^k(t+1)^k$. 	
\end{theorem}

%We include a proof for completeness.
%\begin{proof}
%	We write $H(x) = \sum_{z\in \B^k}H(z) \cdot \prod_{i:z_i=1}(x_i) \cdot \prod_{i:z_i=1} (1-x_i)$.
%	We prove that $h(x) = H(F_1, \ldots, F_k)$ has as an upper sandwiching approximation with small spectral-norm.
%	We take
%	$$u(x) = \sum_{z \in \B^k} H(z) \prod_{i:z_i=1}(u_i(x)) \cdot \prod_{i:z_i=1} (1-\ell_i(x)).$$
%	For any $z\in \B^k$ we have $$\prod_{i:z_i=1}(F_i(x)) \cdot \prod_{i:z_i=1} (1-F_i(x)) \le \prod_{i:z_i=1}(u_i(x)) \cdot \prod_{i:z_i=1} (1-\ell_i(x)),$$ and since all $H(z) \ge 0$ we get $h(x) \le u(x)$.
%	We bound the expected difference between the two monomials. Let $\mu_i = \E[F_i]$ for $i=1, \ldots, k$. Then,
%	\begin{align*}
%	\E_{x}&\left[\prod_{i:z_i=1} u_i(x) \cdot \prod_{i:z_i=1} (1-\ell_i(x)) - \prod_{i:z_i=1} F_i(x) \cdot \prod_{i:z_i=1} (1-F_i(x))\right]\\
%	&\le \prod_{i:z_i=1} (\mu_i + \delta) \cdot \prod_{i:z_i=0} (1-\mu_i + \delta)  -  \prod_{i:z_i=1} (\mu_i ) \cdot \prod_{i:z_i=0} (1-\mu_i) \le (1+\delta)^k - 1
%	\end{align*}
%	where the last inequality can be proved by induction on $k$ noting that all $\mu_i$ and $1-\mu_i$ are in the interval $[0,1]$. That is, we use the fact that for all $k\in \N$ and any $\alpha_1, \ldots, \alpha_k\in [0,1]$
%	$$(\alpha_1 +\delta)\cdots (\alpha_k+ \delta) - \alpha_1 \cdots \alpha_k \le (1+\delta)^k - 1.$$
%		We can assume without loss of generality that $\delta k \le 1$ (as otherwise the conclusion would trivially hold with $u \equiv 1$) and get that $((1+\delta)^k-1) \le 2\delta k$.
%	Overall we got an upper sandwiching approximation with error at most $2^{k} \cdot 2\delta k$ and spectral-norm at most $2^k \cdot (t+1)^k$.
%	
%	To get a lower sandwiching approximation, we take an upper sandwiching approximation for $G(F_1, \ldots, F_k)$ where $G(x) = 1-H(x)$. Let $u'(x)$ be this polynomial.
%	By the same analysis $\E_{x\sim U_n}[u'(x) - G(F_1(x), \ldots, F_k(x))] \le 2^{k} \cdot 2\delta k$ and $u'$ has $L_1$-norm at most $2^k \cdot (t+1)^k$.
%	Finally we take $\ell(x) = 1-u'(x)$. We have that $\ell(x) \le h(x)$ for all $x\in \pmone^n$, the spectral-norm of $\ell$ is at most $1+2^k \cdot (t+1)^k$ and $\E_{x\sim U_n}[h(x) - \ell(x)]\le 2^{k} \cdot 2\delta k$.
%\end{proof}




We translate the domain $[0,1]$ to $[-1,1]$ to get  a restatement of the previous theorem.
\begin{theorem}[\protect{\cite[Thm.~4.1]{GopalanMRTV12}, $\pm1$-version}]\label{thm:GMRTV:4.1-new-version}
 	Let $F_1, \ldots, F_k : \pmone^n \to [-1,1]$ be functions on disjoint input variables such that each $F_i$ has $\delta$-sandwiching approximation of spectral-norm at most $t$.
 	Let $H:[-1,1]^k \to [-1,1]$ be a multilinear function in its inputs.
 	Let $h: \pmone^n \to [-1,1]$ be defined as $h(x) = H(F_1(x), \ldots, F_k(x))$. Then $h$ has $(16^k\delta)$-sandwiching approximations of spectral-norm at most $2^{k+1}(t+4)^k$.
\end{theorem}
\begin{proof} We take $F'_1, \ldots, F'_k$ to be $\frac{F_1+1}{2}, \ldots, \frac{F_{k}+1}{2}$ respectively.
We get that $F'_i$ has $\delta/2$-sandwiching approximations of spectral-norm at most $(t+1)/2$, for all $i\in \{1,\ldots, k\}$.
We take $H':[0,1]^k\to [0,1]$ to be 
$H'(y_1, \ldots, y_k) = \frac{1+H(2y_1 - 1, \ldots, 2y_k-1)}{2}$.
Since $H$ is multilinear, so is $H'$.
By Theorem~\ref{thm:GMRTV:4.1}, we get that $H'(F'_1, \ldots, F'_k)$ 
has $(16^{k} \cdot \delta/2)$-sandwiching approximations of spectral-norm at most $4^k(\frac{t+1}{2}+1)^k$.
Since $H(F_1, \ldots, F_k) = 2\cdot H'(F'_1, \ldots, F'_k) - 1$
we got that $H$ as a $(16^{k} \cdot \delta)$-sandwiching approximations of spectral-norm at most $1+2\cdot4^k(\frac{t+1}{2}+1)^k = 1+2\cdot 2^{k}(t+3)^k \le 2^{k+1} \cdot (t+4)^k$.
\end{proof} 
	
Finally, we restate Lemma~\ref{lemma:3.2} and prove it.
\begin{lemma}
	Let $0 < \eps< \delta\le 1$.
 	Let $F_1, \ldots, F_k : \pmone^n \to [-1,1]$ be functions on disjoint input variables such that each $F_i$ is $\delta$-fooled by any $\eps$-biased distribution.
 	Let $H:[-1,1]^k \to [-1,1]$ be a multilinear function in its inputs.
 	Then $H(F_1(x), \ldots, F_k(x))$ is $(16^k \cdot 2\delta)$-fooled by any $\eps^k$-biased distribution.
\end{lemma}
\begin{proof}
	Using the second item in Lemma~\ref{lem:DETT}, since $F_1, \ldots, F_k$ are $\delta$-fooled by any $\eps$-biased distribution, we have that there exist $2\delta$-sandwiching approximations of spectral-norm at most $1  + \delta/\eps$.
	Thus by Thm.~\ref{thm:GMRTV:4.1-new-version}, $H(F_1, \ldots, F_k)$ has $ (16^k \cdot 2 \delta)$-sandwiching approximations of spectral-norm at most $2^{k+1} \cdot (\delta/\eps + 5)^k$.
	Set $\delta' := 16^k \cdot 2 \delta$ and $\eps' := \delta'/(2^{k+1} \cdot (\delta/\eps + 5)^k)$. Then, $H(F_1, \ldots, F_k)$ has $ \delta'$-sandwiching approximations of spectral-norm at most $\delta'/\eps'$.
	Using the first item in Lemma~\ref{lem:DETT} (noting that $\eps'<\delta'$), any $\eps'$-biased distribution $\delta'$-fools $H(F_1, \ldots, F_k)$.
	A small calculation shows that $\eps' \ge \eps^k$, hence any $\eps^k$-biased distribution also $\delta'$-fools $H(F_1, \ldots, F_k)$.
	\end{proof}
	% A small calculation:
	%eps' = 16^k * 2 delta/(2^{k+1} * (delta/eps+5)^k) 
	%\ge 8^k * delta / (6delta/eps)^k = (eps*8/6)^k * delta^{-(k-1)} >= eps^k
	%%%%%%%%%%%%


%\section{Deferred Proofs from Section~\ref{sec:assign_all}}\label{sec:aggressive}







\section{Pseudorandom restrictions for the composition of 3ROBPs}\label{app:composition 3ROBPs}
We restate and prove Lemma~\ref{lemma:fooling H of width-3}.
\begin{lemma}
Let $f_1, \ldots, f_k$ be 3ROBPs on disjoint sets of variables of $[n]$. 
Let $H:\pmone^k \to \pmone$ be any Boolean function.
Then, $f = H(f_1, f_2, \ldots, f_k)$ is $\eps\cdot ((n+k)/k)^k$-fooled by the pseudorandom partial assignment in Theorem~\ref{thm:main_two_steps}.
\end{lemma}
\begin{proof}
Let $V(f_1), \ldots, V(f_k)$ be the sets of variables on which $f_1, \ldots, f_k$ depend.
We write $H$ in the Fourier basis:
$H(y_1, \ldots, y_k) = \sum_{S\subseteq [k]} \hat{H}(S) \cdot \prod_{i\in S}{y_i}$.
Thus,
$H(f_1(x), \ldots, f_k(x)) = \sum_{S\subseteq [k]} \hat{H}(S) \cdot \prod_{i\in S}{f_i(x)}$.
Recall that the pseudorandom assignment in Theorem~\ref{thm:main_two_steps} is composed of two stages:
Let $\eps_1 = \eps/2$ and $\eps_2 = \eps/2n$.
\begin{enumerate}
	\item Pick $T_0 \subseteq [n]$ using a $(\eps_1/n)^{10}$-biased distribution with marginals $1/2$.
	\item Assign the coordinates in $[n]\setminus T_0$ uniformly at random.
	\item
	\begin{enumerate}
	\item Pick $T\subseteq T_0$ using a $\delta_T$-biased distribution with marginals $p = 1/O(\log \log ( n/\eps_2))^{6}$.
	\item  Assign the coordinates in $T_0\setminus T$ uniformly at random.
	\item Assign the coordinates in $T$ using a $(\eps_2/n)^{O(\log \log (n/\eps_2))}$-biased distribution $\Dx$.
	\end{enumerate}
\end{enumerate}
Recall that for a fixed $T_0$,  the bias-function of the program behaves the same under any relabeling of the layers in $[n]\setminus T_0$. We imagine as if these layers are relabeled so that a collision is possible, and denote this relabeled program by $f_i^{T_0}$.
We have $\Bias_{T_0}(f_i)(x) = \E_{y\sim U_{[n]\setminus T_0}}[(f_i^{T_0})_{T_0|y}(x)]$ and similarly
since the sets $V(f_1), \ldots, V(f_k)$ are disjoint
$\Bias_{T_0}(H(f_1, \ldots, f_k))(x) = \E_{y\sim U_{[n]\setminus T_0}}[H((f_1^{T_0})_{T_0|y}(x), \ldots, (f_k^{T_0})_{T_0|y}(x))]$.
By Theorem~\ref{thm:the-bias-trick} and~\ref{thm:BDVY}, with  probability at least $1-\eps_1\cdot k$ the choice of $T_0$ and $y$, we can write each $(f_i^{T_0})_{T_0|y}(x)$ for $i=1, \ldots, k$ as a linear combination of $\prod_{j \in [m_i]}[f_{i,j}]$ where the sum of coefficients in absolute value is at most the number of variables in $f_i$ (i.e., $|V(f_i)|$), and each $f_{i,j}$ is a ROBP on at most $O(\log(n/\eps))$ bits.
Overall with high probability over $T_0, y$ the product $\prod_{i\in S} (f_i^{T_0})_{T_0|y}$ can be written as a linear combination of the functions $\prod_{i\in S} \prod_{j \in [m_i]}[f_{i,j}]$ where the sum of coefficients in absolute values in the linear combination is at most $\prod_{i\in S}{|V(f_i)|}$.
Thus, $H((f_1^{T_0})_{T_0|y}, \ldots (f_k^{T_0})_{T_0|y})$ can be written as a linear combination of XOR of $O(\log(n/\eps))$-length width-3 ROBPs where the sum of coefficients is a most
$$
\sum_{S\subseteq [k]} |\hat{H}(S)| \cdot \prod_{i\in S}|V(f_i)| 
\le \sum_{S\subseteq [k]} 1 \cdot \prod_{i\in S}|V(f_i)| 
= \prod_{i=1}^{k}(1+|V(f_i)|) 
\le ((n+k)/k)^k
$$
where in the last inequality we used the fact that the sets $V(f_1), \ldots, V(f_k)$ are disjoint along with a convexity argument.

By Theorem~\ref{thm:main}, each XOR of $O(\log(n/\eps))$-length width-3 ROBPs is $\eps_2$-fooled by the pseudorandom assignment defined by Step~3 above, thus the overall error is at most $\eps_1\cdot k + \eps_2 \cdot ((n+k)/k)^k \le \eps\cdot ((n+k)/k)^k$.
\end{proof}




\end{document}

