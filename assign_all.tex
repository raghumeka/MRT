\newcommand{\VBad}{\mathsf{VarBad}}
\newcommand{\V}{\mathsf{Var}}
\newcommand{\Bad}{\mathsf{Bad}}
\newcommand{\Good}{\mathsf{Good}}


\newcommand{\GMany}{\mathbf{G_{\oplus Many}}}
\newcommand{\GXOR}{\mathbf{GXOR}}




\section{Assigning all the variables: a pseudorandom generator for the XOR of short ROBPs}\label{sec:assign_all}

In Theorem~\ref{thm:main}, we proved that we can pseudorandomly assign $p$-fraction of the coordinates of $f(x) = \prod_{i=1}^{m}{f_i(x)}$, while maintaining its acceptance probability up to an additive error  of $\eps$, using $\tilde{O}(\log(n/\eps)\cdot \log(b))$ random bits.
In this section, we will construct a pseudorandom generator $\eps$-fooling $f$, by applying Theorem~\ref{thm:main} $\poly\log\log(n/\eps)$ times, combined with Lovett's \cite{Lovett08} or Viola's \cite{Viola08} pseudorandom generator for low-degree polynomials, and CHRT's pseudorandom generator for constant-width ROBPs~\cite{CHRT17}. Our main result is:
\begin{thm}\label{thm:GXOR}
Let $n,w,b\in \N$, $\eps>0$. There exists an explicit pseudorandom generator that $\eps$-fools any XOR of ROBPs of  width-$w$ and length-$b$ (defined on disjoint sets of variables),using seed-length $O(\log(b) + \log\log(n/\eps))^{2w+2} \cdot \log(n/\eps)$.
\end{thm}

\paragraph{Assigning $0.9999$-fraction of the variables.}
The first step is rather standard. 
By making $t$ recursive calls to Theorem~\ref{thm:main} we can assign all but $(1-p)^t \le e^{-pt}$ fraction of the coordinates while maintaining the acceptance probability.
Note that we rely on the fact that under restrictions, the restricted function is still of the form $\prod_{i=1}^{m}{g_i(x)}$ where each $g_i$ is a ROBP of width $w$ that depends on at most $b$ bits (in other words, the class of functions we are trying to fool is closed under restrictions).
Setting $t = O(1/p)$, we can assign $0.9999$ fraction of the inputs bits while changing the acceptance probability by at most $\eps/n$.

\begin{claim}\label{claim:assigning-0.9999}
Let $f = \prod_{i=1}^{m} f_i(x)$, with block-length $b$.
Then, there is a pseudorandom restriction fixing each variable with probability $0.9999$, using at most $
 O\big(\log(b) +\log\log(n/\eps)\big)^{2w+1} \cdot \log(n/\eps)$ random bits, and changing the acceptance probability by at most $(\eps/n)$.

%Furthermore, any fixed set $S\subseteq[n]$ of  $k \le 5\log(n/\eps)$ variables remains alive with probability at most $2^t \cdot 0.0001^k$.
Furthermore, any fixed set $S\subseteq[n]$ of  $k \le 5\log(n/\eps)$ variables remains alive with probability at most $2 \cdot 0.0001^k$.
\end{claim}
\begin{proof}[Proof Sketch]
Apply Theorem~\ref{thm:main} with error $\eps/n^2$ for $t = \log(0.0001)/\log(1-p) = O(1/p)$ times iteratively, with independent random bits per each iteration. This changes the probability of acceptance by at most $(\eps/n^2)\cdot t \le \eps/n$ and keeps each variable alive with probability $(1-p)^{t} =0.0001$.
The amount of random bits used to sample the restriction is 
$$O(p^{-1}
 \cdot w  \log (n/\eps)  (\log\log(n/\eps) +  \log(b))
 \le 
 O\big(\log(b) +\log\log(n/\eps)\big)^{2w+1} \cdot \log(n/\eps).$$

Next, we show the furthermore part. Let $S$ be a fixed set of $k$ variables. By Claim~\ref{claim:inclusion-exclusion} the probability that $S$  remains alive under the $t$ pseudorandom restrictions is at most $((1-p)^{k} + 2^{k}\delta_T)^t$. 
Recall that $\delta_T = (n/\eps)^{\omega(1)}$.
For $k \le 5 \log(n/\eps)$, we get 
$((1-p)^{k} + 2^{k} \delta_T)^t \le (1-p)^{kt} \cdot (1+4^{k}\delta_T)^{t} \le 2\cdot 0.0001^{k}$.
\end{proof}


We would like to claim that $f = \prod_{i=1}^{m}f_i$ simplifies after assigning $0.9999$ of the coordinates. 
For a particular function $f_i$, with high probability, at least $1-1/32^b$, the block length decreases under a random restriction by a factor of $2$.
This is due to the fact that on expectation  at most $0.0001\cdot b$ of the variables will survive, and we can apply Chernoff's bound.
Now, if $m\le 16^b$, we can apply a union bound and get that with high probability the block-length decreases by a factor of $2$ in all functions $f_1, \ldots, f_m$ simultaneously.
We seem to have been making progress, going from block-length $b$ to block-length $b/2$, and we might hope that $\log(b)$ iterations of Claim~\ref{claim:assigning-0.9999} are enough to get a function that depends on $O(1)$ many variables (which is easy to fool). 
But, in order to carry the argument, even in the second step, we need to be able to afford the union bound on all functions. 
Ideally, the number of functions that are still alive also decreases from at most $16^b$ to at most $16^{b/2}$, and a similar union bound works replacing $b$ by $b/2$. We can continue similarly as long as in each iteration the block-length decreases by half and the number of functions by a square root.

We run into trouble if at some iteration we have more than $16^{b'}$ functions of block-length $b'$.
The first observation is that in this case the total variance of the functions is extremely high, exponential in $b'$. Recall that the expected value of the product is exponentially small in the total variance. This means that the expected value of the product is doubly-exponentially small in $b'$.
The second observation is that under $(1-\alpha)$-random restrictions, on average, the total variance decreases by a factor of $\alpha$.
Hence, we aim to apply a pseudorandom restriction assigning $(1-\exp(-b'))$ fraction of the variables alive, while keeping the total variance higher than $\log(n/\eps)$. This restriction is extremely aggressive, keeping only a polynomial fraction of the remaining variables alive (compared to say a constant fraction in Claim~\ref{claim:assigning-0.9999}).
However, we claim that in this case, such a restriction maintains the total variance high and thus the expected value of $\prod_{i=1}^{m}{f_i(x)}$ small (at most $\poly(\eps/n)$) in absolute value.

The nice thing about these ``aggressive pseudorandom restrictions'' is that they keep variables alive with such small probability that with high probability each function $f_i$ will depend on at most $O(1)$ variables after the restriction, except for a small number of functions covering at most $O(\log(n/\eps))$ ``bad variables''. This will allow us to fool the restricted function using Lovett's \cite{Lovett08} or Viola's \cite{Viola08} pseudorandom generator for low-degree polynomials.
In the next section, we explain how to handle this case in more details. 
Then, in Section~\ref{sec:4.2} we describe as a thought experiment a ``fake PRG'': an adaptive process that fools the XOR of short ROBPs, but depends on the function being fooled. In Section~\ref{sec:4.3} we show how to eliminate the adaptiveness and construct a true PRG for this class of functions. 

\subsection{PRG for the XOR of many functions with block-length $b$}\label{sec:4.1}

Let $\mathcal{F}_{b,n,t}$ be the class of functions of the form
$f(x) = f_0(x) \cdot \prod_{i=1}^{m}{f_i(x)}$
where $f_0, \ldots, f_m$ are Boolean functions on disjoint sets of variables, $f_0$ (the `junta') depends on at most $t$ variables, $f_1, \ldots, f_m$ are {\bf non-constant} and depend on at most $b$ variables and 
$16^b \le m \le 2\cdot 16^{2b}$.

\begin{lemma}\label{lemma:Gb}
There exists a constant $C>0$ such that the following holds.
For all $n,b,t$ such that  $C\cdot \log\log(n/\eps) \le b \le \log(n)$, there exists an explicit pseudorandom generator $\GMany(b,n,t,\eps):\pmone^{O(t + \log n/\eps)} \to \pmone^n$ that $\eps$-fools $\mathcal{F}_{b,n,t}$.
\end{lemma}

%The algorithm will use a universal large enough constant $C'\ge C$ in its definition.
\begin{algorithm}[H]
\caption{The Pseudorandom Generator $\GMany(b,n,t,\eps)$} 
\begin{algorithmic}[1]
\Require{A block-length $b$, the output length $n$, a junta-size $t$, an error parameter $\eps\in (0,1)$}
\State Set $x:= 1^n$
\State Pick $T\subseteq [n]$ using a $(\eps/n)^{10C}$-biased distribution with marginals $2^{-b}$.
\State Assign coordinates of $x$ in $[n]\setminus T$ using a $(\eps/n)^{10C}$-biased distribution.
\State Assign coordinates of $x$ in $T$ using Viola's generator with error $\frac{\eps}{4} \cdot (\frac{\eps}{n})^{C} \cdot 2^{-t}$ and degree~$16$.
\State \Return $x$.
\end{algorithmic}\label{test:1}
\end{algorithm}

\begin{lemma}\label{lemma:aggressive-real}
Let $C>0$ be a sufficiently large constant.
Let $C\cdot \log\log(n/\eps) \le b \le \log(n)$ be some integer.
Let $f_1, \ldots, f_m$ be non-constant Boolean functions that depend on disjoint sets of at most $b$ variables each.
Assume $m \ge 16^b$.
Suppose $T$ is $(\eps/n)^{10C}$-biased distribution with marginals $2^{-b}$.
Suppose $x$ is sampled from a $(\eps/n)^{10C}$-biased distribution.
Then, with  probability at least $1-(\eps/n)^{C/4}$, at least $4^b$ of the functions $(f_i)_{T|x}$ will be non-constant. 
\end{lemma}

%\begin{lemma}[Lemma~\ref{lemma:aggressive-real}, restated]
%Let $C>0$ be a sufficiently large constant.
%Let $C\cdot \log\log(n/\eps) \le b \le \log(n)$ be some integer.
%Let $f_1, \ldots, f_m$ be non-constant Boolean functions that depend on disjoint sets of at most $b$ variables each.
%Assume $m \ge 16^b$.
%Suppose $T$ is $(\eps/n)^{10C}$-biased distribution with marginals $2^{-b}$.
%Suppose $x$ is sampled from a $(\eps/n)^{10C}$-biased distribution.
%Then, with  probability at least $1-(\eps/n)^{C/4}$, at least $4^b$ of the functions $(f_i)_{T|x}$ will be non-constant. 
%\end{lemma}
\begin{proof}
Without loss of generality $m = 16^b$.
Let $k = C \log(n/\eps)/b$, and note that $k \le 2^b$ since $b\ge C \cdot \log \log(n/\eps)$ for a sufficiently large constant $C>0$.

Let $B_1, \ldots, B_m\subseteq [n]$ be the disjoint sets of variables on which $f_1, \ldots, f_m$ depend respectively.
For any function $f_i: \pmone^{B_i} \to \pmone$, there exists a sensitive pair of inputs $(\alpha^{(i)},\beta^{(i)}) \in \pmone^{B_i}$ such that $\alpha^{(i)}$ and $\beta^{(i)}$ differ in exactly one coordinate $j_i$ and such that $f_i(\alpha^{(i)}) \neq f_i(\beta^{(i)})$.
We say that the sensitive pair ``survives'' the random restriction defined by $(T,x)$ if both $\alpha^{(i)}$ and $\beta^{(i)}$ are consistent with the partial assignment defined by the restriction (i.e., if they agree with $x$ on $B_i \setminus T$).
For each function $f_1,\ldots, f_m$ fix one sensitive pair $(\alpha^{(1)}, \beta^{(1)}), \ldots, (\alpha^{(m)}, \beta^{(m)})$ and denote by $\cE_1, \ldots, \cE_m$ the events that these sensitive pairs survive.
Next, we claim that $\cE_1, \ldots, \cE_m$ are almost $k$-wise independent.
We compare them to the events $\cE'_1, \ldots, \cE'_m$ that indicate whether the sensitive pairs survive under a truly random restriction sampled from $\mathcal{R}_{2^{-b}}$.
Denote by $p_i = \Pr(\cE'_i)$.
Observe that $p_i \ge 2^{1-2b}$ since in order for the pair to survive it is enough that the sensitive coordinate remains alive (happens with probability $2^{-b}$) and that the partial assignment on the remaining coordinates agrees with $\alpha^{(i)}$ (happens with probability at least $2^{1-b}$). %Let $\cE'_1, \ldots, \cE'_m$ be i.i.d. random events happening with probability $p = 2^{1-2b}$.
Then,
	$$
	\E\left[\Big(\sum_{i=1}^{m}{(\one_{\cE_i}-p_i)}\Big)^k\right] \le 
		\E\left[\Big(\sum_{i=1}^{m}(\one_{\cE'_i}-p_i)\Big)^k\right] + (2m)^k \cdot \max_{K\subseteq [m]: |K|\le k} \left|\E[\prod_{i\in K} \one_{\cE_i}] - \E[\prod_{i\in K} \one_{\cE'_i}]\right|.$$ 
	We upper bound the first and second summands separately. 
	By Lemma~\ref{lemma:tail_bounds} and Lemma~\ref{lemma:var}, the first summand is upper bounded by $\max\{k^k, (Vk)^{k/2}\}$ where 
	$$
	V := \sum_{i=1}^{m}{p_i}.
	$$ 
	Since $V \ge m\cdot 2^{1-2b} = 2\cdot 4^b  \ge k$, the first summand is upper bounded by $(Vk)^{k/2}$.
		
	Next, we upper bound the second summand. By Vazirani's XOR lemma, since $x$ is $(\eps/n)^{10C}$-biased, we have that the marginal distribution of
	 any set of at most $k\cdot b$ bits in $x$ is $(\eps/n)^{10C} \cdot 2^{kb/2}$-close to uniform in statistical distance.
	Since $T$ is $(\eps/n)^{10C}$-biased with marginals $2^{-b}$, using Claim~\ref{claim:inclusion-exclusion} we have that the marginal distribution on any set of at most $k$ coordinates in $T$ is $(\eps/n)^{10C} \cdot 4^{k}$-close in statistical distance to the distribution sampled according to $\mathcal{R}_{2^{-b}}$.
	Thus, $|\E[\prod_{i\in K} {\one_{\cE_i}}] - \E[\prod_{i\in K} {\one_{\cE'_i}}]| \le (2^{kb/2} +4^{k}) \cdot (\eps/n)^{10C} \le 2^{kb/2+1} \cdot (\eps/n)^{10C}$ and we get 
	$(2m)^{k} \cdot 2^{kb/2+1} \cdot (\eps/n)^{10C}\le 1$.
	
	Combining the bounds on both summands we get
	$$
	\E\left[\Big(\sum_{i=1}^{m}{(\one_{\cE_i}-p_i)}\Big)^k\right] 
	\;\le\; 
	(Vk)^{k/2} + 1 
	\;\le\; 
	2\cdot (Vk)^{k/2}.
	$$
	Using $V = \sum_{i=1}^{m} p_i$ we get 
	$$\Pr\left[\sum_{i=1}^{m}\one_{\cE_i} \le V/2 \right] \le 
	 \Pr\left[\Big(\sum_{i=1}^{m}{(\one_{\cE_i}-p_i)}\Big)^k  \ge (V/2)^k \right]  \le 2(Vk)^{k/2} \cdot (V/2)^{-k}$$
Using $V \ge 2\cdot 4^b$ and $k\le 2^b$ we get $\Pr\left[\sum_{i=1}^{m}\one_{\cE_i} \le V/2 \right] \le 2(4k/V)^{k/2} \le 2\cdot(2/2^b)^{k/2}\le  (\eps/n)^{C/4}$.
In the complement event, at least $V/2 \ge 4^b$ of the functions $(f_1)_{T|x}, \ldots, (f_m)_{T|x}$ are non-constant.
\end{proof}

%We defer the proof of Lemma~\ref{lemma:aggressive-real} to Appendix~\ref{sec:aggressive}.

\begin{lemma}\label{lemma:low-deg}
Let $f: \F_2^n \to \pmone$.
Suppose $f(x) = h(x) \cdot (-1)^{g(x)}$ where $h$ is a $k$-junta 
and $g$ is a polynomial of degree-$d$ over $\F_2$.
	If $\D$ fools degree-$d$ polynomials over $\F_2$ with error $\eps$, then $\D$ fools $f$ with error $\eps \cdot 2^{k/2}.$
\end{lemma}
\begin{proof}
Let $J$ be the set of variables on which $h$ depends.
Using the Fourier transform of $h$:
$h(x) = \sum_{S \subseteq J} \hat{h}(S) \cdot (-1)^{\sum_{i\in S} x_i}$
we write $f$ as
$f(x) = \sum_{S \subseteq J} \hat{h}(S) \cdot (-1)^{\sum_{i\in S} x_i + g(x)}$.
Note that $\sum_{i\in S} x_i + g(x)$ is a polynomial of degree-$d$ over $\F_2$ as well, thus we get
\begin{align*} \left|\E_{x\sim U}[f(x)]-\E_{x\sim \D}[f(x)]\right| 
&\le \sum_{S}|\hat{h}(S)| \cdot 
\Big|\E_{x\sim U}[(-1)^{\sum_{i\in S} x_i + g(x)}] - \E_{x\sim \D}[(-1)^{\sum_{i\in S} x_i + g(x)}]\Big| \\
&\le L_1(h) \cdot \eps \le 2^{k/2} \cdot \eps\;.\qedhere\end{align*}
\end{proof}


\begin{proof}[Proof of Lemma~\ref{lemma:Gb}]
First note that $f$ has very small expectation under the uniform distribution
$$
\left|\E_{z\sim U}\Big[f_0(z) \cdot \prod_{j=1}^{m} f_j(x)\Big]
\right| \le (1-2^{-b})^{16^{b}} \ll \frac{\eps}{4}.
$$
using the assumption $b\ge C \log \log(n/\eps)$.
Thus, we need to maintain  low-expectancy under the pseudorandom assignment.
By Lemma~\ref{lemma:aggressive-real}, with probability at least $1-(\eps/n)^{C/4}\ge 1-\frac{\eps}{100}$ after the aggressive random restriction at least $4^{b}$ of the functions $f_1, \ldots, f_m$ remain non-constant. Since $b\ge C \log\log(n/\eps)$ we maintained the low-expectancy under aggressive random restrictions. That is, whenever $4^{b}$ of the functions $f_1, \ldots, f_m$ remain non-constant under restriction, the expected value of the restricted function under the uniform distribution is at most $(1-2^{-b})^{4^b} \ll \eps/4$ in absolute value.

Furthermore, we wish to show that with high probability, except for a set of at most $C\log (n/\eps)$ ``bad variables'' all functions have block-length at most $16$.
Recall that there are at most $2\cdot 16^{2b}$ functions.
The probability that any particular $k$ variables survive is at most $2^{-b k} + (\eps/n)^{10C}$.
Pick $k = C \log(n/\eps)/b \le C \log(n/\eps)$.
The probability that at least $k$ variables in at most $\ell$ functions survive is 
$$
\binom{2\cdot 16^{2b}}{\ell} \cdot \binom{\ell \cdot b}{k} \cdot (2^{-b k} + (\eps/n)^{10C})
\le 2 \cdot 2^{\ell + 9b\ell - bk} \le 2^{10b\ell-bk}
$$
If $\ell \le k/16$, then this probability is at most $2^{-6bk/16} = (\eps/n)^{6C/16} \ll \frac{\eps}{100}$.
This means that, with high probability, there are less than $k$ variables from all functions with more than $16$ effective variables remaining.
Otherwise, there would have been $\ell \le k/16$ functions accountable to a total number of more than $k$ variables that remained alive and effective, under the restriction.

Overall, with probability at least $1-\eps/50$ we are left with the XOR of a small-junta, on at most $t + C\log(n/\eps)$ variables, and an XOR of at least $4^b$ non-constant functions on at most $16$ variables (i.e., a degree $16$ polynomial). Moreover, the restricted function has expected value at most $\eps/4$ in absolute value under the uniform distribution.
Using Claim~\ref{lemma:low-deg} we get that Viola's~\cite{Viola08} or Lovett's~\cite{Lovett08} PRG for low-degree polynomials $\eps/4$-fools the remaining function. Combining all estimates we get that the expected value of the restricted function under our distribution is at most $3\eps/4$ in absolute value which completes the proof.
\end{proof}

\subsection{A Thought Experiment}\label{sec:4.2}
We are ready to describe the pseudo-random restriction process in full detail.
We start by describing a process that iteratively ``looks'' at the restricted functions in order to decide which pseudorandom restriction to apply next: the one described in Lemma~\ref{claim:assigning-0.9999} or the one from Lemma~\ref{lemma:aggressive-real}.
This ultimately defines a decision tree of random restrictions.
We then show in Section~\ref{sec:4.3} how to transform the adaptive process into a non-adaptive pseudorandom generator that (by definition) does not depend on the function it tries to fool.
Namely, we would generate a pseudorandom string that  fools the function no matter what path was taken in the decision tree.


We start with $m \le n$ blocks of length $b$.
We assume that $m \le 16^b$ (if not set $b= \log_{16}(m)$).
\begin{algorithm}[H]
\caption{an ``adaptive pseudorandom generator''} 
\begin{algorithmic}[1]
\For{$i=0,1,\ldots$} 
\State Let $b_i = b/2^i$.
\If{$b_i \le C\log \log(n/\eps)$} 
apply CHRT's PRG on the remaining coordinates, and Halt!
\EndIf
\If{more than $16^{b_i}$ of the restricted functions are non-constant and depend on at most $b_i$ variables} 
apply $\GMany(b_i,10\log(n/\eps),n,\eps/2)$ from Lemma~\ref{lemma:Gb} on the remaining variables, and Halt!
\Else{ apply the pseudorandom restriction from Lemma~\ref{claim:assigning-0.9999} on the remaining variables.}
\EndIf
\EndFor
\end{algorithmic}\label{alg:adaptive}
\end{algorithm}

Next, we show that the process yields a pseudorandom string fooling $f = \prod_{i=1}^m{f_i(x)}$.
First, note that the process either stops at Step 3 or at Step 4. In both cases we assign all the variables according to some pseudorandom generator, hence all the variables will be assigned by the end of the process.

For $i=0,1, \ldots, $.
Let $T_i$ be the set of coordinates that remain alive at the beginning of the $i$-th iteration.
Denote by $f_j^{(i)}$ the $j$-th function under the restriction at the beginning of the $i$-th iteration. 
Define $\V[f_j^{(i)}]$ to be the set of variables that affect the output of $f_j^{(i)}$.
For example if $f_j^{(i)}$ is a constant function, then $\V[f_j^{(i)}] = \emptyset$.

Let $\Good_i = \{j:  1 \le |\V[f_j^{(i)}]| \le b_i\}$ be the set of functions that depend on some but not more than $b_i$ variables, $\Bad_i = \{j:  |\V[f_j^{(i)}]| > b_i\}$ be the set of functions that depend on more than $b_i$ variables
%. Let $\Good_{\le i} = \bigcap_{i'\le i}\Good_{i'}$, $\Bad_{\le i}= \bigcup_{i'\le i} \Bad_{i'}$ 
and  $\VBad_{i} = \bigcup_{j\in \Bad_{i} }{\V[f_j^{(i)}]}$.

\begin{claim}\label{claim:VBad}
Let $b_i > C\log \log(n/\eps)$.
Suppose $|\VBad_{i}| \le 10 \log(n/\eps)$ and $|\Good_{i}| \le 16^{b_{i}}$.
Then, with probability at least $1-(\eps/n)$ 
we have $|\VBad_{(i+1)}| \le 10 \log(n/\eps)$.
\end{claim}
\begin{proof}
Under the assumptions we reach Step 5 in Algorithm~\ref{alg:adaptive}. We show that:
\begin{enumerate}
	\item With probability at least $1-\frac{1}{2}(\eps/n)$, at most $5\log(n/\eps)$ of the variables in $\VBad_{i}$ remain alive in Step 5.
	\item With probability at least $1-\frac{1}{2}(\eps/n)$, at most $5 \log(n/\eps)$ new variables are added to $\VBad_{(i+1)}$.
\end{enumerate}
Both claims rely on the fact that  any set of $k \le 5 \log(n/\eps)$ variables remain alive under the pseudorandom restriction in Lemma~\ref{claim:assigning-0.9999} with probability at most $2\cdot 0.0001^k$.

%Both claims rely on the fact that for any set of $k \le 5 \log(n/\eps)$ variables, the probability that all of them survive the pseudorandom restriction in Lemma~\ref{claim:assigning-0.9999} is at most $2\cdot 0.0001^k$.

This  first item follows since the probability that more than $5\log(n/\eps)$ variables in $\VBad_i$ survive is at most 
$$\binom{10\log(n/\eps)}{5\log(n/\eps)} \cdot 2\cdot0.0001^{5\log(n/\eps)} \le \tfrac{1}{2} (\eps/n).$$

As for the second item, we start with the case where $b_i \le 2\log(n/\eps)$.
Assume that more than $5\log(n/\eps)$ new variables were added to $\VBad_{(i+1)}$. 
This implies that there is a set of $k = \lceil{5\log(n/\eps)/(b_i/2)\rceil}$ good functions in step $i$ that are accountable to at least $5\log(n/\eps)$ bad variables in step $i+1$.
The latter event happens with probability at most 
$$\binom{16^{b_i}}{k} \cdot \binom{k b_i}{5\log(n/\eps)} \cdot 2\cdot 0.0001^{5\log(n/\eps)} \le 32^{b_i k} \cdot 2 \cdot 0.0001^{5\log(n/\eps)} \le \tfrac{1}{2} (\eps/n)$$
(where we used $k b_i \le b_i+10\log(n/\eps) \le 12\log(n/\eps)$)
which finishes the case $b_i \le 2\log(n/\eps)$.

In the case where $b_i > 2\log(n/\eps)$, we show that with high probability all good functions remain good.
For each individual function, using Markov's inequality
\begin{align*}\Pr[ |\V[f^{(i+1)}_j]| \ge b_i/2] 
&\le \frac{\E\left[\binom{|\V[f^{(i+1)}_j]|}{ \log (n/\eps)}\right]}{\binom{b_i/2}{\log(n/\eps)}}
\le 2\cdot 0.0001^{\log n/\eps} \cdot \frac{\binom{b_i}{\log(n/\eps)}}{\binom{b_i/2}{\log(n/\eps)}}\\
&\le 2\cdot  0.0001^{\log n/\eps} \cdot \frac{(e \cdot b_i/ \log(n/\eps))^{\log(n/\eps)}}{((b_i/2)/ \log(n/\eps))^{\log(n/\eps)}} \\
&= 2\cdot(0.0001 \cdot e\cdot 2)^{\log(n/\eps)} \le \tfrac{1}{2} (\eps/n^2).\end{align*}
Thus, we can apply a union bound and show that all good functions remain good with  probability at least $1-\frac{1}{2}(\eps/n)$.
\end{proof}


Say the process finished. 
We shall assume that $|\VBad_i| \le 10 \log(n/\eps)$ for every iteration $i$ until the process stopped. By Claim~\ref{claim:VBad} this happens with probability at least $1-\log(b)\cdot(\eps/n) \ge 1-\eps/2$ by applying a union bound on the at most $\log(b)$ iterations.
We wish to show that we constructed a pseudorandom string fooling $f$. We consider two cases:
\begin{enumerate}
	\item 
	We stopped on Step~3 at some iteration $i$. If $i=0$ then $m \le 16^{b_0} \le \poly\log(n/\eps)$ and at most $\poly\log(n/\eps)$ variables remain that affect the functions $f_{j}^{(i)}$.
	 %In this case, by Claim~\ref{claim:VBad}, with high probability, 
	 Otherwise, since $|\Good_{(i-1)}| \le 16^{2b_i} \le \poly\log(n/\eps)$ and $|\VBad_{(i-1)}| \le 10 \log(n/\eps)$, at most $\poly\log(n/\eps)$ variables remain that affect the functions $f_{j}^{(i-1)}$, and thus at most $\poly\log(n/\eps)$ variables remain that affect the functions $f_{j}^{(i)}$.
%	Since $|\VBad_i| \le 10\log(n/\eps)$, and there are at most $16^{2C\log \log(n/\eps)} \le \poly\log(n/\eps)$ ``good'' functions that depend on at most $C\log\log(n/\eps)$ variables. 
%	 Overall, only $\poly\log(n/\eps)$ variables remain that affect the functions $f_{j}^{(i)}$, and 
Thus, we can write $\prod_{j=1}^{m} f_j^{(i)}$ as a ROBP of width $2w$ and length $\poly\log(n/\eps)$, which is $(\eps/2)$-fooled by the pseudorandom generator from Theorem~\ref{thm:CHRT} using  $\tildeO(\log(n/\eps))$ random bits.
	 
\item We stopped at Step~4 at some iteration $i$.
%We wish to bound $|\Good_{i}|$ from above. 
%Let $G = \Good_{i} \cap \Good_{(i-1)}$.
%Since $\Good_{(i-1)} \le 16^{2b_i}$ we have $|G| \le 16^{2b_i}$.
%All functions in $G$ are non-constant and have at most $b_i$ affecting variables.
%Other than $G$ there are at most $|\VBad_{(i-1)}| + |\VBad_{i}| \le 20\log(n/\eps)$ alive variables.
Certainly, $|\Good_{i}|\le  |\Good_{(i-1)}|+|\Bad_{(i-1)}| \le 16^{2b_i} + 10 \log(n/\eps) \le 2\cdot 16^{2b_i}$.
%Note that there are at most $16^{2 b_i}$ good functions in the beginning of the $i$-th iteration, 
%otherwise we would have stopped earlier.
Thus, we are in the case that was handled in Section~\ref{sec:4.1}, with $t \le 10 \log(n/\eps)$. Indeed, Lemma~\ref{lemma:Gb} guarantees that $\GMany(b_i,10\log(n/\eps), n, \eps/2)$ fools the remaining function with error at most $\eps/2$ using $O(\log(n/\eps))$ random bits.
\end{enumerate}


\subsection{The Actual Generator}\label{sec:4.3}
Algorithm~\ref{alg:adaptive} described the pseudo-random generator as if we knew whether or not the condition in step 3 holds. 
However, a pseudorandom generator cannot depend on the function it tries to fool.
To overcome this issue, we use the following general observation regarding pseudorandom generators.

\begin{claim}\label{claim:XOR-of-PRGs}
Say there are two families of functions $\mathcal{F}_1$ and $\mathcal{F}_2$ that are both closed under shifts (i.e., closed under XORing a constant string to the input). Say that $\D_1$ is an $\eps$-PRG for $\mathcal{F}_1$ and $\D_2$ is an $\eps$-PRG for $\mathcal{F}_2$  then $\D_1 \oplus \D_2$ is an $\eps$-PRG for $\mathcal{F}_1 \cup \mathcal{F}_2$.\end{claim}
\begin{proof}
%
Let $f \in \mathcal{F}_1 \cup \mathcal{F}_2$, we show that $\D_1 \oplus \D_2$ fools $f$.
By symmetry assume $f\in \mathcal{F}_1$.
\begin{align*}
\abs{\E_{\substack{x_1\sim \D_1\\ x_2 \sim \D_2}}[f(x_1 \oplus x_2)] -\E_{\substack{z\sim U}}[f(z)]}
&=\abs{\E_{\substack{x_1\sim \D_1\\ x_2 \sim \D_2}}[f(x_1 \oplus x_2)] -\E_{\substack{x_1\sim U\\ x_2 \sim \D_2}}[f(x_1 \oplus x_2)]}\\ 
&\le  \E_{x_2 \sim \D_2} 
\abs{\E_{x_1 \sim \D_1}[f(x_1 \oplus x_2)] -\E_{x_1\sim U}[f(x_1 \oplus x_2)]}\\
&=  \E_{x_2 \sim \D_2} \abs{\E_{x_1 \sim \D_1}[f_{x_2}(x_1)] -\E_{x_1\sim U}[f_{x_2}(x_1)]}\end{align*}
where $f_{y}(x):= f(x\oplus y)$. Since $\mathcal{F}_1$ is closed under shifts, we have that $f_{x_2} \in \mathcal{F}_1$ thus $\D_1$ $\eps$-fools $f_{x_2}$ and we get
$
\E_{x_2 \sim \D_2} \abs{\E_{x_1 \sim \D_1}[f_{x_2}(x_1)] -\E_{x_1\sim U}[f_{x_2}(x_1)]} \le \eps\;.$
\end{proof}

The actual generator would proceed as follows.
\begin{algorithm}[H]
\caption{The Pseudorandom Generator $\GXOR(T, w, b,\eps)$} % (A,m)}
\begin{algorithmic}[1]
\Require a set $T\subseteq [n]$ of the ``live'' coordinates, a width $w$,  an integer $b$, a parameter $\eps\in (0,1)$.
\If{$b \le C\log\log(n/\eps)$} \Return $\mathbf{CHRT}(n, n', 2w,  \eps)|_{T}$ for $n' = 2\cdot 16^{2b}\cdot b +10\log(n/\eps)$
\EndIf
\State Let $x := \GMany(b,t,n,\eps)|_{T}$ for $t = 10 \log(n/\eps)$.
\State Pick $T' \subseteq T$, $y\in \pmone^{T\setminus T'}$ according to Claim~\ref{claim:assigning-0.9999}
\State Let $z  := \GXOR(T', w, b/2,\eps/2)$.
\State \Return  $x \oplus \Sel_{T'}(z,y)$.
\end{algorithmic}
\end{algorithm}


\begin{claim}[Proof of Correctness]\label{claim:correctness}
Let $T \subseteq [n]$.
Suppose $f_1, \ldots, f_m$ are functions on disjoint sets of $T$.
Suppose each function depends on at most $b$ variables except for a total of at most $10 \log(n/\eps)$ variables, and the number of non-constant functions is at most $2 \cdot 16^{2b}$.
Then, $\GXOR(T,w, b,\eps)$ fools $f = \prod_{i=1}^{m} f_i$ with error $\eps$.
\end{claim}
\begin{proof}
We prove the claim by induction on $b$.
	If $b \le C \log \log(n/\eps)$ then Theorem~\ref{thm:CHRT} implies correctness.
	If $b > C \log \log(n/\eps)$ then we
	consider the following two cases:
	\begin{enumerate}
%	{\color{red}
		\item If there are more than $16^b$ good functions, then $x = \GMany(b,t,n,\eps)|_{T}$ fools $\prod_{i=1}^{m}f_i$ with error $\eps$.
%	}
		\item Otherwise, there are at most $16^b$ good functions and we apply Step~3. According to Claim~\ref{claim:assigning-0.9999}, the average acceptance probability of $f_{T'|y}$ is $\eps/4$ close to that of $f$. 
	Furthermore, with  probability at least $1-\eps/4$ all functions $(f_1)_{T'|y}, \ldots, (f_m)_{T'|y}$ depend on at most $b/2$ variables except for at most $10 \log(n/\eps)$ variables (by Claim~\ref{claim:VBad}).
	In such a case, the number of non-constant functions among $(f_1)_{T'|y}, \ldots, (f_m)_{T'|y}$ is at most $16^{b} + 10\log(n/\eps) \le 2\cdot 16^{b}$.
	Using induction, $z = \GXOR(T',w,b/2,\eps/2)$ fools $f_{T'|y}$ with error $\eps/2$, and we get that $\Sel_{T'}(z,y)$ fools $f$ with error $\eps$.

	\end{enumerate}
	Since we have a pseudorandom generator fooling the function in each case,
	 Claim~\ref{claim:XOR-of-PRGs} shows that $x \oplus \Sel_{T'}(z,y)$ fools $f$ with error $\eps$.
	\end{proof}
	
	


\begin{claim}[Seed Length]\label{claim:seed-length}
	The amount of random bits used to calculate $\GXOR([n],w, b,\eps)$ is at most $O(\log(b) + \log\log(n/\eps))^{2w+2} \cdot \log(n/\eps)$.
\end{claim}
\begin{proof}
Unwrapping the recursive calls in the evaluation of $\GXOR([n],w, b,\eps)$ we see that there are at most $\log(b)$ recursive calls to the procedure and that the error parameters are at least $\eps/2^{\log(b)} \ge \eps/n$ in all of them.

We apply the generator from Theorem~\ref{thm:CHRT} only once during these recursive calls, on a ROBP of width-$w$ and length $\poly\log(n/\eps)$. Thus, the application of Theorem~\ref{thm:CHRT} uses at most  $O(\log\log(n/\eps)^{w+2} \log(n/\eps))$ random bits.

The partial assignment from Claim~\ref{claim:assigning-0.9999} uses  at most $O\big(\log(b) + \log\log(n/\eps)\big)^{2w+1} \cdot \log(n/\eps)$ each time we invoke it, and we invoke it at most $\log(b)$ times.

The generator $\GMany$ uses $O(\log (n/\eps))$ random bits each time we invoke it, and we invoke it at most $\log(b)$ times.
\end{proof}

Claims~\ref{claim:correctness} and~\ref{claim:seed-length} completes the proof Theorem~\ref{thm:GXOR} with $\GXOR([n], w, b, \eps)$ as the generator.



\subsection{Pseudorandom generator for read-once polynomials}

\begin{thm}\label{thm:read-once poly}
There exists an explicit $\eps$-PRG for the class of read-once polynomials on $n$ variables with seed-length $O((\log \log(n/\eps))^6 \cdot \log(n/\eps))$.
%$\GXOR([n],2, \log(4n/\eps),\eps/4n)$ fools any read-once polynomial with error at most $\eps$. Its seed length is $O((\log \log(n/\eps))^6 \cdot \log(n/\eps))$.
\end{thm}
\begin{proof}
We show that $\GXOR([n],2, \log(8n/\eps),\eps/8n)$ fools any read-once polynomial with error at most $\eps$. Its seed length is $O((\log \log(n/\eps))^6 \cdot \log(n/\eps))$.

	A read-once polynomial can be written as the XOR of AND functions on disjoint variables, i.e., as the XOR of width-$2$ ROBPs on disjoint variables.
	It remains to show that these ROBPs are short.
	Rather, we show that any PRG that $(\eps/8n)$-fools read-once polynomials of degree at most $b = \log(8n/\eps)$ also $\eps$-fools all read-once polynomials.
Let	$$f(x) = \sum_{i=1}^{m} \prod_{j\in B_i} x_j$$ be a read-once polynomial over $\F_2$, where $B_1, \ldots, B_m$ are disjoint subsets of $[n]$.
Without loss of generality let $B_1, \ldots, B_\ell$ be the blocks of length bigger than $b$. 
Let $$f'(x) = \sum_{i=\ell+1}^{m} \prod_{j\in B_i} x_j,$$ be the sum over monomials of degree at most $b$ of $f$.
Let $\D = \GXOR([n],2, \log(8n/\eps),\eps/8n)$. By triangle inequality
\begin{align}\nonumber\abs{\E_{x\sim \D}[f(x)] -\E_{x\sim U_n}[f(x)]} &\le \Pr_{x\sim \D}[f(x) \neq f'(x)] + \Pr_{x\sim U_n} [f(x) \neq f'(x)] + 
\abs{\E_{x\sim \D}[f'(x)] -
\E_{x\sim U_n}[f'(x)]}\\
&\le \sum_{i=1}^{\ell} \Pr_{x\sim \D}[\wedge_{j\in B_i}(x_j=1)] + \sum_{i=1}^{\ell} \Pr_{x\sim U_n}[\wedge_{j\in B_i}(x_j=1)] + \eps/8n\label{eq:read-once polys}
\end{align}
For $i\in \{1,\ldots, \ell\}$, since $|B_i|\ge b$, we have
$\Pr_{x\sim U_n}[\wedge_{j\in B_i}(x_j=1)] \le 2^{-b} \le \eps/8n$. 
As for the distribution $\D$, by monotonicity 
$$\Pr_{x\sim \D}[\wedge_{j\in B_i}(x_j=1)] \le
\Pr_{x\sim \D}[\wedge_{j\in B'_i}(x_j=1)]$$ where $B'_i$ is any arbitrary subset of exactly $b$ variables from $B_i$. Since $\D$ fools degree-$b$ read-once polynomials with error at most $\eps/8n$, and $\wedge_{j\in B'_i}(x_j=1)$ is such a polynomial, we get that $\Pr_{x\sim \D}[\wedge_{j\in B'_i}(x_j=1)]$  is at most $2^{-b} + \eps/8n \le \eps/4n$.
Plugging both bounds into Eq.~\eqref{eq:read-once polys} we get $\abs{\E_{x\sim \D}[f(x)] -\E_{x\sim U_n}[f(x)]}\le (\eps/4n)\cdot \ell + (\eps/8n)\cdot \ell + \eps/8n \le \eps$.
\end{proof}


